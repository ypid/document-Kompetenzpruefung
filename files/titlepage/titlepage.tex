%% Vorlage für die Titelseite eines Epochenheftes
\newcommand{\URLTitlepageBG}%
{http://commons.wikimedia.org/w/index.php?title=File:Vitruvian-Icon.png&oldid=53430871}
\newcommand{\BackgroundPic}{
\put(0,0){
	\parbox[b][\paperheight]{\paperwidth}{%
		\vfill
		\centering
		\vspace{7cm}
		\href{\URLTitlepageBG}{
		\includegraphics[width=0.6\paperwidth,
			keepaspectratio]{files/titlepage/Vitruvian-Icon}}%
		\vfill
	}
}}
\AddToShipoutPicture{\BackgroundPic}
\begin{titlepage}
	\centering\label{titlepage}
	\vspace*{5ex}
	\huge \begin{tikzpicture}\spiegelschrift{\TITEL}{-0.15}\end{tikzpicture} \\[2cm]
	\LARGE\textsc{\SUBJECT} \\[0.3cm]
	\large Beteiligte Fächer: \SchoolSubjects \\[0.5cm]
	\normalsize
	\ABGABEDATUM \\[5cm]
	\Large\hspace{-8cm}
	\rotatebox{-90}{\printClass} \\
	\vfill
	\large
	\AUTHOR \\[0.1em]
	Betreut von \Betreuer
\end{titlepage}
\ClearShipoutPicture
