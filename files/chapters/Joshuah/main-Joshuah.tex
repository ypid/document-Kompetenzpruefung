\chapter{Joshuah Bärmann}
\label{sec:Joshuah_Baermann}

\section{Definition}

\subsection{Suggestion}
\begin{enumerate}
	\item Suggestion wird in der Medizin für die bewusste oder unbewusste Beeinflussung von Patienten
in mit einem emotionalen Tief eingesetzt. Eine suggestive Methode ist beispielsweise der so genannte
Placebo-Effekt, bei welchem ein in jedem Fall unwirksames Medikament als scheinbar wirksam empfunden
wird. Suggestive Verfahren kommen außerdem zur Anwendung bei Hypnose oder autogenem
Training.\footcite[145]{Frank_Neurologie_und_Psychiatrie}

	\item „Psychischer Vorgang, durch den ein Mensch unter weitgehender Umgehung der rationalen
Persönlichkeitsbereiche dazu gebracht wird, unkritisch (ohne eigene Einsicht) Gedanken, Gefühle,
Vorstellungen und Wahrnehmungen zu übernehmen“.\footcite[514]{Peters_Woerterbuch_der_Psychiatrie}

	\item Im Duden wird Suggestion als die bewusste Beeinflussung eines Menschen beschrieben. Ziel
dieser Beeinflussung ist es, die Person zu einer spezifischen Handlung zu bewegen, ohne dass dies
dieser bewusst ist.\footcite[782]{Duden:Fremdwoerterbuch}

	\item Seelische Selbst- oder Fremdbeeinflussung zur Übernahme von Gedanken, Gefühlen,
Wahrnehmungen oder
Handlungsabsichten.\footcite[217]{Enzyklopaedie_elektrophysiologischer_Untersuchungen}

	\item Der Begriff Suggestion leitet sich vom lateinischen Wort „suggere“ ab, was übersetzt
„einschieben“ oder „eingeben“ bedeutet. Definiert man Suggestion als Interaktion, versteht man unter
dem Begriff die Einflussnahme auf eine Person mit dem Ziel dessen Gedanken, Gefühle oder dessen
Willen und seine Handlungen zu beeinflussen. Rationale Teile der Persönlichkeit werden dabei bewusst
um-gangen. Grundsätzlich ist jeder Mensch in einem bestimmten Ausmaß suggestibel, also durch
Suggestion beeinflussbar. Das Ausmaß der suggestiven Empfänglichkeit einer Person ist dabei
einerseits von Persönlichkeitsmerkmalen wie Urteilsvermögen oder Selbstständigkeit, andererseits aber
auch von Merkmalen wie Alter oder Geschlecht abhängig. Auch die gegenwärtige Situation spielt eine
Rolle, da man beispielsweise unter Hypnose, in Angst oder beim Fehlen von sozialen Kontakten leichter
suggestiv beeinflussbar ist.\footcite[848]{Psychologisches_Woerterbuch}
\end{enumerate}

Der Ausdruck „Suggestion“ ist schon seit dem 17./18. Jahrhundert bekannt und bezeichnet die
manipulative Beeinflussung der Vorstellung oder Empfindung. Diese Beeinflussungsform wird so
praktiziert, dass die Manipulation nicht wahrgenommen wird oder nicht sofort bemerkt wird. Der
Begriff „Suggestion“ ist zurückführbar auf das lateinische Substantiv „suggestio, -onis“ was
Hinzufügung, Eingebung oder Einflüsterung bedeutet oder auf das lateinische Verb „suggerere“ was
zuführen, unterschieben bedeutet.

Die Psychologie versteht unter „Suggestion“ eine Beeinflussung von Fühlen, Denken und Handeln. Der
Psychologe Jamse Braid verwendete diesen Begriff erstmals in diesem Bereich.
Man unterscheidet hier zwischen der Auto- und Heterosuggestion. Also der Beeinflussung durch sich
selbst oder durch andere.
Zu unterscheiden ist jedenfalls zwischen Suggestion als Handlung beziehungsweise Suggestibilität als
jene Person die beeinflusst werden will (Hypnose, Placebo-Effekt etc.).

„Suggestion“ und mehr noch „Suggestibilität“ werden oft auch im Zusammenhang mit
Willensbeeinflussung, Machtausübung, Gutgläubigkeit, Beeinflussbarkeit und Willensschwäche gebracht.

Noch vor der Entwicklung der Psychoanalyse verwendete Josef Breuer Suggestion als Akt, um Hysterie zu
heilen. Die Hysterie wir durch auflegen einer Suggestion bekämpft (z.B. Spinnen sind nicht Böse). Der
betroffenen Person wird eingeredet oder gezeigt, dass die Angst nicht Wirklichkeit ist. Wobei die
Wirkung im Laufe der Zeit nachlässt und die Behandlung wiederholt werden muss.

Weiter alltägliche Effekte, bei denen Suggestion als Erklärungsansatz dienen kann ist:
\begin{itemize}
	\item Placebo-Effekt
	\item Selbstfühlende Prophezeiung (wenn man sich was wünscht, einbildet und es geschiet)
	\item Werbung
\end{itemize}

\subsection{Autosuggestion}
Autosuggestion: eine Person beeinflusst sich selbst („Das schaff ich schon“).

Grundlegend geht es in der Autosuggestion darum, sich selbst mental auf etwas Erwünschtes zu
programmieren.
Programmieren, ist nicht in dem Sinne eines technischen Vorgangs gemeint, sondern als Fähigkeit sich
selbst systematisch und zielsicher zu steuern. Im Unterbewusstsein tun wir dies sowieso schon.

Botschaften, die wir immer und immer wieder hören, fassen wir meist als wahr auf, ohne dass wir ihr
groß Aufmerksamkeit schenken. Diese Fähigkeit unseres Unterbewusstseins funktioniert auch mit unseren
persönlichen Botschaften, dass heißt dass wir uns selbst Manipulieren können. Wenn sie sich z.B. vor
einem Wettbewerb immer wieder sagen, dass Sie niemals gewinnen können, weil Sie viel zu schlecht
sind, werden Sie eher verlieren als gewinnen. Viele Sportler arbeiten mit positiven Botschaften an
sich selbst, feuern sich an und beeinflussen sich so auf den Erfolg.

\subsubsection{Eigentlich nichts Unbekanntes}

Die meisten Menschen beeinflussen sich bereits selbst, allerdings sind sie sich dessen oft nicht
bewusst und leider auch nicht immer zum positiven hin. So reden wir uns z.B. ein, etwas nicht gut
oder gar nicht zu können, „Ich werde das nie lernen, ich bin einfach zu dumm“. Oder man spricht sich
selbst die Aussicht auf einen bestimmten Erfolg ab, „Ich werde eh nicht befördert, die Stelle bekommt
ganz sicher Max Mustermann und nicht ich“. In den meisten Fällen feuert man sich aber selbst
innerlich an, wenn man etwas erreichen will. „Los jetzt du schaffst das“. Oder man macht sich in
einer bedrohlichen Situation Mut.

In der Autosuggestion setzt man diese Methoden systematisch und erfolgreich ein. Wenn man weiß wie.

\subsubsection{Die Autosuggestion kann  Verhaltensweißen und Einstellungen ändern}

Durch gezielte Autosuggestion kann man einige unerwünschte Verhaltensweißen und Einstellungen ändern.
Oft fühlt man sich und seinem Verhalten ausgeliefert und man denk dass man daran nichts ändern kann.
Allerdings haben wir viel mehr Möglichkeiten auf unser eigenes Verhalten und Denken Einfluss zu
nehmen, als wir glauben. Diese Einsicht ist ein großer Schritt zu einem selbstbestimmten Leben. Man
kann beispielsweise Autosuggestion benutzen, ums sich das Rauchen abzugewöhnen oder um ruhig und
gelassen zu werden, um selbstbewusst zu werden oder optimistischer.

Das Unterbewusstsein ist die Quelle der Verhaltensweißen und Einstellungen.
Die meisten Verhaltensweisen und Einstellungen haben ihren Ursprung in unserem Unterbewusstsein. Wenn
man sich also ändern will, klappt das nicht über rationales Vorgehen. Hier gilt die Kräfte des
Unterbewusstseins.

\subsubsection{Was will man mit der Methode der Autosuggestion erreichen?}
Autosuggestion kann man in sehr vielen Bereichen einsetze. Man muss nur dabei beachten, dass man nur
sich selbst per Autosuggestion beeinflussen kann. Auf gar keinen Fall kann man  mit der Methode der
Autosuggestion seine Mitmenschen beeinflussen. Es ist wichtig sich ein Ziel zu setzten, welches in
meinem Wirkungskreis liegt.



Beispielsweise:
\begin{itemize}
	\item Verhaltensveränderungen: z.B. um mit dem Rauchen aufzuhören, weniger zu essen, gelassener
zu reagieren, offener und mutig zu sein, Launen und Stimmung zu beeinflussen usw.

	\item persönliche Ausstrahlung: z.B. um selbstbewusster, offener, Mutiger zu wirken
	\item Einstellungen: z.B. um sein eigenes Leben positiver zu sehen, glücklicher und zufriedener
zu sein, mehr Hoffnung zu haben

	\item Gesundheit: z.B. um sein Immunsystem zu stärken, schmerzen zu lindern, Krankheiten zu
mindern usw.

	\item Kreativität: z.B. u seine Kreativität zu steigern, neue Ideen zu finden u.ä.
	\item Entspannungstechnik
\end{itemize}

\subsubsection{Ziele der Autosuggestion}

Sie Bestimmen ihre Ziele selbst.

Tipps zur  Zielformulierung:
\begin{itemize}
	\item Formulieren sie ihre Ziele immer Positiv. Anstelle von „Ich will nicht mehr wütend werde“
		sagen sie besser „Ich bleibe ruhig und gelassen“.
	\item Die gesetzten Ziele sollten nicht zu weit entfernt oder unerreichbar sein.
	\item Je mehr Details, desto mehr Kraft hat ihre Zielsetzung.
\end{itemize}


\subsubsection{Wieso will ich genau dieses Ziel erreichen?}

Viele Menschen glauben nicht wirklich an die Kraft von Suggestionen. Zu weit hergeholt oder zu
Unwahrscheinlich scheint es, durch mentale Technik Veränderungen zu erreichen. Die Kraft des
Unterbewusstseins ist aber erstaunlich.

Menschen können z.B. kraft ihres willens Krebserkrankungen überwinden oder trotz schwerer Behinderung
Leistungssport betreiben.

Bei der Autosuggestion kommt es auf die richtige Formulierung an, also wie man sich selbst etwas
„einredet“ auch Affirmation genannt. Dies sind so etwas wie Schlüsselsätze oder Eisbrecher. Wenn man
die Aussagen sich dann immer und immer wieder selbst sagen, bewegt man sich fast von alleine in diese
Richtung.


\subsection{Manipulation}
Der Begriff Manipulation (lat. für Handgriff, Kunstgriff) bedeutet im eigentlichen Sinne „Handhabung“
und wird in der Technik auch so verwendet. Im Grunde genommen ist Manipulation also neutral.
Allgemein ist Manipulation ein Begriff aus der Psychologie, Soziologie und Politik und bedeutet die
gezielte und verdeckte Einflussnahme, also sämtliche Prozesse, welche auf eine Steuerung des Erlebens
und Verhaltens von Einzelnen und Gruppen zielen und diesen verborgen bleiben sollen (Camouflage,
Propaganda). Abzugrenzen sind der Vorgang der Manipulation und seine Alltagserscheinungen von der
psychologischen Methode der Experimentellen Manipulation. In seiner ursprünglichen Bedeutung
„Handgriff“ steht Manipulation in der manuellen Medizin für eine Reihe von mit der Hand
durchgeführten Techniken, die dem Lösen einer Blockierung dienen.\footcite{Wikipedia:Manipulation}


\begin{eqlist}
	\item[Die Manipulation:] Kunstgriff
	\item[Die Manipulationen:] Machenschaften, manipulierbar
	\item[manipulieren:] geschickt handhaben, beeinflussen
\end{eqlist}
Quelle: \cite{Duden:Fremdwoerterbuch}
%% Quellangabe laut Josh: Duden (2007). Schülerduden. 6. Auflage. Mannheim: Dudenverlag.

\enquote{Das Fehlen sichtbarer Gewalt erlaubt der Manipulation, sich als jene Freiheit auszugeben,
die sie entzieht.}
\signed{Friedrich Hacker}

Friedrich Hacker geboren am 19. Januar 1914 in Wien, gestorben am 23. Juni 1989 in Mainz, war ein
US-amerikanisch-österreichischer Psychiater, Psychoanalytiker und Aggressionsforscher. Hacker
beschreibt die Manipulation als eine Art freiheits- Beraubun.

\bigskip
Wenn man von der Manipulation eines Menschen spricht, meint man fast immer wenn das Angebot einer
Ware oder Dienstleistung nicht zu seinem Vorteil, sondern zu seinem Nachteil -führt.

Menschen die Minderheitsgefühle, mangelndes Selbstvertrauen oder Angst hat, lässt sich einfacher zu
täuschen  also leichter Manipulieren. Manipulation von Menschen dient dazu, andere hinsichtlich ihres
Verhaltens zu beeinflussen. Der Begriff der Manipulation ist im Grund neutral aber in dem Fall der
Manipulation des Menschen negativ besetzt. Man muss bedenken, dass das handeln der Manipulierten
Person nicht aus eigener Einsicht oder Überzeugung handelt. Sondern fremdbestimmt. Die angestrebte
Lenkung durch gezielte Beeinflussung von Außen erzeugt beim Erkennen zumeist negative Emotionen, da
der Manipulierte zur bloßen Marionette des Manipulierenden gemacht werden und nur noch dessen
Vorstellungen gemäß reagieren soll.

Die organisierte Vorbeugung, von Menschen oder Organisationen, gegenüber Manipulation, kann man
Mind-Security nennen. Auch Aufklärung und Emanzipation sind gegen verschiedene Arten von Manipulation
gerichtet. Spricht man von einer gewollten Veränderung auch seitens der Zielperson, wird eher von
Lernen oder Entwicklung gesprochen.

In der Informationssicherheit wird die Manipulation von Menschen zum Zweck der unerlaubten Gewinnung
von Informationen auch unter dem Begriff Social Engineering diskutiert.

In der neurolinguistischen Programmierung wird gesagt, dass Menschen einander manipulierten, sobald
sie miteinander kommunizieren. Es gibt verschiedene Ausprägungen in Form und Stärke, bei jedem
Individuum. Die einfachste Manipulation bestehe bereits darin, den anderen zum Zuhören zu bewegen.
Dies gilt somit nicht nur auf zielgerichtete Gespräche, sondern auch für einfache Unterhaltungen.
Jeder manipuliere somit jederzeit jeden Anderen, mit dem er zu tun hat. Auch im einfachen Gespräch
hat der Sender eine Absicht, die er erreichen oder sogar durchsetzen will. Selbst der im Augenblick
Manipulierte sorge eigentlich erst mit seiner Haltung dafür, dass der Manipulierende sich so verhält.
Er manipuliere in diesem Sinne den eigentlich (objektiv/subjektiv) als aktiv gesehenen Manipulator.
Wer Manipulation zulasse, gestalte sie mit.

Manipulation ist also eine alltägliche Vorgehensweise und stellt lediglich Beeinflussung dar und sei
demnach nicht negativ zu bewerten. Erst der Zweck und die Eindringlichkeit der Manipulation, zum
Beispiel in Form einer Botschaft der Konsumwerbung, könne negative bzw. positive Wertung ermöglichen.

Der Soziologe Herbert Marcuse war einer der schärfsten Kritiker von Werbemanipulation, die seiner
Ansicht nach den Menschen völlig eindimensional auf Konsumverhalten zu steuert.

Manipulation ist etwas völlig Normales, es gab sie schon immer und wird sie immer geben. Immer dann,
wenn Personen versuchen ihre Interessen durchzusetzen, dafür wird sie selbstverständlich alle zur
Verfügung stehenden Mittel anwenden. Schon ein Kind versucht durch Trotz und Heulen seine Eltern so
zu beeinflussen, dass diese das gewünschte Verhalten zeigen. Gegen Manipulation ist grundsätzlich
auch nicht einzuwenden.

Eine Manipulation wirkt am besten, wenn der Manipulierte:
\begin{enumerate}
	\item Die Manipulation und deren Ziel nicht bemerkt
	\item Den Manipulator nicht als Manipulator wahrnimmt
	\item Und die Ziele des Manipulators für seine eigene hält
\end{enumerate}

Manipulation kann in verschiedenen Stufen vorkommen, die sich darin unterscheiden, wie direkt bzw.
indirekt die Manipulation auftritt.

Die direkteste Form der Manipulation ist die persönliche Beeinflussung einer Person durch
Scheinargumente, Fehlinformationen, Drohungen und Lob
z.B. \enquote{Ein Zeitschriftenverkäufer an der Tür versucht den Kunden zu manipulieren, indem er ihm
erklärt, dass er eine schwere Kindheit hatte und durch den Verkauf von Zeitschriften-Abo`s wieder
auf dem rechten Weg zurückfinden kann.}

Etwas indirekter ist die Werbung im Fernsehen, in Zeitschriften, auf Plakaten und im Kino. Der
Manipulator und sein Ziel sind klar erkennbar. Durch den Einsatz der Medien kann er seine Botschaft
aber häufiger an sein Opfer richten.
z.B. \enquote{In einer Werbeunterbrechung im Fernsehen kann es schon vorkommen, dass die Werbung für
ein Produkt gleich mehrmals gezeigt wird, damit die Botschaft sich fest im Unterbewusstsein
verankert.}
Mehrere Vorteile dieser Methode liegen darin, dass der Manipulator beliebig viel Zeit hat, den
Werbespot zu gestalten.

z.B. Ein Fertigmenü wird einfacher zubereitet als ein Halbfertigprodukt, bei dem die Hausfrau noch
die Zutaten mischen oder würzen muss. Aber kann eine Hausfrau ein Fertigmenü genauso stolz
präsentieren, wie etwas, das sie selbst gekocht hat? Die Hausfrau wird eine unterschiedliche
Erfahrung machen, wenn sie ein Fertigmenü oder ein selbstgekochtes Mittagessen aus
Halbfertigprodukten präsentiert. Beim nächsten Kauf entscheidet sie sich dann für die
Halbfertigprodukte, weil sie ihrer Meinung nach besser schmecken und schließlich kann sie ja von
keinem Familienmitglied verlangen, dass er sie wegen eines Fertigmenüs lobt.

Wenn dieses Prinzip einmal von der Werbeagentur erkannt ist, so wird dieses Vorurteil durch die
Werbung unterstützt. Man zeigt, was die Hausfrau durch wenige Zutaten alles aus den Produkten machen
kann und wie sie dafür gelobt wird. Die Werbeagentur stellt fest, dass Menschen sich nach
immateriellen Güter wie Lob, Anerkennung, Liebe, Freiheit und Frieden sehnen, Güter die man nicht
kaufen kann, und nutzt das aus, indem sie zwischen den immateriellen Güter und konkreten Produkten
einen Zusammenhang herstellt.

„Willst du gelobt und anerkannt werden? Dann koche dieses Produkt!“


\section{Werbung}
\subsection{Vorform der Werbung}

\begin{description}
	\item[In der Antiken:] Ausrufe (seit Antikem Ägyptischen Reich)
Tafel/Schilder (Werbung für Veranstaltungen)

	\item[Im Mittelalter:] Entwicklung der Werbetechnik mit der Erfindung des Buchdrucks
		(ca. 1440) \\
		Handzettel (ab 1466)

	\item[Ab 1630:] Gedruckte Anzeigeblätter
	\item[In den drauf folgenden Jahren:]
		Weitere Verbreitung und kunstvollere Gestaltung, aufgrund neu erfundener Drucktechniken.
\end{description}

\subsection{Anfänge der modernen Werbung}
\begin{description}
	\item[1850:] Eröffnung der ersten Werbeagentur in der USA
	\item[Ab 1860:] Erste Markenartikel kommen auf den Markt
	\item[Ab 1870:] Werbung wird zunehmend übertreibender.
Starke Ausweitung des Warensortiments und intensiver Zuschnitt auf bestimmte Käufergruppen.
Werbung wird Kundenorientierter.
\end{description}

\subsection{Moderne Werbung}
\begin{description}
	\item[Ab 1890:] Bedürfnisse bei Kunden hervorrufen mit Hilfe von Werbung
Erstmals geplante Werbefeldzüge (z.B. Dr. Oetker 1899)
Erste Fachzeitschrift für die Werbewirtschaft
Zeitung als Hauptwerbeträger

	\item[Ab 1900:] Immer stärkere Psychologisierung
	\item[1906:] Erste Werbefilme
	\item[1912:] Beginn der experimentellen wissenschaftlichen
Werbepsychologie
	\item[Ab 1920	:] Verwissenschaftlichung und immer umfassendere Psychologie-
sierung (Marktanalyse)
Unternehmen schaffen zunehmend eigene Werbeabteilungen
(Coca-Cola)
Zielsetzung wird die schaffung eines Marktimages
	\item[Ab 1950:] Erstmals tauchen Fernsehwerbespots auf und werden zunehmend wichtiger
	\item[Ab 1970:] Werbung verändert sich aufgrund eines neuen Verhältnisses zur Sexualität
	Größtenteils handelsorientierte Werbung (Staubsauger, etc.)

	\item[Ab 1980:] Beginn des strategischen Marketing
	Start des Werbefinanzierten Privatfernsehn

	\item[Ab 1990:] Werbung wird zunehmend aggressiver (z.B. Spam-Mails)
	Werbung in den neuen Medien wird immer wichtiger
\end{description}

\section{Medienmanipulation}
Medienmanipulation ist Mittlerweilen ein fester Bestandteil unserer Gesellschaft
geworden. Medienmanipulation ist die gezielte Beeinflussung der Mediennutzer. Sie zeigt
eine einseitige oder verzerrte Darstellung von Deteils und Fakten und dient dazu
die Meinung, Aufmerksamkeit und Wertvorstellung von Menschen zu verändern oder in
Gezielte Richtungen zu lenken. Durch Journalisten, Nachrichtensprecher, Werbesender,
Radio und Zeitung. Auch wenn man nicht einmal die Absicht hat Informationen aufzunehmen, werden sie
einem untergeschoben:

\begin{enumerate}
	\item Im Radio werden die neusten Songs gespielt die einem den richtigen Musikgeschmack zeigen
soll.
	\item Im Fernsehen wirkt die Werbung auf uns und setzt ebenfalls mit jedem Spott ein Meilenstein
in die Meinung der Menschen. In dem sie z.B. sagen wie welche Herren Mode in diesem Frühling getragen
wird.
	\item Die Zeitung überflutet uns mit Nachrichten und Informationen
	\item Prospekte geben einen Einseitigen einblick in der Produkt Welt der jeweiligen Firma und
formen unser Vorstellungsvermögen.
	\item Boulevard Print Medien teilen unser weitere belanglosen dinge mit, die weniger elementar
für die Welt sind.
\end{enumerate}

Medienmanipulation kann verschieden gedeutet werden einerseits:
Wird es verwendet um eine tatsächliche oder vermeintliche Manipulation der öffentlichen Meinung durch
die Medien zu beeinflussen.
Anderseits findet der Begriff Verwendung, um eine Manipulation der Medien mit dem Ziel einer
bestimmten Veröffentlichung zu beschrieben  Pressearbeit).

\subsection{Alle Manipulieren}
Manipulation erlangt in der Masse erst an Bedeutung, wenn es professionell durchgeführt wird. Das
minimale Manipulieren beschränkt sich mehr auf kleine Gruppen, wie in der Firma oder Daheim. Da dies
nicht wirklich ausgeübt wird, fällt dies nicht weiter ins Gewicht, da der Personenkreis überschaubar
ist.

Manipulation in den Medien hingegen ist professionell und organisiert und damit auch erfolgreich, da
hinter diesen Mitteln Konzerne, Banken, Kirche, Militär, Parteien, Adel, einzelne Politiker
Industrielle oder Reporter sitzen.

All diese Personen haben das Ziel der Gewinnmaximierung und ihre Meinung und Ziel auf die anderen zu
übertragen. Dies ist aber nicht ein Prozess, sondern vielmehr ein Produkt von vielen Jahren
(Marktimage).


\subsection{Schwerpunkte der Medienmanipulation}

\subsubsection{Marktkontrolle}
Es geht darum den User zum Kauf verschiedenster Waren zu bringen um die Nachfrage aufrecht zu
erhalten oder meist sogar zu verstärken (oft durch bessere Leistung bei Computeren etc.
Es werden zum Beispiel immer neue Trends entwickelt, die dem User immer wieder in den  Medien
begegnen. Vor etwa zwei Jahren waren es etwa Handys mit guter Kamera, heute sind es 3D fähige
Fernseher, Kaffeevollautomaten für \EUR{1000} oder Autos die mit Elektrizität fahren.
Der Käufer wird auch zum schnellen Kauf animiert. Immer möglichst schnell. So gab es am Anfang von
1\&1 den DSL-WLAN-Router, bei Abschluss eines 24-monatigen Vertrags kostenlos dazu, angeblich nur bis
31.09.2007. Das Angebot gibt es heute noch.

\subsubsection{Politische Manipulation}
Dies ist ein schwerer Fall der Medienmanipulation, da sie am wahrscheinlichsten mit journalistischen
Grundsätzen wie z.B. dem Pressekodex kollidiert. Eigentlich ist auch die freie Meinungsbildung eines
Menschen gefährdet. Die sachliche Berichtserstattung bleibt hier bei auf der Strecke. Die Autoren der
Medien bestimmen einen politischen Standpunkt. Zeitung, Nachrichtensendungen und auch das „freie“
Internet.

\subsubsection{Propaganda}
Ist der wohl schlimmste Fall der Medienmanipulation. Der Mediennutzer wird im Bezug auf Rasse, Ethik,
Politik, Gesellschaftsklassen beeinflusst. Der größte Propaganda Aufruf war wahrscheinlich im 3.
Reich, gegen die Juden.

\subsubsection{Unterhaltung}
Die von der Boulevard Presse aufgeputschten Skandale von Mord, Tod oder Stars und Sternchen, werden
in den Medien zur Unterhaltung angeboten. Man versucht kurzzeitig Aufmerksamkeit zu erlangen und
macht aus einer Mücke einen Elefanten.

\subsubsection{Methoden}
Jede Werbung hat das Ziel den Konsumenten zum Kaufen zu verführen, doch nicht jede Werbung ist gleich
gute Werbung. Mit folgenden Methoden lassen sich Medien gezielt Manipulieren.

\subsubsection{Selektion}
Auswahl der Inhalte ist ein grundsätzliches Element der Berichterstattung, hierbei zieht die
Redaktion die für sich wichtigen oder anregenden Informationen  heraus. Über 99% aller Nachrichten
die ein Journalist aufnimmt erreichen nie die Leser oder Hörer.
Wichtig für die Wirkung des Mediums (Bild bzw. Zeit)

\subsubsection{Gewichtung}
Das relevante wird am stärksten hervorgehoben, gezielt beeinflusst. Eine Herausforderung für jeden
Journalisten, da jedem etwas anderes Wichtig ist. Er muss aufzeigen, was wichtig ist und durch
Gewichtung seine Argumentation verdeutlichen.
Gelingt ihm dies nicht, spricht man von einer „Verzehrten“ oder „tendenziösen“ Berichtserstattung.
Dies passiert z.B., wenn gegen die Journalistische Ethik versoßen wird.

\subsubsection{Sprachgestaltung}
Die Anwendung von positiven bzw. negativen Worten an Stellen eines neutralen Synonyms ist geeignet,
den Leser zu beeinflussen. Sind einzelne Themen unstrittig wie zum Beispiel in der Zeit nach dem
2.Weltkrieg, wo es auf der einen Seite bei den deutschen „Flucht und Vertreibung“ heißt wird es in
Polen „Umsiedlung“ genannt. Gleiches Geschehnis verschiedene Wortwahl. Ein größeres Problem stellt
die Übersetzung dar, hier hat der Übersetzer die Gelegenheit zu Manipulieren, wenn er nicht
Wortwörtlich und Originalgetreu Übersetzt.

\subsubsection{Unwahrheitsgemäße Berichterstattung}
Kurzzeitig hohe Aufmerksamkeit durch den Nutzer der sich wundert oder es nicht glauben will.

\subsection{AIDA-Regel}
Werbung versucht mit Hilfe der Psychologie uns durch besondere Methoden Empfindungen oder Gefühle zu
vermitteln. Die uns zum Konsum anregen sollen. In diesem Zusammenhang wird von Manipulation von
Werbung gesprochen.
Das Grundmuster für solche Werberegeln ist die AIDA-Regel.

\begin{eqlist}
	\item[Attention:] Aufmerksamkeit erringen
					z.B. Gefühle ansprechen, zum Lachen bringen
	\item[Interest:] Interesse erregen
					z.B. Neugierde wecken, Informationen liefern
	\item[Desire:] Wünsche, Bedürfnisse hervorrufen
					z.B. Glücksversprechen, Aufforderung
	%	Übergang von der Botschaft zur Handlung
	\item[Action:] Handlung
					z.B. Kauf, Verhaltensweiße ändern
\end{eqlist}

Der AIDA-Regel folgend ist die Sprache gestaltet.

\subsection{Sprache der Werbung}
\begin{description}
	\item[Ist einprägsam, witzig durch:] Reime, Slogans, Wortspiele, Wortneubildungen
	\item[Macht persönlich betroffen durch:] Ausrufe, Befehle, Aufforderung, Behauptung, persönliche
		Anrede, Fragestellungen
	\item[Ist schnell auffassbar durch:] Einfachheit, Verkürzung, Auslassung, Verständlich,
		Eingängigkeit
\end{description}

%Bleibt im Gedächtnis, steuert unbewusst

\subsection{Tricks der Medien}
\subsubsection{Sprache}
\begin{itemize}
	\item Englisch klingt modern, besonders bei technischen Dingen. Selbst ein Begriff wie
„Kassettenrekorder“ klingt nicht mehr modern genug und muss in „Tape-Deck“ umgetauft werden.

	\item Verharmlosende Begriffe werden gewählt, um auf weniger Wiederstand zu stoßen.
z.B. „Terrorist heißt dann Freiheitskämpfer, Mülldeponie heißt dann Entsorgungspark und Fremdarbeiter
heißt Gastarbeiter.“

	\item Gebrauch übermäßig harter Begriffe soll Empörung erzeugen.
z.B. \enquote{jemand, der mit Tempo 60 im Ort geblitzt wird, gilt als \enquote{Raser}, und Soldaten
gelten als \enquote{Potentiellen Mörder}.}

	\item Falsche Anwendung eines Begriffes kann zur Wandlung der Bedeutung führen.
z.B. „das Wort „Idiot“ stamm vom griechischen „idiotes“ und bedeutet „gewöhnlicher, einfacher Mann“,
von Geisteskrankheit ist da keine Spur.“

	\item Gefühlsbetonte Worte werden benutzt: „Lebensqualität“
	\item Scheintatsachen werden veröffentlicht:
z.B. „70\Prozent{} der Kunden sind mit dem Produkt zufrieden. (Wie viele wurden befragt; Wie wurde
befragt?“
\end{itemize}

\subsubsection{Hintergrundmusik, Beleuchtung, Boden, Geruch}
\begin{itemize}
	\item Stille im Supermarkt hat für den Manipulator den Nachteil, dass die Opfer ständig durch
irgendwelche Geräusche abgelenkt werden, so dass sie sich nicht mehr richtig auf die Ware
konzentrieren können. Hintergrundmusik darf den Kunden hier nicht überlasten, also wählt man eine
ruhige Musik ohne Gesang und ohne hohen Töne, damit der Kunde sich entspannt.

	\item In einer Kneipe darf der Kunde nicht ruhig gestellt werden, deswegen wählt man eine laute
Musik, mit Texten, die die Aufmerksamkeit auf sich ziehen, damit die Kunden sich nicht langweilen.
Hier werden Lieder mit einfachen Texten gewählt, damit der Kunde mitsingen kann.

	\item Im Kaufhaus ist der Boden sehr wichtig. Er muss dumpf klingen und Schall schlucken, weil
Tritt- und Klappergeräusche den Kunden nur ablenken würden.

	\item Auch der Geruch ist ein wichtiges Detail. Wenn es an einem Würstchenstand richtig lecker
riecht, dann bekommt man Appetit auf dem Produkt.

	\item Die Beleuchtung in Kaufhäuser darf nicht zu grell sein, sondern gemütlich, damit der Kunde
sich entspannt.
\end{itemize}


\subsubsection{Die große Verschwendung}
\begin{itemize}
	\item Menschen die schon alles haben, brauchen nichts mehr. Das würde ein Zusammenbruch für die
Wirtschaft bedeuten. Deswegen werden Menschen durch einen enormen Aufwand überzeugt, dass sie immer
noch mehr brauchen, egal wie viel sie schon haben. Sie sollen so viel wie möglich weg werfen, damit
sie neues kaufen.

	\item Menschen brauchen alles doppelt und dreifach. Jedes volljährige Familienmitglied braucht
ein eigenes Auto, eins für die ganze Familie reicht nicht.
	\item Normalerweise braucht man eine Brille zur Korrektur der Sehschwächen. Doch wenn man Brillen
als modische Bedeutung sieht, dann braucht man eine für die Arbeit, eine für den Abend, eine für
Feste~\dots

	\item Gegenstände bekommen Schwachstellen, damit sie schneller kaputt gehen und man neues kauft.

	\item Normen werden geändert. Die alte Kamera ist nicht mehr gut genug. Jetzt braucht man eine
Digitalkamera mit 15 Megapixel.

	\item Viele Gegenstände können nur einmal benutzt werden, dadurch muss man es immer neu kaufen.
(Pappteller, Taschentücher, Windeln~\dots)
\end{itemize}

\subsubsection{Schwächen und Denkfallen}
Die Kenntnis unserer Schwächen ist ein großer Triumph in der Hand des Manipulators. Eine Manipulation
ist dann am effektivsten, wenn man mit wenigen Mitteln den größten Erfolg hat.
z.B. „bei Autowerbung. Würde man ein Auto durch Nennung seiner Eigenschaften (Benzinverbrauch)
anpreisen, so würde der Kunde diese Eigenschaften mit anderen Modelle vergleichen. Hier ist er stark.
Wenn man also ein Auto verkaufen will, dann kann der Verkäufer z.B. verkünden, dass dieses Auto ein
spezielles Sicherheitsfahrwerk hat, dass mit irgendeiner Abkürzung bezeichnet wird. Woher soll nun
der Kunde wissen, was die speziellen Vorteile dieses Sicherheitsfahrwerkes sind? Womit soll er es
vergleichen? Da die Abkürzung natürlich als Warenzeichen geschützt ist, werden andere Hersteller ein
Fahrwerk gleichen Namens nicht anbieten können.

Damit dieses System funktioniert gelten einige Regel:
\begin{itemize}
	\item Vergleichende Werbung ist in Deutschland verboten
	\item Es ist verboten gegen ein Produkt Werbung zu machen
	\item Produkte werden unvergleichbar gemacht
\end{itemize}


\newpage
\section{Beispiele}
\subsection{Bildmanipulation}
Ein anderes Beispiel war während es Irak-Kriegs im Jahre 2003 aufgetaucht wo der linke und rechte
Ausschnitt verschiedene Aussagen machen.

\begin{figurewrapper}
	\includegraphics[width=10cm]{files/images/Joshuah/Irak}
	\caption{Beispiel Bildmanipulation}
\end{figurewrapper}
%Quelle: Ursula Dahmen, Wanderausstellung "X für U -Bilder, die lügen". Spiegel.
%% http://einestages.spiegel.de/static/entry/finden_sie_die_fehler/17385/irakischer_soldat.html


\subsection{Die Landung von Marsmenschen auf der Erde}
Von Orson Welles stammt das legendäre Hörspiel „Invasion vom Mars“. Berühmt wurde es vor allem, weil
viele Hörer während der Ausstrahlung ängstlich, fast hysterisch reagierten. Das kam so:

Am Abend von Halloween, dem 30. Oktober 1938, wird plötzlich das gewöhnliche amerikanische
Hörfunkprogramm von der Durchsage des Brigadegenerals Montgomery SMITH unterbrochen:

„Über die Gebiete um Middlesex wird der Ausnahmezustand verhängt. Niemand darf diese Gebiete
betreten. Vier Kompanien rücken nach Grovers Mill vor, um die Bevölkerung zu evakuieren.“

Ein größerer Meteorit war niedergegangen. Rundfunkreporter begaben sich mit ihrem Übertragungswagen
sofort zur Einschlagstelle. Sie berichten live, der Meteorit sei aus Metall. Ein Reporter schildert
seine Eindrücke:

„Meine Damen und Herren, das ist einmalig! Das ist die fürchterlichste Sache, die ich jemals erlebt
habe. Es ist das außergewöhnlichste Ereignis. Ich finde keine Worte.“

Wenig später wird der Reporter tot aufgefunden. Das Metallgehäuse hat sich geöffnet. Außerirdische
Wesen beginnen die Invasion vom Mars. Das Radio berichtet weiter vom Kampf der Militärs gegen die
vorrückenden Wesen. Das Programm ist unterbrochen. Politiker geben Stellungnahmen, Ärzte erklären
Vorsichtsmaßnahmen, Feuerwehrleute regeln den Verkehr~\dots

Der Realismus dieses Hörspiels ging so weit, dass Orson Welles einen Angriff der Marsmenschen auf das
Funkhaus vorführte und plötzlich der Sender tot war und kein Laut mehr aus dem Radio kam.

Mittlerweile kam es bei den sechs Millionen Radiohörern zu panikartigen Re-aktionen. Viele flüchteten
mit Autos oder suchten in verbarrikadierten Kellern Schutz. Andere eilten in Kirchen, beteten und
flehten um Schutz vor den Marsmonstern. Andere stürmten Polizeiwachen oder die Redaktionen von
Tageszeitungen, um Neuigkeiten zu erfahren. Nachbarn warnten einander. Hörer, die später befragt
wurden, sagten etwa:

Eine Mutter: „Ich wollte packen, mein Kind auf den Arm nehmen, im Auto so weit nach Norden fahren,
wie wir konnten. Ich war ganz steif vor Schreck“.

Das ganze Spektakel war ein im Stil eines Live-Berichts inszeniertes Hörspiel. Orson Welles bediente
sich der Gestaltungsmittel des Dokumentarradios, um seiner erfundenen Handlung mehr Glaubwürdigkeit
zu verleihen. Viele Hörer, zumal wenn sie sich erst später ins Programm eingeschaltet hatten, konnten
Realität und Fiktion nicht mehr unterscheiden und reagierten mit Angst.

Den Abschluss der Sendung mit den Worten: „Leute, ich hoffe wir haben euch nicht durcheinander
gebracht. Dies ist nur ein Hörspiel.“ konnten viele nicht mehr glauben.\footnote{Übernommen aus
\cite[4]{Stuttgart:Manipulation}}

\subsection{Leben Sie jetzt -- zahlen Sie später}
Hier wird ganz offen Lebensfreude und Partnerglück mit der Illusion verknüpft, dass es gar kein
Problem sei jetzt richtig zu leben -- was heißen soll Geld auszugeben -- weil man ja ganz einfach
später zahlen könne. Die Botschaft heißt, genieße den Augenblick mit Schulden. Zurückzahlen kannst Du
ja ganz \enquote{einfach} später. Damit wird Leben auf Pump und Schuldenmachen mit
Lebensfreude Glück und Liebe identifiziert. Es wird suggeriert, dass finanzielle KREDITFreiheit
Lebensfreude, Glück, Liebe und Partnerschaft fördert. Wie viele Partnerschaften an ihren finanziellen
Problemen zerbrechen, die so erzeugt werden, wird natürlich nicht mitgeteilt, ebenso wenig die
gigantischen Probleme, die mit der öffentlichen und privaten Verschuldung einhergehen.

\begin{figurewrapper}
	\includegraphics[width=7cm]{files/images/Joshuah/Leben_Sie_jetzt}
	\caption{Beispiel Leben Sie jetzt -- zahlen Sie später}
\end{figurewrapper}

%\nocite{
%}

\clearpage
\printbibliography[heading=source,keyword=Joshuah]
