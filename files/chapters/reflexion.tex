\chapter{Persönliche Reflexion}
\label{sec:reflexion}

Als wir in der Schule auf Themensuche und an die Gruppeneinteilung gingen, wussten wir schnell, was
uns interessiert. Unser Thema hieß anfänglich „Manipulation am Menschen“. Nach einigen
Vorüberlegungen fiel uns auf das der Begriff Manipulation in unserem Sprachgebrauch sehr negativ
belegt ist und wir in unserer Arbeit sehr wohl auch auf positive Aspekte eingehen wollten.

Nach langen Überlegungen kamen wir auf den Titel: „Veränderung und Beeinflussung am Menschen“. Im
Folgenden überlegten wir, was wichtige Themen bei Veränderung und Beeinflussung am Menschen für uns
sind. Diese Themen teilten wir in unserer Gruppe ein (\siehe{sec:preface}).

Die nächste Phase war wohl die schwerste für uns alle: Mit Recherchen anfangen und nebenher die
Gruppe im Blick behalten. Wir wussten, was zutun war, theoretisch ganz einfach, praktisch hatten
wir aber alle zu kämpfen.

Doch die anfänglichen Probleme wurden immer weniger, wir blieben in Kontakt, arbeiteten
erstaunlicherweise in einem ähnlichen Tempo und halfen und unterstützen uns wenn nötig gerne. Auch
ohne genauen Zeitplan kamen gut ans Ziel. Uns kamen dabei sicherlich die
Erfahrungen, die wir mit den Projektarbeiten gesammelt haben, zugute.

Sodass wir nun ein schönes und vielseitiges Dokument unser Eigen nennen können.

Die Arbeit hat uns oft Kopfzerbrechen bereitet, sodass wir jetzt stolz sind, es geschafft zu haben.

\begin{figurewrapper}
	\includegraphics[width=0.7\hsize]{files/images/Gruppenfotos/IMG_8096\imageresize}
	\captionof{figure}{Gruppenfoto von uns}
	\label{fig:group_picture}
\end{figurewrapper}
