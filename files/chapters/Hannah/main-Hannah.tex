\chapter{Hannah Haid}
\label{sec:Hannah_Haid}

\section{Einleitung}
Schon Kant sagte \enquote{Schönheit muss nicht nützlich sein}. Trotzdem beschäftigt sich jeder Mensch damit
und das ist schon immer so. Wir wollen schön sein und sind bereit viel dafür zu tun und hohe
gesundheitliche Schäden auf uns zu nehmen. In jeder alten Kultur sind schönheitssteigernde Dinge zu
finden und das trotz sehr verschiedener Ideale. Tattoos, Schmuck und Ohrlöcher gehören zu den
häufigsten Veränderungen. Ziernarben, Intimpiercings und vieles mehr sind jedoch auch weit
verbreitet und
sie sollten meist Clanzugehörigkeiten oder den familiären Status
sichtbar machen.

Schönheitsideale sind im ständigen Wandel und von vielen, teilweise lang zurückliegenden
Faktoren abhängig. Eindeutige Strukturen, die allgemein gültig sind, gibt es kaum. Das
womöglich
Wichtigste ist jedoch das in Zeiten, in welchen Menschen im Wohlstand leben schlanke Figuren als
schön
gelten, in Hungerzeiten sind dicke Menschen meist schöner. Dies kann man heutzutage an den
Dritteweltländern gut beobachten, obwohl sich der Schönheitstrend immer mehr an der westlichen Welt
orientiert.

In unserer Welt werden die Ideale immer verrückter, Schönheitsoperationen immer häufiger und
normaler. Umso schlanker umso schöner lautet die Devise. Wer dem Bild nicht entspricht erlebt oft
einen großen Zwang sich anzupassen oder hat mit hohen, zum Beispiel finanziellen, Einbußen zu
rechnen.

Im Folgenden will ich auf die Entwicklung der Schönheit im Laufe der Zeit und in verschiedenen
Kulturen
eingehen, sowie auf unser Schönheitsbild welches viele verschiedene Formen kennt, von Body
Modification zu Operationen und \href{http://de.wikipedia.org/wiki/Bulimie}{Bulimie}.

\clearpage
\section{Schönheit in verschiedenen Kulturen}
Viele Kulturen haben ihre ganz eigenen Schönheitskulturen, mein erstes Beispiel sind die
Tellerlippen in Afrika.

\subsection{Tellerlippen}
Das Dehnen der Lippen wird vor allem von den, in Äthiopien lebenden, Mursi und Surma Frauen
praktiziert.
Am Ende der Pubertät
bekommen die Mädchen einen Schnitt in die Unterlippe, außerdem werden die unteren Schneidezähne
entfernt, in das entstehende Loch wird eine Tonscheibe eingesetzt, die über längere Zeit durch immer
größere ersetzt wird, die Lippen werden bis zu \SI{15}{\centi\metre} gedehnt.

Umso größer der Teller ist umso größer ist das Ansehen der Frau und umso höher ist ihr Brautpreis.
Heutzutage verdienen sich viele Frauen durch ihre Lippenteller Geld von Touristen, welche für Fotos
mit ihnen Geld zahlen, mit welchem Missernten ausgeglichen werden. In ihrem normalen Leben
werden
die Teller jedoch nicht verwendet, zum essen und arbeiten nehmen die Frauen die Teller heraus, erst
wenn sich Touristen nähern werden die Scheiben eingesetzt.

Es wird angenommen, dass die Lippenteller früher jedoch nicht als attraktiv empfunden wurden, sondern
die Frauen vor Sklavenhandel schützen sollten.

\begin{figurewrapper} %% Einbindung steht im Konflikt mit dem Urheberrecht
	\subfigure[Frau mit	Tellerlippe]%
		{\includegraphics[width=0.45\textwidth]{files/images/Hannah/alles-4_1}} \hfill
	\subfigure[Frau ohne Tellerlippenring]%
		{\includegraphics[width=0.45\textwidth]{files/images/Hannah/alles-4_2}}%
	\caption{Tellerlippen}
\end{figurewrapper}

\newpage
Bekannte Völker, welche Lippenteller verwenden sind:
\begin{multicols}{2}
\begin{itemize}
	\item die Botokudo-Frauen und Männer in Brasilien
	\item die Kayapo in Brasilien
	\item die Kichepo in Äthiopien
	\item die Makonde in Tansania und Mosambik
	\item die Mursi-Frauen in Äthiopien
	\item die Sara-Frauen im Tschad
	\item die Surma-Frauen in Äthiopien
	\item die Suyá-Männer in Brasilien
\end{itemize}
\end{multicols}

\subsection{Giraffenhals}
\begin{wrapfigure}{r}{0pt}
	\includegraphics[width=5cm]{files/images/Hannah/alles-6_2}%
	\captionof{figure}{Kind mit den ersten Ringen}%
\end{wrapfigure}

Die Padaung, ein Bergvolk in Myanmars sind bekannt für ihre langen Hälse auch unter dem Namen
Giraffenhals bekannt.

Im Alter von fünf Jahren bekommen die Kinder ihre ersten Reifen um den Hals, welche in einem
mehrstündigen Ritual angepasst werden. Nach und nach werden es immer mehr Spiralen.
Diese Messingspiralen werden extra für diesen Zweck in Myanmar
angefertigt werden. Die ersten Spiralen sind circa \SI{10}{\centi\metre} hoch mit einem Durchmesser von
\SIrange{30}{40}{\centi\metre}. Alle zwei bis drei Jahre, je nach Wachstum der Mädchen werden neue
Spiralen angelegt. An den Schultern werden etwas breitere Spiralen getragen, welche nochmals das Bild
der langen Hälse verstärkt. Insgesamt können die Spiralen eine Länge von bis zu \SI{30}{\centi\metre}
erreichen, dies sind 20 bis 25 Windungen.

Die Reifen können bis zu \SI{10}{\kilo\gram} wiegen und drücken somit die Schultern nach unten,
wodurch das Bild des langen Halses zusätzlich verstärkt wird.
Durch diesen Schmuck müssen die Frauen einen Großteil ihrer
Beweglichkeit einbüßen und Schwierigkeiten beim Schlucken hinnehmen. Außerdem ist komplizierte
Hygiene nötig und
eine andauernde Last mitzutragen. Die weitverbreitete Annahme das die Frauen sich durch das Abnehmen
der Reifen das Genick brechen würden ist jedoch nur ein Mythos, die Halsmuskulatur ist geschwächt und
muss am Anfang geschont werden, baut sich jedoch wieder auf. Es ist bekannt, dass es schon im
Mittelalter Frauen mit Giraffenhälsen gab, die Entstehung ist jedoch von vielen Sagen umwoben.

Es wird erzählt, dass die Messingringe die Frauen vor Tigerbissen schützen sollten, andere meinen
dass der Halsschmuck auf den Ursprungsmythos der Paduas zurückzuführen ist. Nach ihm stammen die
Padua von einem gepanzerten weiblichen Drachen ab und ihr Halsschmuck soll an die \enquote{Drachenmutter}
erinnern.

Erst durch die Missionierung der Engländer 1820 wurde der Schmuck als Schönheitsideal und
Statussymbol umgedeutet,
seither ist es ein Zeichen für Reichtum, Würde und Erhabenheit.

\begin{figurewrapper} %% Einbindung steht im Konflikt mit dem Urheberrecht
	\subfigure[Traditionell gekleidete Frauen]%
		{\includegraphics[height=8cm]{files/images/Hannah/alles-5_1}} \hfill
	\subfigure[Giraffenhals voll gedehnt]%
		{\includegraphics[height=8cm]{files/images/Hannah/alles-6_1}}%
	\caption{Giraffenhals}
\end{figurewrapper}

\subsection{Lotosfuß}

\begin{wrapfigure}{r}{0pt}
	\includegraphics[width=5cm]{files/images/Hannah/alles-7_1}%
	\captionof{figure}{Normaler und verformter Fuß}%
\end{wrapfigure}
Ein weiteres Beispiel sind die Lotosfüße, welcher über
mehrere Jahrhunderte in China als Schönheitsideal angesehen wurde.
Diese gelten als besonders barbarisch da die Entstehung eines solcher Füße circa 15 Jahre lang dauert.
Den Mädchen werden im Alter von 3 bis 5 Jahren vier Zehn gebrochen und nach hinten, unter die
Fußsohle gebunden und dort mit festen Bandagen umwickelt, welche jeden Tag enger geschnürt werden.
Dabei sterben die Zehen oft ab oder fangen unter den Stoffwickeln an zu faulen.

Das Ideal war der \enquote{goldene Lotosfuß}, welcher eine Fußlänge von \SI{10}{\centi\metre} besitzt,
was der europäischen Schuhgröße 17 entspricht und Kinder im Alter von 2 bis 3 Monaten erreichen.
Dieses
Ideal wurde jedoch nur von wenigen Frauen erreicht, normal war eine Fußlänge von
\SIrange{13}{14}{\centi\metre}. Mit dem Lotosfuß sind lebenslange Schmerzen verbunden, diese nahm
die Familie jedoch gerne in Kauf, da es eine Garantie für Ansehen und eine gute Hochzeit war. Die
Frauen konnten nie mehr normal gehen und erst recht keine größeren Strecken zurücklegen. Dies passt
gut in die Frauenrolle der damaligen Zeit durch die Füße mussten die Frauen zuhause bleiben, hatten
kein Chance selbstständig zu werden und mussten sich somit den Männern unterwerfen und sich
ausschließlich dem Haushalt widmen.

Nur Töchtern von sehr armen Bauern wurden verschont um die Feldarbeit jener gewährleisten zu können.

Der Ursprung dieses Brauches führt ins Jahre 975 zurück. Der damalige Kaiser Li Yu schenkte seiner
Geliebten, eine Tänzerin, eine Bühne, die als goldene Lotosblüte geformt war. Um auf ihr tanzen
zu können bandagierte sie sich die Füße, jedoch lang nicht so stark, die schmerzhafte Verkrüppelung
und Verstümmelung kam erst später.

1911 wurde die Abbindung verboten allerdings ohne Erfolg. Erst 1949 nachdem
die Volksrepublik China gegründet wurde gelang es Mao Zedong diesen Brauch endgültig zu verbieten.
1988 schloss die letzte Fabrik, welche Spezialschuhe anfertigte. Es gibt somit immer noch ältere
Frauen deren Füße abgebunden wurden.

\begin{figurewrapper}
	\subfigure[Frau mit verformtem Fuß]%
		{\includegraphics[width=0.47\textwidth]{files/images/Hannah/alles-8_1}} \hfill
	\subfigure[Spezialschuhe für Lotosfüße]%
		{\includegraphics[width=0.43\textwidth]{files/images/Hannah/alles-8_2}}%
	\caption{Lotosfuß}
\end{figurewrapper}


\section{Schönheit in verschiedenen Zeiten}

Im Folgenden will ich einen kurzen Überblick über die verschiedenen Epochen und Schönheitsideale
in Europa geben.

\subsection{Mittelalter}

\begin{wrapfigure}{r}{0pt}
	\includegraphics[width=5cm]{files/images/Hannah/alles-9_1}%
	\captionof{figure}{Typische Kleidung im Mittelalter}%
\end{wrapfigure}
Das Mittelalter war geprägt vom Christentum. Walter von der Vogelweide beschrieb das Frauenideal im
13. Jahrhundert als \enquote{edeliu schoene Frouwe reine} in seinen Minneliedern. Einfach und
schmucklos war die Devise. Schminke und übermäßige Pflege des Körpers galten als heidnisch, so dass
vor allem schönen Frauen mit Misstrauen begegnet wurde, war es doch Eva welche auch durch
ihre Schönheit als gefährliche Verführerin galt und die Vertreibung
aus dem Paradies herbeigeführt hat. Trotz allem verfasste der Gelehrte Gilbertus
Anglicus ein zwölfteiliges Dossier über die Kosmetik und Pflege von Gesicht und Haaren. Sein Werk war
richtungsweisend hatte jedoch im Spätmittelalter nur noch wenig Einfluss.

Natürlich war Schönheit nicht völlig egal so dass eine möglichst helle Haut als schön und
erstrebenswert galt. Wer es sich leisten konnte mied die Sonne, regelmäßige Aderlässe verstärkten die
Blässe. Hygiene war unwichtig, man verdächtigte sie im übertriebenen Masse sogar als Wegbereiter der
Pest. Wasser und Seife wurden durch Parfum und Puder ersetzt.

\subsection{Renaissance/Barock/Rokoko}

\begin{figurewrapper}
	\subfigure[Schönheitsideal in der Renaissance\label{fig:RenaissanceOne}]%
		{\includegraphics[height=10cm]{files/images/Hannah/alles-10_1}} \hfill
	\subfigure[Gemälde von Ruben, sehr ausschlaggebend für das Schönheitsideal in der Renaissance]%
		{\includegraphics[height=10cm]{files/images/Hannah/alles-10_2}}%
	\caption{Schönheitsideal in der Renaissance}
\end{figurewrapper}

Die Adlige Lucrezia Panciatichi (siehe Abb.~\vref{fig:RenaissanceOne}) galt als typische
Schönheit im Zeitalter der italienischen Renaissance. \enquote{Rubensfigur} oder \enquote{barocke Lebensart} sind
wohl die bekanntesten Begriffe des Barocks.


Diese beiden Begriffe beinhalten eine ausschweifende Lebensart in welcher der Genuss sehr wichtig ist
(1600--1720). Was auf Bildern zu sehen ist war jedoch in der Realität anders, dass Korsett kam in
Mode und damit die künstliche Wespentaille, das gelockt oder wellende Haar musste in der
Öffentlichkeit gescheitelt und nach hinten frisiert werden.
Dies blieb im Rokoko ähnlich um \enquote{schön} zu sein mussten Haare aufgesteckt, gepudert und mit Blumen,
Schleifen und Ähnlichem geschmückt werden

Bei Männern kamen Perücken in Mode, die Locken umspielten Wangen und Kinn.
Zurzeit von Ludwig XVI kamen hochtoupierte Frisuren in Mode und damit entstanden die ersten Perücken
für Frauen. Bleiche Haut, rougerote Wangen, kohlengeschwärzte Augenbrauen und hohe Frisuren wurden
schick. Parfum war angesehener als Waschen, sodass die Menschen
von Duftwolken umgeben waren.

\subsection{Klassizismus}
Im Klassizismus ist Natürlichkeit sehr wichtig, wie bei den Römern sind eine über die Hüfte
verschobene Taille, ein rosiges Gesicht und ausdrucksvolle, möglichst wenig geschminkte Augen in
Mode.

Im 20. Jahrhundert ändern sich die Schönheitsideale immer schneller. Korsagen werden der
Vergangenheit angehängt dadurch können überflüssige Kilos jedoch nicht mehr versteckt werden,
sondern müssen abgespeckt werden. Zu Beginn der 20er Jahre ändert sich das Schönheitsideal
grundlegend. Haare werden wenn möglich in weichen Wellen, Kinnlang getragen, ein kleiner Kopf, große
Augen und ein voller Mund werden zum neuen Ideal erklärt.

Unser heutiges Schönheitsideal ist schlank. Magersucht ist eine häufige Krankheit unserer Zeit. Für
unser verdrehtes Schönheitsbild wird die Kultpuppe Barbie oft als mitverantwortliche verschrien. Im
Folgenden
möchte ich kurz auf die Barbie, ihre Entstehungsgeschichte, ihre Auswirkungen und die Kritik an ihr
eingehen.

\bigskip
\begin{figurewrapper}
	\includegraphics[width=10cm]{files/images/Hannah/alles-11_1}%
	\captionof{figure}{Schönheitsideal im Klassizismus}%
\end{figurewrapper}

\newpage
\section{Die Barbiepuppe}

\begin{wrapfigure}{r}{0pt}
	\includegraphics[width=7.5cm]{files/images/Hannah/alles-12_1}%
	\captionof{figure}{Die erste Barbie}%
\end{wrapfigure}
Die Barbiepuppe wird von der Firma Mattel produziert, diese wurde 1945 von Ruth und Elliott Handler
gegründet.
Am \printdate{09.03.1959} wurde die erste Mattel-Puppe auf dem American Toy Fair in New York
präsentiert und nach Ruth und Elliotts Tochter Barbara benannt.

Die Barbie orientiert sich immer an den aktuellen Mode Trends, deshalb hat sich ihre Figur oft
geändert.
Neu an der Barbie ist, dass sie eine erwachsene Frau ist und somit nicht wie frühere Puppen den
Kindern helfen soll zu erziehen, die Aufgabe der Barbie ist das Spiel der Zukunft. Um dieses
spannender zu gestalten, hat Barbie viele Freunde und Verwandte, der Stammbaum ist jedoch sehr
unübersichtlich. Sie haben eigene Körper und Gesichter, in der Packung ist immer ein kleiner Zettel,
der Beziehung zwischen der Barbie und jeweiligen Puppe beschreibt, was sie verbindet und wie sie
heißt.

Von Anfang an hatte Barbie eine vielfältige Garderobe für jede Gelegenheit. Ab den 90er Jahren trägt
Barbie Alltagskleidung und ist somit auch für Unter- und Mittelschicht erreichbar, davor hatte sie
aufwendige Kostüme und war sehr teuer.

Manche Kleidung der Barbie können Kinder durch Pailletten oder Stoffmalfarbe selbst Verändern und neu
kreieren. Seit 1980 gibt es auch afroamerikanische und hispanische Barbies.

\subsection{Kritik}
\begin{itemize}
	\item Die Proportionen der ersten Barbie waren 99-46-84 nach heftiger Kritik wurde die
		Oberweite verringert und die Taille vergrößert, die neuen Maße entsprechen immer noch
		nicht der Realität, ließen jedoch viele Kritiker verstummen.
	\item Feministen kritisieren an der Barbie das sie das traditionelle Frauenbild zeigt und zu
		kritiklosem Konsum anregt.
	\item Im September 2003 verbot Saudi-Arabien die Barbie da sie den Islam untergrabe, seit
		November 2003 gibt es das gegen Model Fulla zu kaufen.
	\item Die Wirkung der Barbie wird als demütigend und für das Schönheitsbild der Kinder prägend
		beschreiben, eine Frau wird niemals Barbies Masse haben können, Kindern wird jedoch durch das
		Spielen mit Babie genau dieser Körper als schön beigebracht
\end{itemize}

\subsection{Wissenswertes über die Barbie}
\begin{itemize}
	\item Jeder Deutsche kennt die Marke Barbie
	\item Pro Sekunde werden statistisch 3 Barbiepuppen verkauft
	\item Jedes Mädchen besitzt durchschnittlich 7 Barbie
	\item Ein Mensch mir Barbie Massen wäre nicht lebensfähig da im Unterleib nicht genug Platz
		für alle Organe wäre
\end{itemize}

Seit 2001 gibt es Filme mit der Barbie nicht mehr nur als Kleiderpuppe.

\begin{figurewrapper}
	\includegraphics[width=7cm]{files/images/Hannah/alles-14_1}%
	\captionof{figure}{Barbie und Ken}%
\end{figurewrapper}

\section{Attraktivitätsforschung}
Die Menschheit beschäftigte sich schon immer mit Attraktivität in den letzten 20 Jahren kamen
weiterführende Erkenntnisse über die Attraktivitätsforschung hinzu.

Albrecht Dürer schrieb \enquote{Schönheit -- was das sey, weiß ich nit, obwohl sie vielen dingen anhanget.}
Leonardo da Vinci war ebenfalls auf der Suche nach dem perfekten Körper und seiner Erfassung in
Formeln.

Obwohl die Attraktivitätsforschung noch vor vielen ungeklärten Fragen steht, ist eine wichtige
Erkenntnis: \enquote{Eine Frau ist schön, wenn sie typisch weiblich aussieht, und ein Mann ist schön, wenn er
typisch männlich aussieht!}

Die Formel durch die Schönheit berechnet wird ergibt Attraktivitätswerte die von 1 für sehr
unattraktiv bis 7 für sehr attraktiv reichen.

Eine neu entwickelte Methode, welche Morphing genannt wird ist dabei schlaggebend für die
Grundaussagen der
Attraktivitätsforschung. Bei ihr werden mehrere Fotos von Frauen welche ähnlich attraktiv sind
übereinandergelegt, dadurch entsteht ein mittleres Gesicht. Werden von besonders attraktiven Frauen
die Gesichter übereinandergelegt entsteht ein \enquote{superschönes} Gesicht. In Befragungen welches Gesicht
das attraktivste ist sind die übereinandergelegten oft an erster Stelle.

Symmetrie ist dabei sehr wichtig, und war es schon immer. Der römische Architekt Vitruv (8427 vor
Christus) kam durch seine Proportionsstudien zu der Erkenntnis, dass der Abstand zwischen den Augen
genauso groß ist wie die Augen selbst und die Nasenbreite.
Manche Schönheitschirurgen arbeiten immer noch nach seiner Proportionslehre, obwohl sie von
Wissenschaftlern nicht unterstützt wird. Total symmetrische Gesichter wirken zu perfekt,
sodass kleine Asymmetrien dem Gesicht Lebensfreude und Wärme verleihen.

Eine glatte Haut ist ebenfalls sehr wichtig und wurde auf den Bildern durch das
Übereinanderlegen stark ausgeprägt.

Bei Männern ist es schwerer einzelne Faktoren zur Attraktivitätsbestimmung zu finden.
Dies liegt an dem Geschlechtshormon Testosteron, es sogt für die männlichen Züge und wirkt auf jeden
Betrachter anders.

Weltweit unterziehen sich nach Schätzungen jährlich zehn Millionen Menschen einer
Schönheitsoperation. In Deutschland rund \numprint{600000}.

\newpage
Bei Frauen wichtige Merkmale für ein \enquote{sexy-Gesicht} sind:
\begin{multicols}{2}
\begin{itemize}
	\item eine braune Haut
	\item ein schmales Gesicht
	\item wenig Fett
	\item volle, gepflegte Lippen
	\item ein großer Abstand zwischen den Augen
	\item dunkle schmale Augenbrauen
	\item lange, dunkle Wimpern
	\item hohe Wangenknochen
	\item eine schmale Nase
	\item keine Augenringe
	\item dünne Augenlider
\end{itemize}
\end{multicols}

Bei Männern wichtige Merkmale für ein \enquote{sexy-Gesicht} sind:
\begin{multicols}{2}
\begin{itemize}
	\item hohe Wangenknochen
	\item tiefe Brauen
	\item schmale Augen
	\item starker Bartwuchs
	\item ein markantes Kinn
\end{itemize}
\end{multicols}


\section{Unser heutiges Schönheitsbild}
Unser heutiges Schönheitsideal ist einerseits geprägt von dem Ideal welches in den 1960gern entstand,
Frauen sollten dünne knabenhafte Körper haben. Neu kommt hinzu, dass Frauen ohne ein Gramm Fett aber
weibliche Rundungen wie Brüste, Hüften und Hintern haben müssen.

Der Druck diesem Ideal zu entsprechen ist höher den je, durch Werbung, Fernsehsendungen (zum Beispiel
Germany’s Next Topmodel) sowie zahlreiche Diät und Schönheitsoperationen in Magazinen und
Ratgebern. Immer jünger werden die Kinder welche ihren Körper nicht als schön erachten und deshalb
gerne durch ein oder mehrere Operationen ihren Körper verändern wollen.

92\Prozent{} aller 15 bis 17 jährigen Mädchen weltweit gerne mindestens eine Stelle an ihrem
Körper verändern würden, die meist gewünschten Operationen sollen ein größeres Dekolleté,
einen strammeren Po oder einen flacheren Bauch schaffen.

Schönheitsoperationen sind bis jetzt für unter 18 Jährige nur mit der Zustimmung von
Erziehungsberechtigten erlaubt, der Bundestag plant jedoch, diese komplett zu verbieten. Dies ist
sehr umstritten, da in manchen Fällen Schönheitsoperationen eine große Hilfe sein können. Wer früher
mit Hasenscharten, schweren Verbrennungen und großen Verletzungen nach Unfällen leben musste, kann
heute durch die rekonstruktive plastische Chirurgie oft gleich oder annähernd gleich leben wie vor
dem Ereignis.

Dies zeigt, dass Schönheitsoperationen nicht nur für eine gestrafte Haut und eine größere Brust
verantwortlich sind, sondern eine sehr wichtige Hilfe sein können, dadurch wird die Entscheidung,
wann eine Operation sinnvoll ist, sehr individuell. Kritiker kritisieren an dem neuen
Gesetzesentwurf, dass er die rekonstruktive plastische Chirurgie schwerer zugänglich macht.

\begin{wrapfigure}{r}{0pt}
	\includegraphics[width=5cm]{files/images/Hannah/alles-17_1}%
	\captionof{figure}{Botox Frau}%
	\label{fig:Botox}
\end{wrapfigure}
Jedes Jahr werden mehr Schönheitsoperationen durchgeführt, mittlerweile wird auf jährlich über eine
Million kosmetische Operationen geschätzt. Bei Männern kommen die Schönheitsoperationen auch in Mode.
Die meisten durchgeführten Operationen sind für die Entfernung von Narben oder Muttermalen im
Gesicht, das absaugen von Fett, Faltenlifting und bei Frauen Brustvergrößerungen verantwortlich.

Botox (Botulinumtoxin), ein Nervengift welches die Haut straffen soll sorgt für ein Lahmlegen der
Gesichtsmuskulatur, damit diese keine Falten mehr bilden kann, das Ziel, das das Gesicht jünger
aussehen soll, wird dabei häufig verfehlt. Vor allem nach mehreren Anwendungen sieht das Gesicht aus
wie Wachs und die natürliche Gesichtsbewegung wird eingeschränkt.

Gesichter wie auf dieser Abbildung \vref{fig:Botox} zu sehen sind, sind in unserem Alltag
keine Besonderheit mehr und gründen eindeutig auf zu viel Botox.

\section{Magersucht und Bulimie, Krankheiten unserer Zeit?}
\enquote{Lieber tot als fett} ist ein Satz, welchen man nicht selten in
\href{http://de.wikipedia.org/wiki/Pro-Ana}{Pro-Ana} Foren liest.

In diesem Satz liegen letztendlich all die Gefahren der Magersucht (Anorexia nervosa), welche eine
sehr starke psychische Krankheit ist. In Foren werden Rituale und Gebote angeboten, welche die Sucht
verherrlichen. Jeder zehnte Fall endet tödlich doch dies schreckt nicht ab, wie man meinen sollte.
Das Leben dreht sich nur noch darum möglichst wenig zu essen und dies vor den anderen zu
verheimlichen. So kann in den Foren ein Freund gefunden werden, oft sind dies die einzigen Plätze an
denen sich die Mädchen (immer häufiger auch Jungs) über ihre Probleme und ihren
nicht ganz gewöhnlichen Alltag austauschen können. Das Internet ist somit einerseits gefährlich,
weil es viele erfahrene Leute gibt, die berichten wie sie ihre Familie und Freunde zu
täuschen versuchen, andererseits finden
dort aber viele Menschen Hilfe um den ersten Schritt aus ihrer Krankheit zu wagen. Durch den
unpersönlichen Rahmen trauen sich viele das erste mal im Internet über ihre Sucht zu reden und sich
Hilfe zu suchen.

Das Robert-Koch Institut leitete die erste deutsche umfangreiche Studie zum Thema Magersucht mit
\numprint{17600} Jugendlichen. Nach ihr sind bereits 30\Prozent{} aller 17-jährigen Mädchen
Essgestört. \numprint{100000} Mädchen und Frauen im Alter von 15 bis 35 Jahren leiden in Deutschland
an Magersucht, \numprint{600000} an Bulimie. Mittlerweile sind 5\Prozent{} der Magersüchtigen Jungs,
die Zahlen steigen zusehends weiter. Oft trauen sich Jungs nicht sich zu der vermeintlichen
\enquote{Mädchenkrankheit} zu bekennen und erliegen somit noch schlimmer den Spätfolgen.

Bulimie wird auch Ess-Brech-Sucht genannt und ist etwa dreimal so häufig wie Magersucht. Bulimische
Mädchen und Jungs haben ebenfalls wie Magersüchtige Angst zu dick zu werden sind jedoch im
Unterschied zu ihnen oft normalgewichtig, deshalb fällt diese Krankheit oft erst sehr spät auf. Sie
versuchen so wenig wie möglich oder gar nicht zu essen, verlieren dabei jedoch hin und wieder die
Kontrolle und bekommen \enquote{Fress-Attacken}. Die Betroffenen essen in kurzer Zeit soviel wie nur
irgendwie geht, nach einem Anfall erbrechen sie um die Kalorien wieder loszuwerden. Mit der Zeit
kommt der Brechreflex automatisch und die anfängliche Überwindung schwindet.

Oft nehmen Menschen die unter Bulimie leiden zusätzlich noch Medikamente zum Beispiel Abführmittel
ein, um an Gewicht zu verlieren, zusätzlich treiben viele noch übermäßig viel Sport. Bulimiekranke
versuchen wie Magersüchtige ihre Krankheit geheim zu halten und sind somit Meister des Verstecken.
Die Schokoladenstückchen verschwinden während man davon schwärmt wie toll sie ist in der Hosentasche,
einfach nur damit niemand etwas merkt. Selbstständig kommt fast niemand von Magersucht oder Bulimie
fort. Oft fehlt die Kraft um dem altgewohnten Muster zu entkommen.

Bei Bulimie droht als Folge des häufigen Erbrechens oft starkes Karies, eine chronisch wunde
Speiseröhre und Herz-Kreislauf-Probleme.

Bulimie ist eine erst zunehmende Suchtkrankheit welche man mit Alkohol oder anderen
Drogenabhängigkeiten vergleichen kann.

\begin{figurewrapper}
	\subfigure%
		{\includegraphics[width=0.5\textwidth]{files/images/Hannah/alles-19_1}} \hfill
	\subfigure%
		{\includegraphics[width=0.4\textwidth]{files/images/Hannah/alles-19_2}}%
	\caption{Magersüchtige Models}
\end{figurewrapper}


\section{Body Modification}
Body Modification\footnote{auf deutsch Körperveränderung} gründet auf die früheren traditionellen
Veränderungen des Körpers. In alte Kulturen wurde aus verschiedenen Gründen der Körper verändert, um
unter anderm Clanzugehörigkeiten zu erkennen aber auch um Status zu zeigen.

Der moderne Begriff beinhaltet jedoch mehr: die allgemeine Veränderung des Körpers wobei meistens
Schmuck, Schminke und Ähnliches ausgeklammert ist. Body Modifications entstehen meistens in einem
langen und oft mit Schmerzen verbundenen Prozess.

Viele stellen sich die Frage, warum man sich selbst solche Schmerzen zufügt und ob man damit sein
ganzes Leben verbringen will.

Die meisten Gründe sind Ästhetik, Provokation, sexuelle Bereicherung sowie spirituelle Gründe. Man
darf jedoch nicht vergessen, dass es viele Gründe gibt und die Fabrizierenden es sich gut überlegen,
oft sind dahinter sehr eigene, private Erlebnisse.

Die Bewegung der Modern Primitives ist die Bewegung der Body Modifications und bei uns noch nicht
sehr verbreitet. Auch deshalb werden Piercings meistens schräg angeschaut und mit negativen
Vorurteilen belegt.

Im Folgenden will ich verschiedene Arten der Body Modification sowie einige Vertreter zeigen.

\subsection{Piercing}
Piercings sind wohl das bekannteste und am weitesten verbreitete Modifizierungs mittel. Piercings
kann man schießen oder stechen lassen, dabei ist das Stechen jedoch das genauere und für den Piercer
schwierigere. Hygiene ist wie bei allen Veränderungen sehr wichtig!

\subsection{Dehnen}
Als Erweiterung von Ohrlöchern oder anderen Piercings kann man durch Stäbe oder Schnecken das Ohrloch
dehnen dadurch entsteht ein größeres Loch in welches man Plugs einsetzten kann. Wichtig ist langsam
zu dehnen und keine Dehnschritte zu überspringen. Bis zu \SI{1,2}{\milli\metre} schliesen sich
gedehnte Ohrlöcher wieder!

\subsection{Tätowieren}
Ist sehr schmerzhaft und bleibt für immer. Das Motiv sollte deshalb gut überlegt werden, genauso die
Stelle. Bei Menschen die Body Modification betreiben ist meistens der ganze Körper oder sehr große
Teile davon tätowiert.

\subsection{Brandings}
Wurde ursprünglich bei Pferden gemacht um sie zu kennzeichnen. Bei Menschen werden die Muster aus
mehreren kleinen Eisenteilen nach geformt, erhitzt und für einige Sekunden auf die Haut gepresst. Die
Schmerzen werden als aushaltbar beschrieben. Die Narben bleiben circa 5 bis 7 Jahre auf der Haut zu
sehen, bis sie vollständig verheilt sind.

\subsection{Cutting}
Cutting ist das schneiden von Ziernarben in die Haut und wurde von vielen alten Völkern besonders aus
Nordamerika, Asien und Afrika praktiziert.

\begin{figurewrapper}
	\includegraphics[width=7cm]{files/images/Hannah/alles-21_1}%
	\captionof{figure}{Besonders aufwändiges, frisches cutting}%
\end{figurewrapper}

\subsection{Suspensions}
Suspensions haben eine lange Tradition bei den amerikanischen Völkern sowie in Indien. Die Menschen
bekommen Haken in die Haut an denen man sich aufhängen lassen kann, einige Minuten oder auch einige
Stunden je nach Vorliebe. Die Schmerzen sollen gut auszuhalten sein, die häufigsten Gründe sind:
Neugierde, Adrenalin, als Werkzeug der Spiritualität und um die eigenen Grenzen zu erfahren.

\begin{figurewrapper}
	\includegraphics[width=7cm]{files/images/Hannah/alles-21_2}%
	\captionof{figure}{Suspernsionskünstler}%
\end{figurewrapper}

Dies sind einige Bespiele was Bodiemodification beinhaltet, wie man seinen Körper verändern kann und
will, der Kreativität sind jedoch keine Grenzen gesetzt,
von gespaltenen Zungen über Feenohren und Implantate
reicht die Spanne.

Dazu ist Erik Sprague (siehe Abb.~\vref{fig:Erik_Sprague}) ein gutes
Beispiel. Er veränderte sein Aussehen so das er einer Echse gleicht.


\begin{figurewrapper}
	\subfigure[Erik Sprague \enquote{The Lizardman}, geboren am \printdate{12.06.1972}%
		\label{fig:Erik_Sprague}]%
		{\includegraphics[height=9cm]{files/images/Hannah/alles-22_2}} \hfill
	\subfigure[Rick Genest \enquote{The Zombie man}, geboren am \printdate{07.08.1985}]%
		{\includegraphics[height=9cm]{files/images/Hannah/alles-22_1}}%
	\caption{Body Modification Künstler}
\end{figurewrapper}

\begin{figurewrapper}
	\subfigure[Dennis Avner \enquote{stalking cat}, geboren am \printdate{27.08.1958}]%
		{\includegraphics[width=0.55\textwidth]{files/images/Hannah/alles-23_1}} \hfill
	\subfigure[Etienne Dumont]%
		{\includegraphics[width=0.35\textwidth]{files/images/Hannah/alles-23_2}}%
	\caption{Body Modification Künstler}
\end{figurewrapper}

\clearpage
\section{Schlussworte}
Durch meine Arbeit an diesem Projekt hab ich sehr deutlich gemerkt, was für ein verzerrtes
Schönheitsbild wir haben. Wie
sehr wir unbewusst darin manipuliert werden und wie sehr wir daran haften.
Es hat mir Spaß gemacht mich mit einem so alltäglichen und eigentlich unbewusst klaren Thema zu
beschäftigen.

Ich hoffe in Zukunft öfters daran zu denken, dass unser Schönheitsbild nicht alles ist
und wir unsere Meinung über Menschen nicht so sehr Manipulieren lassen sollten.

Mit meiner Gruppe kam ich gut aus und gemeinsam kamen wir gut zurecht.

\nocite{
	koerperkultur:korper-und-mode-korperinszenierung,
	Mueller:im_kampf_mit_dem_eigenen_koerper,
	Spiegel:body-modification-lust-am-horror-koerper,
	Lotze:Bodymodification.org,
	Wikipedia:Lippenteller,
	Wikipedia:Padaung,
	sweetminds:schonheitsideale-im-wandel,
	Wikipedia:Attraktivitaetsforschung,
	Wikipedia:Attraktivitaet,
	Wikipedia:Schoenheitsideal,
	crossdress:schonheitsideale-im-wandel,
	rpi-ekhn:Schoenheitsideale,
	Moeller:Schoenheitsformel,
	beautycheck.de:Durchschnittsgesichter,
}

\clearpage
\printbibliography[heading=source,keyword=Hannah]
