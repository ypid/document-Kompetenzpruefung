\chapter{Hannah Haid}
\label{sec:Hannah_Haid}

\section{Einleitung}
Schon Kant sagte „Schönheit muss nicht nützlich sein“ trotzdem beschäftigt sich jeder Mensch damit
und das ist schon immer so. Wir wollen schön sein dafür sind wir bereit viel zu tun und hohe
gesundheitliche Schäden auf uns zu nehmen. In jeder alten Kultur sind schönheitssteigernde Dinge zu
finden und das trotz sehr verschiedener Ideale. Tattoos, Schmuck und Ohrlöcher gehören zu den
häufigsten Veränderungen, Ziernarben Intimpiercings und vieles mehr sind jedoch auch weit verbreitet,
sie sollten meist Clanzugehörigkeiten \fxnote{ungewöhlicher Ausdruck} oder der familiäre Status
sichtbar machen.

Schönheitsideale sind im ständigen Wandel und von sehr vielen, teilweise sehr lang zurückliegenden
Faktoren abhängig. Eindeutige Strukturen, welche Allgemein gültig sind, gibt es kaum. Das vielleicht
wichtigste ist jedoch das in Zeiten, wo Menschen im Wohlstand leben schlanke Figuren als schön
gelten, in Hungerzeiten sind dicke Menschen meist schöner. Dies kann man heutzutage an den
Drittweltländern gut beobachten, obwohl sich der Schönheitstrend immer mehr an der westlichen Welt
orientiert.

In unserer Welt werden die Ideale immer verrückter, Schönheitsoperationen immer häufiger und
normaler. Umso schlanker umso schöner lautet die Devise, wer dem Bild nicht entspricht erlebt oft
einen großen Zwang sich anzupassen oder hat mit hohen Einbußen zu rechnen.
Im Folgenden will ich auf die Entwicklung der Schönheit im Laufe der Zeit und in verschieden Kulturen
eingehen, sowie auf unser Schönheitsbild welches viele verschiedene Formen kennt, von Body
Modification zu Operationen und \href{http://de.wikipedia.org/wiki/Bulimie}{Bulimie}.

\clearpage
\section{Schönheit in verschiedenen Kulturen}
Verschiedene Kulturen haben verschiedenen Schönheitskulturen, mein erstes Beispiel sind die
Tellerlippen in Afrika.

\subsection{Tellerlippen}
Vor allem Mursi und Surma Frauen, welche in Äthiopien leben, dehnen ihre Lippen. Am Ende der Pubertät
bekommen die Mädchen einen Schnitt in die Unterlippe, außerdem werden die unteren Schneidezähne
entfernt, in das entstehende Loch wird eine Tonscheibe eingesetzt, die über längere Zeit durch immer
größere ersetzt wird, die Lippen werden bis zu \SI{15}{\centi\metre} gedehnt.

Umso größer der Teller ist umso größer ist das Ansehen der Frau und umso höher ist ihr Brautpreis.
Heutzutage verdienen sich viele Frauen durch ihre Lippenteller Geld von Touristen, welche für Fotos
mit ihnen zahlen, durch dieses Geld werden Missernten ausgeglichen. In ihrem normalen Leben werden
die Teller jedoch nicht verwendet, zum essen und arbeiten nehmen die Frauen die Teller raus, erst
wenn sich Touristen nähern werden die Scheiben eingesetzt.

Es wird angenommen, dass die Lippenteller früher nicht als attraktiv empfunden wurden, sondern die
Frauen vor Sklavenhandel schützen sollten.

\begin{figurewrapper} %% Einbindung steht im Konflikt mit dem Urheberrecht
	\subfigure[Frau ohne Tellerlippenring]%
		{\includegraphics[width=0.45\textwidth]{files/images/Hannah/alles-4_2}} \hfill
	\subfigure[Frau mit	Tellerlippe]%
		{\includegraphics[width=0.45\textwidth]{files/images/Hannah/alles-4_1}}%
	\caption{Tellerlippen}
\end{figurewrapper}

\newpage
Bekannte Völker, welche Lippenteller verwenden sind:
\begin{multicols}{2}
\begin{itemize}
	\item die Botokudo-Frauen und Männer in Brasilien
	\item die Kayapo in Brasilien
	\item die Kichepo in Äthiopien
	\item die Makonde in Tansania und Mosambik
	\item die Mursi-Frauen in Äthiopien
	\item die Sara-Frauen im Tschad
	\item die Surma-Frauen in Äthiopien
	\item die Suyá-Männer in Brasilien
\end{itemize}
\end{multicols}

\subsection{Giraffenhals}
\begin{wrapfigure}{r}{0pt}
	\includegraphics[width=5cm]{files/images/Hannah/alles-6_2}%
	\captionof{figure}{Kind mit den ersten Ringen}%
\end{wrapfigure}

Die Padaung, ein Bergvolk in Myanmars sind bekannt für ihre langen Hälse auch unter dem Namen
Giraffenhals bekannt.

Im Alter von fünf Jahren bekommen die Kinder ihre ersten Reifen um den Hals, welche in einem
mehrstündigen Ritual angepasst werden. Nach und nach werden es immer mehr Reifen. Es sind nicht, wie
oft angenommen einzelne Reifen, sondern Messingspiralen, welche extra für diesen Zweck in Myanmars
angefertigt werden. Die ersten Spiralen sind ca. \SI{10}{\centi\metre} hoch mit einem Durchmesser von
\SIrange{30}{40}{\centi\metre}, alle zwei bis drei Jahre, je nach Wachstum der Mädchen werden neue
Spiralen angelegt. An den Schultern werden etwas breitere Spiralen getragen, welche nochmals das Bild
der langen Hälse verstärkt. Insgesamt können die Spiralen eine Länge von bis zu \SI{30}{\centi\metre}
erreichen, dies sind 20 bis 25 Windungen.

Die Reifen können bis zu \SI{10}{\kilo\gram} wiegen und drücken somit die Schultern nach unten,
dadurch sieht der Hals lang aus. Durch diesen Schmuck müssen die Frauen einen Großteil ihrer
Beweglichkeit einbüßen, dies erschwert das Schlucken, außerdem ist komplizierte Hygiene nötig und
eine andauernde Last mitzutragen. Die weitverbreitete Annahme das die Frauen sich durch das Abnehmen
der Reifen das Genick brechen würden ist jedoch nur ein Mythos, die Halsmuskulatur ist geschwächt und
muss am Anfang geschont werden, baut sich jedoch wieder auf. Es ist bekannt, dass es schon im
Mittelalter Frauen mit Giraffenhälsen gab, die Entstehung ist jedoch von vielen Sagen umwoben.

Es wird erzählt, dass die Messingringe die Frauen vor Tigerbissen schützen sollten, andere meinen
dass der Halsschmuck auf den Ursprungsmythos der Paduas zurückzuführen ist. Nach ihm stammen die
Padua von einem gepanzerten weiblichen Drachen ab und ihr Halsschmuck soll an die „Drachenmutter“
erinnern.

Erst durch die Missionierung 1820 wurde der Schmuck als Schönheitsideal und Statussymbol umgedeutet,
seither ist es ein Zeichen für Reichtum, Würde und Erhabenheit.

\begin{figurewrapper} %% Einbindung steht im Konflikt mit dem Urheberrecht
	\subfigure[Traditionell gekleidete Frauen]%
		{\includegraphics[height=8cm]{files/images/Hannah/alles-5_1}} \hfill
	\subfigure[Giraffenhals voll gedehnt]%
		{\includegraphics[height=8cm]{files/images/Hannah/alles-6_1}}%
	\caption{Giraffenhals}
\end{figurewrapper}
\fxwarning{Traditionell gekleidete Frauen -- Falsche Bildunterschrift?}

\subsection{Lotosfuß}

\begin{wrapfigure}{r}{0pt}
	\includegraphics[width=5cm]{files/images/Hannah/alles-7_1}%
	\captionof{figure}{Normaler und verformter Fuß}%
\end{wrapfigure}
Über mehrere Jahrhunderte wurde in China der so genannte „Lotosfuß“ als Schönheitsideal fabriziert.
Dies gilt als besonders barbarisch da die Entstehung eines solchen Fußes ca. 15 Jahre lang dauert.
Den Mädchen wurden im alter von 3 bis 5 Jahren vier Zehn gebrochen und nach hinten, unter die
Fußsohle gebunden und dort mit festen Bandagen umwickelt, welche jeden Tag enger geschnürt wurden.
Dabei starben die Zehen oft ab oder fingen unter den Stoffwickeln an zu faulen.

Das Ideal war der „goldene Lotosfuß“ welcher eine Fußlänge von \SI{10}{\centi\metre} besaß, dies
entspricht der europäischen Schuhgröße 17, welche Kinder im alter von 2 bis 3 Monaten haben. Dieses
Ideal wurde jedoch nur von wenigen Frauen erreicht, normal war eine Fußlänge von
\SIrange{13}{14}{\centi\metre}. Mit dem Lotosfuß waren lebenslange Schmerzen verbunden, diese nahm
die Familie jedoch gerne in kauf da es eine Garantie für Ansehen und eine gute Hochzeit war. Die
Frauen konnten nie mehr normal gehen und erst recht keine größeren Strecken zurücklegen. Dies passt
gut in die Frauenrolle der damaligen Zeit durch die Füße mussten die Frauen zuhause bleiben, hatten
kein Chance selbstständig zu werden und mussten sich somit den Männern unterwerfen und sich
ausschließlich dem Haushalt widmen.

Da diese Füße für Feldarbeit unbrauchbar waren, wurden Töchtern von sehr armen Bauern verschont.

Der Ursprung dieses Brauches führt ins Jahre 975 zurück. Der damalige Kaiser Li Yu schenkte seiner
Geliebten, welche Tänzerin war eine Bühne, die als goldne Lotosblüte geformt war. Um auf ihr tanzen
zu können bandagierte sie sich die Füße, jedoch lang nicht so stark, die schmerzhafte Verkrüppelung
und Verstümmelung kam erst später.

1911 wurde dies Verboten allerdings ohne Erfolg, erst 1949 nachdem
die Volksrepublik China gegründet wurde gelang es Mao Zedong diesen Brauch endgültig zu verbieten,
1988 schloss die letzte Fabrik, welche Spezialschuhe anfertigte. Es gibt somit immer noch ältere
Frauen deren Füße abgebunden sind.

\begin{figurewrapper}
	\subfigure[Frau mit verformtem Fuß]%
		{\includegraphics[width=0.47\textwidth]{files/images/Hannah/alles-8_1}} \hfill
	\subfigure[Spezialschuhe für Lotosfüße]%
		{\includegraphics[width=0.43\textwidth]{files/images/Hannah/alles-8_2}}%
	\caption{Lotosfuß}
\end{figurewrapper}


\section{Schönheit in verschiedenen Zeiten}

Im Folgenden will ich einen kurzen Überblick über die Epochen und Schönheitsideale geben.

\subsection{Mittelalter}

\begin{wrapfigure}{r}{0pt}
	\includegraphics[width=5cm]{files/images/Hannah/alles-9_1}%
	\captionof{figure}{Typische Kleidung im Mittelalter}%
\end{wrapfigure}
Das Mittelalter war geprägt vom Christentum. Walter von der Vogelweide beschrieb das Frauenideal im
13. Jahrhundert als \enquote{edeliu schoene Frouwe reine} in seinen Minneliedern. Einfach und
schmucklos war die Devise, Schminke und übermäßige Pflege des Körpers galten als Heidnisch so dass
vor allem schönen Frauen mit Misstrauen begegnet wurde, war es doch Eva welche durch die Vertreibung
aus dem Paradies als gefährliche Verführerin galt. Trotz allem verfasste der gelehrte Gilbertus
Anglicus ein zwölfteiliges Dossier über die Kosmetik und Pflege von Gesicht und Haaren. Sein Werk war
richtungsweisend hatte jedoch im Spätmittelalter wenig Einfluss.

Natürlich war Schönheit nicht völlig egal so dass eine möglichst helle Haut als schön und
erstrebenswert galt, wer es sich leisten konnte mied die Sonne, regelmäßige Aderlässe verstärkten die
Blässe. Hygiene war unwichtig, man verdächtigte sie im übertriebenen Masse sogar als Wegbereiter der
Pest. Wasser und Seife wurden durch Parfum und Puder ersetzt.

\subsection{Renaissance/Barock/Rokoko}

\begin{figurewrapper}
	\subfigure[Schönheitsideal in der Renaissance\label{fig:RenaissanceOne}]%
		{\includegraphics[height=10cm]{files/images/Hannah/alles-10_1}} \hfill
	\subfigure[Gemälde von Ruben, sehr ausschlaggebend für das Schönheitsideal in der Renaissance]%
		{\includegraphics[height=10cm]{files/images/Hannah/alles-10_2}}%
	\caption{Schönheitsideal in der Renaissance}
\end{figurewrapper}

Die Adlige Lucrezia Panciatichi (siehe Abb.~\vref{fig:RenaissanceOne}) galt als typische
Schönheit im Zeitalter der italienischen Renaissance. „Rubensfigur“ oder „barocke Lebensart“ sind
wohl die bekanntesten Begriffe des Barocks.
\fxnote{Zuordnung zu den Bildern überprüfen}


Diese beiden Begriffe beinhalten eine ausschweifende Lebensart in welcher der Genuss sehr wichtig ist
(1600--1720). Was auf Bildern zu sehen ist, war jedoch in der Realität anders, das Korsett kam in
Mode und damit die künstliche Wespentaille, das gelockt oder wellende Haar musste in der
Öffentlichkeit gescheitelt und nach hinten frisiert werden.
Dies blieb im Rokoko ähnlich um „schön“ zu sein mussten Haare aufgesteckt, gepudert und mit Blumen,
Schleifen und Ähnlichem geschmückt werden

Bei Männern kamen Perücken in Mode, die Locken umspielten Wangen und Kinn.
Zurzeit von Ludwig XVI kamen hochtoupierte Frisuren in Mode und damit entstanden die ersten Perücken
für Frauen. Bleiche Haut, Rouge rote Wangen, kohlengeschwärzte Augenbrauen und hohe Frisuren wurden
chic\fxwarning{?}. Parfum war angesehen\fxwarning{Steigerungsform?} als waschen, sodass die Menschen
von großen Duftwolken umgeben waren.

\subsection{Klassizismus}
Im Klassizismus ist Natürlichkeit sehr wichtig, wie bei den Römern ist eine über die Hüfte
verschobene Taille, ein rosiges Gesicht und ausdrucksvolle, möglichst wenig geschminkte Augen in
Mode.

Im 20. Jahrhundert ändern sich die Schönheitsideale immer schneller. Korsagen werden der
Vergangenheit angehängt dadurch konnten überflüssige Kilos jedoch nicht mehr versteckt werden,
sondern mussten abgespeckt werden. Zu Beginn der 20er Jahre änderte sich das Schönheitsideal
grundlegend. Haare wurden wenn möglich in weichen Wellen, Kinnlang getragen, ein kleiner Kopf, große
Augen und ein voller Mund wurden zum neuen Ideal erklärt.

Unser heutiges Schönheitsideal ist schlank. Magersucht ist eine häufige Krankheit unserer Zeit. Für
unser verdrehtes Schönheitsbild wird die Barbie oft als mitverantwortliche verschrien. Im Folgenden
möchte ich kurz auf die Barbie, ihre Entstehungsgeschichte, ihre Auswirkungen und die Kritik an ihr
eingehen.

\begin{figurewrapper}
	\includegraphics[width=8cm]{files/images/Hannah/alles-11_1}%
	\captionof{figure}{Schönheitsideal im Klassizismus}%
\end{figurewrapper}

\newpage
\section{Die Barbiepuppe}

\begin{wrapfigure}{r}{0pt}
	\includegraphics[width=7.5cm]{files/images/Hannah/alles-12_1}%
	\captionof{figure}{Die erste Barbie}%
\end{wrapfigure}
Barbie wird von der Firma Mattel produziert, diese wurde 1945 von Ruth und Elliott Handler gegründet.
Am \printdate{09.03.1959} wurde die erste Mattel-Puppe auf dem American Toy Fair in New York
präsentiert und nach Ruth und Elliotts Tochter Barbara benannt.

Barbie orientiert sich immer an den aktuellen Mode Trends, deshalb hat sich ihre Figur oft geändert.
Was neu an Barbie war, ist das sie eine erwachsene Frau ist und somit nicht wie frühere Puppen den
Kindern helfen soll zu erziehen, die Aufgabe der Barbie ist das Spiel der Zukunft. Um dieses
spannender zu gestalten, hat Barbie viele Freunde und Verwandte diese sind jedoch sehr
unübersichtlich. Sie haben eigene Körper und Gesichter, in der Packung ist immer ein kleiner Zettel,
welcher die Beziehung zwischen Barbie und der Puppe beschreibt, was sie verbindet und wie sieh heißt.

Von Anfang an hatte Barbie eine vielfältige Garderobe für jede Gelegenheit. Ab den 90er Jahren trägt
Barbie Alltagskleidung und ist somit auch für Unter- und Mittelschicht erreichbar, davor hatte sie
aufwendige Kostüme und war sehr teuer.

Manche Kleidung der Barbie können Kinder durch Pailletten oder Stoffmalfarbe selbst Verändern und neu
kreieren. Seit 1980 gibt es auch afroamerikanische und hispanische Barbies.

\subsection{Kritik}
\begin{itemize}
	\item Die Proportionen der ersten Barbie waren 99-46-84 nach heftiger Kritik wurde die
		Oberweite verringert und die Taille vergrößert, die neuen Maße entsprechen immer noch
		nicht der Realität, ließen jedoch viele Kritiker verstummen.
	\item Feministen kritisieren an der Barbie das sie das traditionelle Frauenbild zeigt und zu
		kritiklosem Konsum anregt.
	\item Im September 2003 verbot Saudi-Arabien die Barbie da sie den Islam untergrabe, seit
		November 2003 gibt es das gegen Model Fulla zu kaufen.
	\item Die Wirkung der Barbie wird als demütigend und für das Schönheitsbild der Kinder prägend
		beschreiben, eine Frau wird niemals Barbies Masse haben können, Kindern wird jedoch durch das
		Spielen mit Babie genau dieser Körper als schön beigebracht
\end{itemize}

\subsection{Wissenswertes über die Barbie}
\begin{itemize}
	\item Jeder Deutsche kennt die Marke Barbie
	\item Pro Sekunde werden statistisch 3 Barbiepuppen verkauft
	\item Jedes Mädchen besitzt durchschnittlich 7 Barbie
	\item Ein Mensch mir Barbie Massen wäre nicht lebensfähig da im Unterleib nicht genug Platz
		für alle Organe wäre
\end{itemize}

Seit 2001 gibt es Filme über die Barbie.

\begin{figurewrapper}
	\includegraphics[width=7cm]{files/images/Hannah/alles-14_1}%
	\captionof{figure}{Barbie und Ken}%
\end{figurewrapper}

\section{Attraktivitätsforschung}
Die Menschheit beschäftigte sich schon immer mit Attraktivität in den letzten 20 Jahren kamen
weiterführende Erkenntnisse über die Attraktivitätsforschung hinzu.

Albrecht Dürer schrieb „Schönheit -- was das sey, weiß ich nit, obwohl sie vielen dingen anhanget.“
Leonardo da Vinci war ebenfalls auf der Suche nach dem perfekten Körper und seiner Erfassung in
Formeln.

Obwohl die Attraktivitätsforschung noch vor vielen ungeklärten Fragen steht, ist eine wichtige
Erkenntnis: „Eine Frau ist schön, wenn sie typisch weiblich aussieht, und ein Mann ist schön, wenn er
typisch männlich aussieht!“

Die Formel durch die Schönheit berechnet wird ergibt Attraktivitätswerte die von 1 für sehr
unattraktiv bis 7 für sehr attraktiv reichen.

Eine neu entwickelte Methode, welche Morphing genannt wird, schlaggebend für die Grundaussagen der
Attraktivitätsforschung. Bei ihr werden mehrere Fotos von Frauen welche ähnlich attraktiv sind
übereinandergelegt, dadurch entsteht ein mittleres Gesicht. Werden von besonders attraktiven Frauen
die Gesichter übereinandergelegt entsteht ein „superschönes“ Gesicht. In Befragungen welches Gesicht
das attraktivste ist sind die übereinandergelegten oft an erster Stelle.

Symmetrie ist dabei sehr wichtig, und war es schon immer. Der römische Architekt Vitruv (8427 vor
Christus) kam durch seine Proportionsstudien zu der Erkenntnis, dass der Abstand zwischen den Augen
genauso groß ist wie die Augen selbst und die Nasenbreite.
Manche Schönheitschirurgen arbeiten immer noch nach seiner Proportionslehre, obwohl sie von
Wissenschaftlern nicht unterstützt wird. Total symmetrische Gesichter wirken jedoch zu perfekt,
sodass kleine Asymmetrien dem Gesicht Lebensfreude und Wärme verleihen.

Eine glatte Haut ist ebenfalls sehr wichtig, wurde auf den Bilder aber teilweise durch das
übereinanderlegen der Bilder sehr ausgeprägt.

Bei Männern ist es schwerer einzelne Faktoren zur Attraktivitätsbestimmung zu finden.
Dies liegt an dem Geschlechtshormon Testosteron, es sogt für die männlichen Züge und wirkt auf jeden
Betrachter anders.

Weltweit unterziehen sich nach Schätzungen jährlich zehn Millionen Menschen einer
Schönheitsoperation. In Deutschland rund \numprint{600000}.

Bei Frauen wichtige Merkmale für ein „sexy-Gesicht“ sind:
\begin{multicols}{2}
\begin{itemize}
	\item eine braune Haut
	\item ein schmales Gesicht
	\item wenig Fett
	\item volle,gepflegte Lippen
	\item ein großer Abstand zwischen den Augen
	\item dunkle schmale Augenbrauen
	\item lange, dunkle Wimpern
	\item hohe Wangenknochen
	\item eine schmale Nase
	\item keine Augenringe
	\item dünne Augenlider
\end{itemize}
\end{multicols}

Bei Männern wichtige Merkmale für ein „sexy-Gesicht“ sind:
\begin{multicols}{2}
\begin{itemize}
	\item hohe Wangenknochen
	\item tiefe Brauen
	\item schmale Augen
	\item starker Bartwuchs
	\item ein markantes Kinn
\end{itemize}
\end{multicols}


\nocite{
	koerperkultur:korper-und-mode-korperinszenierung,
	Mueller:im_kampf_mit_dem_eigenen_koerper,
	Spiegel:body-modification-lust-am-horror-koerper,
	Lotze:Bodymodification.org,
	Wikipedia:Lippenteller,
	Wikipedia:Padaung,
	sweetminds:schonheitsideale-im-wandel,
	Wikipedia:Attraktivitaetsforschung,
	Wikipedia:Attraktivitaet,
	Wikipedia:Schoenheitsideal,
	crossdress:schonheitsideale-im-wandel,
	rpi-ekhn:Schoenheitsideale,
	Moeller:Schoenheitsformel,
	beautycheck.de:Durchschnittsgesichter,
}

\clearpage
\printbibliography[heading=source,keyword=Hannah]
