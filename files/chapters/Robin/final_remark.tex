\section{Schlusswort}
Für mich war zu Beginn der Arbeit schnell klar, dass ich den Teil der funktionellen Körperveränderung
übernehmen würde und ich konnte mich damit auch gut verbinden. Bei mir, wie auch bei Hannah und
Joshuah, nahm die Arbeitsgeschwindigkeit über das letzte halbe Jahr stetig zu. Wir begannen mit
Recherchen, Besprechungen und dem schreiben von ersten Texten. In der Intensivphase ging es dann
richtig ans Eingemachte. In dieser Zeit entstand ein Großteil dieser Arbeit. In trockene Tücher wurde
die Dokumentation aber erst die letzten paar Tage vor Abgabe gebracht. In dieser Zeit fiel mir noch
Einiges ein, was ich noch zu Papier brachte. Ebenfalls in den letzten zwei Tagen begann ich,
die Dokumente von Hannah und Joshuah zu
\href{http://de.wikipedia.org/wiki/Satz_(Druck)}{setzen}.

Zu dem Thema hätte ich sicher noch Einiges mehr schreiben können, aber um die Richtlinie bezüglich
der Seitenanzahl nicht zu stark zu strapazieren, spare ich mir das für Fragen bei der Präsentation
auf.

Mit Hannah und Joshuah kam ich gut zurecht. Wir haben öfters Gespräche über unsere
Kompetenzprüfungsinhalte geführt, sowohl in der realen Welt als auch übers Internet. Sie waren auch
bereitwillige Tester von \hyperref[sec:Robin:experiments:myCompassBelt]{meinem Kompass-Gürtel}.

Ich bin sehr zufrieden mit der Arbeit.
