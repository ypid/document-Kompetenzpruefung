\section{Geschichtlicher Überblick}
\label{sec:Robin:historical_overview}

Durch Kriege, Unfälle oder durch Krankheiten kann eine Person durch fehlende Gliedmaßen oder andere
Probleme deutlich eingeschränkt sein. Da Körperteile nicht einfach nachwachsen, musste sich die
Menschheit etwas ausdenken, um diese Komponenten anderweitig zu ersetzen. Für diesen Zweck wurden
viele Hilfsmittel und Methoden erfunden. Ein sehr früh in der Geschichte auftretendes Hilfsmittel ist
die Prothese. Eine Prothese ist ein Ersatz für Gliedmaßen oder Organe durch ein künstliches Produkt.
Ist die Prothese außerhalb des Körpers, spricht man von einer Exoprothese (in der Alltagssprache
meist nur als Prothese bezeichnet). Hierunter fallen Arm-, Bein-, Hand- und Fußprothesen. Wenn die
Prothese vollständig im Körper sitzt, spricht man hingegen von einer Endoprothese oder einem
Implantat. Beispiele sind künstliche Gelenke, Herzklappen oder die Ersetzung von Organen wie Herz
oder Leber. Allerdings lassen sich nicht alle Prothesen so klar unterteilen. Es gibt auch Prothesen,
die aus dem Körper herausragen wie zum Beispiel ein Zahnimplantat. Ich werde im folgenden die
wichtigsten Schritte in chronologischer Reichenfolge  beleuchten, die den heutigen Stand der
Entwicklung erst ermöglichten.
% Im weiteren Text werde ich die Begriffe Prothese und Implantat benutzen.

Die Prothetik, also die Entwicklung und Herstellung von Prothesen, hat eine lange Tradition. So gehen
Medizinhistoriker davon aus, dass bereits 2000 Jahre vor Christus amputierte Gliedmaßen durch
künstliche Körperteile ersetzt wurden. Dies erfüllten anfangs nur ästetische Zwecke, was sich im
Laufe der Geschichte aber deutlich änderte.

Die älteste gefundene Prothese wurde 600 vor Christus von einer Ägypterin benutzt. Sie wurde mit 50
Jahren mit einer hölzernen Zehprothese beigesetzt. Der künstliche Zeh zeigte deutliche
Abnutzungsspuren, sodass davon ausgegangen werden kann, dass es sich hierbei nicht um eine
Grabbeigabe für ein Leben nach dem Tod handelte, sondern um ein nützliches
Hilfsmittel.\footcite{Lancet:origin_prosthetic} Es gibt viele weitere archäologische Funde von frühen
Prothesen die über zwei Jahrtausende alt sind. So wurden in vielen Kulturen schon viele zerstörte
Zähne durch passende aus Elfenbein, Holz oder menschlichen Zähnen ersetzt. Diese neuen Zähne wurden
mit Draht an den anderen Zähnen befestigt. Doch dieser frühe Zahnersatz wurde nur aus ästetischen
Gründen und zu Verbesserung der Aussprache getragen. Beim Essen hatte dieser keine positive Funktion.
Im Gegenteil: Der Zahnersatz führte zu Entzündungen im Mund.

In der Geschichte wurden schon sehr früh Holzbeine eingesetzt, um fehlende Beine zu ersetzten. Es
handelte sich meist um eine einfache Holzstelze mit der eine Fortbewegung wieder möglich wurde.
Ein
Beleg dafür ist der französischer Pirat François Le Clerc, der aus diesem Grund auch den Spitznamen
\enquote{Holzbein} bekam. Er lebte im 16. Jahrhundert und verlor bei einem Überfall ein Bein.
%% http://de.wikipedia.org/wiki/François_Le_Clerc
Das Bild, das wir heute von Piraten haben, ist also nicht ohne Grund ein einbeiniger mit Holzbein,
der eine Augenklappe trägt. Aber auch schon die Ägypter nutzten Holzbeine.

Nach diesen einfachen Prothesen tauchte 1504 die sogenannte eiserne Hand des Ritters Götz von
Berlichingen auf. Nachdem die rechte Hand von Götz in einer Schlacht von einer Kanonenkugel
zerschmettert wurde und diese vorsorglich amputiert wurde, konstruiert ein Waffenschmied eine, für
diese Zeit, sehr fortschrittliche Handprothese, die auch längere Zeit danach einmalig blieb. Die
eiserne Hand sah wie ein Handschuh (siehe Abb.~\vref{fig:eiserneHand}) aus und wurde mit
Lederriemen am noch
\begin{wrapfigure}{r}{0pt}
	\href{\URLeiserneHand}{\includegraphics[width=5cm]%
		{files/images/Robin/Goetz-eiserne-hand1}%
	}
	\captionof{figure}{Die eiserne Hand des Ritters Götz von Berlichingen}%
	\label{fig:eiserneHand}
\end{wrapfigure}
vorhandenem Unterarm befestigt. Die Finger konnten gedreht und fixiert werden. Die Prothese konnte
außerdem nach oben und unten geschwenkt werden. So konnte Götz unter Zuhilfename seiner gesunden
linken Hand mit dieser Prothese Gegenstände greifen und festhalten. Er konnte sogar sein Schwert
halten und dadurch sein Beruf mit gewissen Einschränkungen, weiter ausüben.
%% http://de.wikipedia.org/wiki/Eiserne_Hand_(Götz_von_Berlichingen)

Im Mittelalter tauchten auch erste Implantate für ein fehlendes Auge auf. Diese Einlegeaugen waren
anfangs aus Edelmetallen wie Gold und Silber gefertigt. Im 17. Jahrhundert traten dann die ersten
Glasaugen auf. Diese Augenprothesen
wurden hauptsächlich eingesetzt,
um die Gesichtsharmonie wieder herzustellen.

Im 17. Jahrhundert wurden erste Hörgeräte erfunden. Dabei handelte es sich noch um einfache Hörrohre
die Umgebungsgeräusche um 20 bis 30 Dezibel verstärkten. Eine Verbesserung wurde erst 1878 von
Werner von Siemens mit einem speziellen Telefonhörer erfunden. 1914 wurde dieses Hörgerät mit
Einsteckhörer ausgestattet, um es unauffälliger zu machen. Die Geräte wurden nun immer weiter
miniaturisiert. 1952 kamen erstmals Transistoren zum Einsatz, wodurch die Größe auf eine
Zigarettenschachtel schrumpfte. 1966 brachte Siemens ein Hörgerät auf den Markt, das vollständig im
Gehörgang verschwand. Die nächsten Schritte waren dann der Einsatz von digitalen Signalprozessoren
und die damit verbundene Leistungssteigerung und weitere Miniaturisierung.

Nachdem die Entwicklung an Beinprothese viele Jahre fast stillstand, wurde das simple Holzbein im 19.
Jahrhundert bis zu einem Modell mit gefedertem Prothesenfuß und beweglichem Kniegelenk weiter
entwickelt. Dadurch konnte der Träger die Beinprothese beim Sitzen wie ein normales Bein am
Kniegelenk einklappen.

Die Wichtigkeit der Prothetik nahm im 20. Jahrhundert aufgrund der vielen Verstümmelten des Ersten
und Zweiten Weltkrieges enorm zu. Dies Sorgte für eine rasante Entwicklung in dieser Zeit. So wurden
immer mehr Prothesen mit Sprungfedern und Gelenken konstruiert. Es ist beispielsweise möglich
geworden, dass künstliche Kniegelenke beim Laufen mechanisch einknickten, wenn es zu einer
Schwerpunktverlagerung kam. Zusätzlich wurden Seilzüge und Bolzen dazu genutzt, die Kraft von noch
vorhandenen Muskeln zur Betätigung von einfachen Greifern zu benutzen. Eine Umsetzung dieser Idee
lieferte 1916 der deutsche Chirurg Ferdinand Sauerbruch mit dem nach ihm benannten Sauerbruch-Arm.
Für diese Handprothese wurde ein Kanal durch die Muskeln gelegt. In diesen wurden dann Bolzen
eingeführt, mit denen die Handprothese und sogar die Finger gesteuert werden konnten. Die Kraft, die
mit dieser Prothese ausgeübt werden konnte, war aber bei Weitem schwächer als die einer natürliche
Hand. Der Sauerbruch-Arm konnte sich nie richtig durchsetzten. Dies lag hauptsächlich an zwei
Problemen. Zum einen traten in dem Kanal oft Entzündungen und Infektionen
auf.\footcite{thesis:Karpa:Geschichte_Armprothesen}
Andererseits war die
Prothese für die meisten verwundeten Soldaten des Ersten Weltkrieges zu teuer.

Ende der 50. Jahre wurde der erste Herzschrittmacher, der vollständig im Körper untergebracht wurde,
bei Arne Larsson eingesetzt. Das Gerät war noch sehr einfach aufgebaut. Es bestand aus zwei
Transistoren, die eine Kippschaltung bildeten, einer Quecksilberakkuzelle und einer Spule zum
Aufladen des Akkus über Induktion. Das externe Nachladen war jede Woche nötig, da die Energiedichte
dieser Akkutechnologie noch deutlich unter der heutigen war. Die Bauteile wurden in einer mit
Epoxidharz versiegelten Schuhcremedose untergebracht. Aus dem Gehäuse ging ein Kabel zu den
Elektroden, die auf das Herz aufgenäht wurden.

Ein wesentliches Problem war die kurze Akkulaufzeit. Deshalb wurden in der nächsten Generation
von Herzschrittmachern der Zerfallsprozess von \SI{200}{\milli\gram} Plutonium-238 als Energiequelle
genutzt. Die Entsorgung dieser geringen Mengen radioaktiven
Materials stellt heute gewisse Probleme dar.\footcite{DRadio:strahlendes_Herz}
%% http://de.wikipedia.org/wiki/Radioisotopengenerator

1980 wurde es erstmals möglich, Armprothesen mit Motoren zu konstruieren. Dadurch musste nicht mehr,
wie dies noch bei dem Sauerbruch-Arm der Fall war, die Kraft von den verbleibenden Muskeln über
Mechanik zur Hand gelenkt werden. Diese Prothesen konnten über die Kontraktion noch vorhandene
Muskeln gesteuert werden.

\bigskip
Soviel zur Vergangenheit. An eine funktionelle Ersetzung komplizierterer Körperteile war vor einiger
Zeit noch nicht zu denken. Heute sind viele Eingriffe zur Routine geworden.
