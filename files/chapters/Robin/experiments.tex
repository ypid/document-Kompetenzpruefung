\section{Eigene Versuche -- ein sechster Sinn}
\label{sec:Robin:experiments}
In diesem Kapitel beschreibe ich eigene Versuche, die ich zur Erforschung der Erweiterbarkeit
des Menschen und der Anpassungsfähigkeit des Gehirns durchgeführt habe.


\newcommand{\Linkfeelspace}%
{\enquote{\href{http://feelspace.cogsci.uni-osnabrueck.de/}{Feel Space}}}
\subsection{Haptischer Kompass-Gürtel}
Die Universität Osnabrück entwickelte unter dem Projektnamen \Linkfeelspace{}
einen Gürtel, der über Vibration dem Träger die Himmelsrichtung mitteilt.
Dazu werden zwölf beziehungsweise 30 Vibrationsmotoren mit einem Gürtel um den Bauch angebracht.
Der Träger weiß dadurch, wo Norden ist,
ohne auf einen Kompass schauen zu müssen. Diese Erweiterung der Positionsbestimmung des Menschen
wird mit der Zeit unterbewusst vom Gehirn zur Navigation mit einbezogen. So berichtet Udo Wächter
nach dem Tragen des Gürtels über sechs Wochen, dass er sich besser in Osnabrück orientieren konnte.
Wächters Gehirn hat also eine neue Verbindung zwischen Reizung der
Bauchnerven\footnote{beziehungsweise Rückennerven} und der
Himmelsrichtung hergestellt.\footcite{noz:compass_belt}
Somit stand Wächter, nach einer Lernphase an die nun fühlbare Himmelsrichtung, ein neuer
Sinn für die Zeit des Experiments zur Verfügung.

\subsubsection{Nachbau eines Kompass-Gürtels}
\label{sec:Robin:experiments:myCompassBelt}
Da ich mir die Wirksamkeit dieses \enquote{sechsten Sinn} sehr gut vorstellen konnte, wollte
ich diesen auch selbst testen. Deshalb entschloss ich mich, eine einfache Variante dieses Gürtels
mit nur zwei Vibrationsmotoren nachzubauen. Da zwei Motoren allerdings fast schon das Minimum
darstellen und
damit noch keine wirkliche Orientierung möglich war, rüstete ich später noch auf vier Vibrationsmotoren
auf.
Ein weiterer Grund für die Entscheidung zum Nachbau war,
dass ich die meisten Komponenten und auch das nötige Wissen bereits besaß. Ein solcher Nachbau wurde
schon von ein paar technikinteressierten Menschen umgesetzt, allerdings fand ich keine Anleitung.
Deshalb überlegte ich mir selbst eine Schaltung.

Begonnen habe ich mit einem
\href{http://www.mikrokopter.de/ucwiki/MK3Mag}{elektronischen Kompass}%
\footnote{Nummer 1 auf Abb.~\vref{fig:Technik_compass_belt}}
vom \href{http://www.mikrokopter.de}{MikroKopter Projekt}.
Dieser gibt unter anderem eine Gradangabe aus. Daraus geht die Blickrichtung relativ zum Nordpol
hervor. Ich entschied mich für einen PWM-Ausgang am Kompassmodul, um die Gradangabe in einem
externen Mikrokontroller\footnote{\href{http://www.atmel.com/Images/doc8161.pdf}{ATmega328}, das
rechteckige schwarze Bauteil bei Nummer 2 auf Abb.~\vref{fig:Technik_compass_belt}}
weiter zu verarbeiten. Den Mikrokontroller lötete ich auf eine
Lochstreifenplatine. Eine direkte Verarbeitung der Gradangabe und die Ansteuerung der Motoren mit
dem Mikrokontroller des Kompassmoduls schloss ich aus, um mehr Steuerausgänge zur Verfügung zu
haben\footnote{und um nicht auf der Platine löten zu müssen. Zum Beispiel das direkte abgreifen
von Signalen am SMD-Mikrokontroller.}.

\begin{wrapfigure}{r}{0pt}
	\includegraphics[width=5cm]{files/images/Robin/compass_belt/IMG_7994-rm-bg\imageresize}
	\captionof{figure}{Komponenten meines Kompass-Gürtels}
	\label{fig:Technik_compass_belt}
\end{wrapfigure}
Den externen Mikrokontroller programmierte ich so, dass dieser einen von vier Steuerausgängen
für 150 Millisekunden anschaltet, wenn Norden entweder vorne, rechts, hinten oder links ist.
Um dieses schwache\footnote{Die maximale Belastbarkeit ist mit \SI{40}{\milli\ampere} pro Pin
angegeben.}
Steuersignal zum Schalten eines Vibrationsmotors verwenden zu können, verstärkte ich es über eine
Transistorschaltung. Diese baute ich auf Lochraster (\hyperref[fig:Technik_compass_belt]{Nr. 3})
auf. Die Vibrationsmotoren (\hyperref[fig:Technik_compass_belt]{Nr. 4}) baute ich aus alten
Handys
aus und verkabelte sie mit der Transistorschaltung. Ich opferte anfangs nur zwei Handys, da ich
bei einem Erfolg weitere Vibrationsmotoren sehr günstig
\href{http://www.pollin.de/shop/dt/NDA2OTg2OTk-/Motoren/%
Gleichstrommotoren/Vibrationsmotor_LA4_432A.html}{nachbestellen}
könnte.
Um die Vibrationsmotoren ohne Handy zu betreiben, brauchte ich noch ein Gehäuse, da sich sonst der Motor
nicht drehen kann. Bei den zwei ersten Handys schnitt ich einfach den Teil\footnote{Der Vibrationsmotor
sitzt meist ganz unten im Handy in der Nähe des Mikrofons, außer bei einem Motorola V300 (Klapphandy). Bei
diesem war er im unteren Teil des Display Gehäuses verbaut.}, wo der Motor verbaut war, aus dem Gehäuse
raus und hatte so schon passende Halterungen, die genug Platz für die Unwucht boten. Die Vibrationsmotoren
klebte ich mit Heißkleber in die Halterungen. Als ich dann noch zwei alte Handys %von einem guten Freund
bekam, entschied ich mich dafür doch keine Vibrationsmotoren zu bestellen.

\begin{wrapfigure}{r}{0pt}
	\includegraphics[width=5.1cm]{files/images/Robin/compass_belt/IMG_8070-rm-bg\imageresize}
	\captionof{figure}{Vibrationsmotoren mit Holzgehäuse}
	\label{fig:motors_compass_belt}
\end{wrapfigure}
Außerdem beschloss ich,
nicht mehr das Handygehäuse zu zerschneiden, um eine passende Hülle für die
Motoren zu bekommen, sondern selbst zwei aus Holz zu fertigen. Ich bohrte Löcher mit dem Durchmesser
der
Motoren in kleine Hartholzstücke und brachte diese dann in die Form von Quadern. Anschließend klebte ich
die Vibrationsmotoren im Holz fest. Dabei war darauf zu achten, dass sich die Unwucht weiterhin drehen
konnte.

Somit standen mir vier Vibrationsmotoren zur Verfügung.
Die komplette Technik brachte ich zusammen mit einem Akkupack
an einem Gürtel an. Da das Kompassmodul keine Beschleunigungssensoren zur Verfügung hat und somit
nicht weiß, wo \enquote{unten} ist, musste ich bei der Befestigung am Gürtel, auf eine waggerechte
Lage achten. Um ständiges Vibrieren zu vermeiden, änderte ich das Programm, sodass
nur eine Vibration an den Träger weiter gegeben wird, wenn sich dieser zur nächsten Position gedreht
hat.

Das erste Problem, was mir auffiel war, dass der Bauch sehr viel empfindlicher ist als der Rücken.
Dadurch kitzelte es am Bauch schon, während das Vibrieren am Rücken kaum wahrnehmbar war. Dieses
Problem löste ich, indem ich die Vibrationsstärke, über unterschiedliche Widerstände
vor den Basen der Transistoren, änderte.

Mein Kompass-Gürtel ist auf Abb.~\vref{fig:My_compass_belt} zu sehen. Der Bau hat mich circa 32
Stunden gekostet. Der Materialwert liegt ungefähr bei \EUR{80}.
\href{http://www.cogsci.uni-osnabrueck.de/NBP/peterhome.html}%
{Prof. Dr. Peter König} schrieb mir
in einer eMail, dass neben vielen hundert Arbeitsstunden Entwicklung, der Materialwert
von ihrer Entwicklung noch vierstellig ist.
Es gibt aber aktuell Bemühungen diesen Preis zu drücken.
Mein Nachbau ist also um einiges einfacher als die Entwicklungen des \Linkfeelspace{} Projekts.

Außerdem schrieb er, dass eines der größten Probleme der hohe Stromverbrauch war. Deshalb überprüfte
ich meine eigene Schaltung diesbezüglich. Ich stellte fest, dass die Schaltung im Betrieb
\SI{44}{\milli\ampere} bei \SI{5,35}{\volt} verbrauchte. Ein laufender Vibrationsmotor verbraucht
hingegen schon \SI{80}{\milli\ampere}\footnote{Ein sehr kleiner Vibrationsmotor verbraucht sogar
\SI{130}{\milli\ampere}. Dieser höhere Verbrauch ist notwendig, da man die Vibration wegen der
kleineren Unwucht sonst nicht spürt.}.
Es sieht also gar nicht \emph{so} schlecht aus. Nach meinen Berechnungen lässt sich der Gürtel
gute 14 Stunden\footnote{Unter der Annahme, dass sich ein Motor ständig dreht und ein Akkupack mit
\SI{2}{\ampere\hour} zum Einsatz kommt.} betreiben.
Soviel zur Theorie. In der Praxis hatte ich ebenfalls Probleme mit der Stromversorgung, da die Schaltung
unterhalb von \SI{4,9}{\volt} nicht mehr zuverlässig funktioniert.
Dieses Problem löste ich mit einem Spannungswandler und einem Akkupack mit mehr Zellen.
Später bin ich dann noch auf
\href{http://de.wikipedia.org/wiki/Lithium-Ionen-Akkumulator}{Lithium-Ionen-Akkus} umgestiegen.

\begin{figurewrapper}
	\includegraphics[width=0.7\hsize]{files/images/Robin/compass_belt/IMG_8009-rm-bg\imageresize}
	\captionof{figure}{Nachbau des Kompass-Gürtels}
	\label{fig:My_compass_belt}
\end{figurewrapper}

\subsubsection{Erfahrungen}
Erste Tests bestätigten meine Vermutung, dass sich mit einem solchen Gürtel die Orientierung nach
einer gewissen Zeit verbessern könnte. Hierfür sollten dann aber mindestens vier Vibrationsmotoren
zur Verfügung stehen.

Nach längeren Tests ist mir aufgefallen, dass ich immer wusste, wo Norden ist, teils auch unbewusst.
Mit der aktuellen Uhrzeit wusste ich auch immer, wo sich die Sonne befand. Dies ist, sowohl in
Gebäuden interessant, wie auch draußen.\footnote{Langzeiterfahrungen kann ich leider nicht
präsentieren, da der Gürtel erst zwei Wochen vor Abgabe der Dokumentation fertig wurde.}

Nebenbei machte ich die Erfahrung, dass die Personen, die mich auf den Gürtel ansprachen, zuerst an
einen Bombengürtel dachten~\dots

\subsection{Magnet im Finger}
\begin{wrapfigure}{r}{0pt} %% Einbindung steht im Konflikt mit dem Urheberrecht
	\href{\URLQuinnMagnet}{\includegraphics[width=5.6cm]%
		{files/images/Robin/magnet/Quinn_Norton/Magnet_am_Finger-rm-bg}%
	}
	\captionof{figure}[Magnete hängen am Ringfinger von Quinn Norton]%
	{Magnete hängen am Ringfinger von Quinn Norton\footnotemark}%
	\label{fig:Quinn_Norton_magnet}
\end{wrapfigure}
\footnotetext{Hintergrund entfernt und zugeschnitten.
Das Original ist hier zu finden: \url{\URLQuinnMagnet}}

Die US-Journalistin \href{http://quinnnorton.com/}{Quinn Norton} begann 2005, sich für funktionale
Körperveränderung zu interessieren.
Im September ließ sie sich einen kleinen Magneten in die Spitze des rechten
Ringfingers implantieren, da hier sehr viele Nerven vorhanden sind. Der Magnet bewegt sich angeregt
durch elektromagnetische Felder in ihrem Finger. In der Nähe von elektrischen Leitungen nahm sie ein
Vibrieren wahr. Bei Sicherheitsschranken in Geschäften hatte sie sogar Schmerzen, da diese offenbar
mit hohen Leistungen arbeiten. Die Fingerspitze ist so empfindlich, dass sie sogar die Festplatten
Aktivität, genauer das Anlaufen einer Festplatte, spüren konnte.

Für diese Zeit hatte sie einen sechsten Sinn. Nach ein paar Monaten trat allerdings eine Infektion
auf. Die Silikon-Schutzschicht ist offenbar beschädigt worden und der Magnet wurde von ihrem Körper
angegriffen. Ihr Arzt versuchte den zerbröselnden Magneten herauszupullen, aber dies gelang ihm
nicht. Nach diesem Zwischenfall hatte sie ihr Gespür für elektromagnetische Felder verloren. Aber
Magnete konnte sie immer noch mit ihrem Ringfinger halten
(siehe Abb.~\vref{fig:Quinn_Norton_magnet}). 2007 wurde der Magnet dann komplett entfernt. Sie ist
einerseits glücklich, dass der Magnet entfernt wurde, aber andererseits vermisst sie den
sechsten Sinn. Auch wenn ihre Experimente mit dem Magnet im Finger mit viel Schmerzen verbunden
waren, sagte sie schon, dass wenn das Problem mit der Silikon-Schutzschicht gelöst ist, sie
sich wieder einen Magneten implantieren lassen wird.\footcite{23C3:body_hacking,
mindhacks:magnet_removed}

\subsubsection{Magnet an meinem Finger}
\label{sec:Robin:experiments:myMagnet}
Um ihre Erkenntnisse nachzuvollziehen, überlegte ich mir, wie ich einen Magneten an meinem
Ringfinger anbringen konnte, ohne diesen gleich implantieren zu lassen.

Zuerst versuchte ich, verschiedene Magneten an meinem Ringfinger mit Klebeband zu befestigen. Dies
brachte aber keine Ergebnisse, da hierdurch die Durchblutung verschlechtert wird und man deswegen
weniger Gefühl im Finger hat. Außerdem spürt man den Puls zu stark.

\begin{wrapfigure}{r}{0pt}
	\includegraphics[width=6cm]%
		{files/images/Robin/magnet/Magnet_auf_dem_Finger-rm-bg-cut\imageresize}
	\captionof{figure}{Mein Ringfinger mit Magnet (befestigt mit Sekundenkleber)}
	\label{fig:My_magnet}
\end{wrapfigure}

Es war also nötig mir einen Magnet auf den Fingernagel des rechten Ringfingers zu kleben. Ich
benutze einen normalen Alleskleber, den ich zuerst auf den Fingernagel gab und dann einen
quadratischen sehr starken Magneten mit einem Volumen von \SI{125}{\cubic\milli\metre}
und einem Gewicht von \SI{0,9}{\gram} auf den
Fingernagel setzte. Dann fixierte ich den Magnet mit Klebeband. Das Aushärten des Klebers fühlte
sich etwas merkwürdig an. Das Erste, was mir auffiel, war das zusätzliche Gewicht. Ich ließ den
Kleber eine Stunde aushärten allerdings war der Kleber danach immer noch nicht fest. Als ich eine
Schere benutze, um das Klebeband durchzuschneiden, blieb der Magnet am Metall der Schere hängen und
löste sich vom Fingernagel.

Als Nächstes benutzte ich Sekundenkleber. Dies funktionierte deutlich besser
(siehe Abb.~\vref{fig:My_magnet}).
Direkt nach dem
Aufsetzen des Magneten auf den, mit Kleber bedeckten Fingernagel, haftete der Magnet.
Zusätzliches Klebeband war nicht nötig. Im Gegensatz zum Alleskleber spürte ich beim Aushärten vom
Sekundenkleber nichts.\footnote{Auf den Tuben der Flüssigkleber steht \enquote{kann
allergische Reaktionen auslösen} allerdings hatte ich damit keine Probleme. Es gibt
auch Hautkleber. Dieser ist vermutlich besser geeignet.} Auch konnte ich wären dem Aushärten
diesen Text schreiben. Nach circa 40 Minuten war der Kleber dann gut durchgehärtet.

\subsubsection{Tests und Erfahrungen}
Als Erstes ging ich mit diesem \enquote{neuen Sinn} an das Steckdosenkabel meines
Computers.\footnote{Der Computer hat mit TFT-Monitor circa einen Verbrauch von \SI{150}{\watt}.} Ab
\SI{3}{\milli\metre} Distanz zwischen Magnet und Kabel spürte ich ein leichtes Zittern am
Fingernagel. Bei ausgeschalteten Verbrauchern war hingegen kein Zittern wahrnehmbar. Bei
Trafonetzteilen lässt sich sogar ohne das daran eine Last angeschlossen ist ein deutliches Zittern
aus mehr als \SI{1}{\centi\metre} fühlen. Bei Schaltnetzteilen spürt man ohne Last gar nichts. Erst
mit Verbraucher hat man die Möglichkeit wieder das Zittern zu spüren. Dies liegt natürlich an dem
unterschiedlichen technischen Aufbau. Bei dem Zittern handelt es sich um die \SI{50}{\hertz}
Netzfrequenz. Der fließende Strom baut um die Leitungen ein Magnetfeld auf.

Mit einem Magneten am Finger lassen sich also magnetische Felder wahrnehmen. Man kann somit natürlich
auch magnetische Objekte ertasten.

Bei Festplatten habe ich allerdings nicht viel gespürt. Das bleibt wohl denen vorbehalten, die
einen Magnet unter der Haut haben.

An das zusätzliche Gewicht des Magnets gewöhnt man sich recht schnell. Der Magnet fällt mir nur noch
durch die Trägheitskraft auf.

Die geklebte Verbindung zwischen Fingernagel und Manget ist wirklich sehr gut. Der Magnet hält
Metalle und starke Magnete bis zu einem Gewicht von \SI{600}{\gram}. Mehr hält der Magnet nicht. Der
Kleber hält hingegen mindestens \SI{1}{\kilo\gram}.
Man kann kräftig an dem Magnet ziehen und er rührt sich nicht.
Mehr habe ich nicht getestet, da ab diesem
Gewicht Schmerzen auftreten.

Nach zwei Tagen mit dem Magnet auf meinem Finger begann ich zu überlegen, wie ich den Magnet wieder
entfernen konnte. Ich versuchte es mit einem Seitenschneider. Dies funktionierte auch nach ein paar
Minuten. Ich ging zwischen Fingernagel und Magnet und löste so den Magnet. Die Klebereste ließen
sich größtenteils mit einem Teppichmesser und mit einem Fingernagelknipser entfernen.
Kleinere Klebereste lassen sich dann mit einer Feile oder einer
Nass-Schleifmaschine\footnote{Nass-Schleifmaschine wegen der geringen Drehzahl. Eine normale
Schleifmaschine ist nicht empfehlenswert.} abbekommen.

Mit diesem Experiment konnte ich zumindest etwas nachvollziehen, was Quinn Norton mit einem Magnet
in der Ringfingerspitze berichtete. Obwohl der Fingernagel um einiges unempfindlicher sein dürfe als
das Innere der Fingerspitze.
