\section{Blick in die Zukunft}
\label{sec:Robin:future}

In diesem Kapitel beschreibe ich meine Visionen, was in Zukunft alles möglich werden könnte. Der Blick in
die nahe Zukunft basiert zum Teil auch auf heutigen Forschungen oder absehbaren Entwicklungen.

\subsection{Dem Körper Energie entziehen}
Jedes Implantat, das Elektronik Komponenten besitzt, benötigt Energie. Um nicht in regelmäßigen
Zeitabständen durch Nachladen, Auswechseln, Induktion oder Infrarotstrahlung diese Energie
bereitzustellen, wird es meiner Ansicht nach dazu kommen, die Energie direkt dem Körper zu
entziehen. Zum Beispiel über die Körperwärme oder die Energie, die in chemischer Form durch das Blut
transportiert wird. Da die benötigte Energie bei den meisten Implantaten so gering ist, werden wird
dies nicht einmal merken. Damit wäre das Energieproblem für Implantate und Prothesen endlich gelöst.
Diese Technik wird in jedem Implantat zum Einsatz kommen.

\subsection{Hörverbesserung} %% used with \nameref
\label{sec:Robin:future:hearing}
Die heutigen Hörgeräte sind schon erstaunlich gut.
In naher Zukunft wird es möglich werden auch Menschen mit vollständig defektem Innenohrs zu helfen.
Aktuell wird am Hirnstammimplantat geforscht, das direkt an den Nerven ansetzt.
Wir werden bald einen Zeitpunkt erreicht haben, ab dem Hörgeräte nicht nur die Hörleistung eines gesunden
Menschen erreichen, sondern dies noch übertreffen.
Ein solches Hörgerät könnte uns etwa dazu befähigen, ein breiteres Frequenzband
wahrzunehmen. Beispielsweise \href{http://de.wikipedia.org/wiki/Infraschall}{Infraschall}, der
besonders unter Wasser zur Kommunikation geeignet ist\footnote{Blauwale nutzen Infraschall, also
Schall unter \SI{16}{\hertz}, zu diesem Zweck.}. Aber auch
\href{http://de.wikipedia.org/wiki/Ultraschall}{Ultraschall}\footnote{Schall oberhalb von
\SI{16}{\kilo\hertz}.} bietet einiges Potenzial. So könnte ich mir vorstellen, dass sich mit einem
Ultraschallsender und mit der Schalltrichterwirkung der Ohrmuschel die Umgebung genauer abtasten
lässt. Sowohl Infraschall als auch Ultraschall können von unserem jetzigen Gehör quasi nicht
wahrgenommen werden. Weiterhin könnte ich mir vorstellen, dass ein solches Hörgerät eine
Aufzeichnungsfunktion hat, sodass man sich alles noch einmal anhören kann. Man könnte auch
verschiedene Filter auf das Signal anwenden, Rauschen herausfiltern und laute Töne automatisch leiser
zu machen. Die digitale Signalverarbeitung bietet heute schon einige Möglichkeiten.
Es ließen sich mit einem Hörgerät auch die meisten Einschränkungen aufheben, die durch die
\href{http://de.wikipedia.org/wiki/Psychoakustik}{Psychoakustik} gefunden wurden. Zum Beispiel hören
wir leise Geräusche nicht, wenn sie einige Millisekunden nach einem lauten Geräusche auftreten.

Wenn man diesen Gedanken weiterdenkt, kommt man zu dem Schluss, dass wir noch in diesem Jahrhundert
Hörgeräte haben werden, die als Handy Freisprecheinrichtung funktionieren und alle
\phantomsection\label{sec:Robin:future:hearing:Babel_Fish}%
Sprachen in die
eigene Muttersprache übersetzen\footnote{Der
\href{http://de.wikipedia.org/wiki/Babelfisch}{Babelfisch} aus dem Buch \citetitle
{Per_Anhalter_durch_die_Galaxis} wird Realität.} können. Alles außer der Übersetzungsfunktion lässt
sich meiner Ansicht nach bereits
heute mit überschaubarem Entwicklungsaufwand umsetzen. Ein Problem was für eine optimale Benutzung
noch gelöst werden muss ist aber eine praktikable Steuerung des Geräts.
\fxnote{ref>Gehirnschnittstelle}

\subsection{Sehverbesserung}
Mit dem noch experimentellen Retinaimplantat ist es bereits heute möglich, einfache Bilder direkt
an den Sehnerv zu übertragen. Besonders im subretinalen Implantat sehe ich ein großes Potenzial. Es
wird zu einer erheblichen Verbesserung des Bildes kommen, sodass mit einem Sehimplantat dann auch in
Farbe gesehen werden kann. Es wird also noch eine Weile dauern, bis ein künstliches Auge unser
biologisches Auge übertreffen wird. Aber wenn wir an diesem Punkt sind, stehen uns noch mehr
Möglichkeiten offen, diesen Sinn zu erweitern, als beim Hörsinn. Die Entwicklung wird also nicht
Enden.

Ich könnte mir vorstellen, dass sich der Sehsinn so um eine Einblendung verschiedener Informationen
erweitern lässt. Diese Entwicklung ist heute schon durch den Bereich erweiterter Realität absehbar.
Dabei geht es darum, die Wahrnehmung der (meist visuellen) Realität computergestützt zu erweitern. So
gibt es schon heute Systeme, die Objekte erkennen und über eine Videobrille zusätzliche Informationen
zu diesen mit ins aktuelle Bild einblenden. Die Möglichkeiten, die damit geboten werden, wurden vor
einigen Monaten durch die \enquote{Google-Brille} wieder ins Bewusstsein
gerückt. Mit dieser sollen sich Funktionen wie Navigation, Videos und Terminerinnerungen über eine
Brille direkt ins Blickfeld einblenden lassen.\footcite{Heise:TR:Project_Glass}
Eine solche Brille\footnote{oder später auch als
Kontaktlinse} könnte in einigen Jahren auf den Markt kommen und den Handys Konkurrenz machen. Dies
hätte den Vorteil, dass gewisse Informationen ständig im Blickfeld sind.\footnote{Beispielsweise
bekannt aus den Terminator Filmen und Serien.} Man könnte Telemetriedaten vom eigenen
Körper\footnote{die über entsprechende Implantate erfasst werden}
hierüber Anzeigen. Beispielweise Blutzuckerspiegel, Herzschlag, freigesetzte Hormone, vorhandene
Energiereserven, Temperaturen, Sauerstoffgehalt im Blut,
\href{http://de.wikipedia.org/wiki/Elektrokardiogramm}{Elektrokardiogramm} und so weiter.
Der Hauptanwendungsfall wird aber dem eines
Smartphones\footnote{\href{http://de.wikipedia.org/wiki/Smartphone}{Wikipedia}: Ein Mobiltelefon, das
mehr Computerfunktionalität und -konnektivität als ein herkömmliches fortschrittliches Mobiltelefon
zur Verfügung stellt.} nahe kommen. Es wird aber auch viele Programme geben, die die Umgebung
analysieren und Wichtiges hervorheben und zusätzliche Informationen zur Umgebung einblenden. Als
Analogie zum Hörgerät, \hyperref[sec:Robin:future:hearing:Babel_Fish]{das jede Sprache in die
Muttersprache übersetzt}, könnte es Programme geben die Texte übersetzt einblenden oder auch falsch
geschriebene Wörter signalisieren aber auch welche die Bedienungsanleitungen und Arbeitsabläufe
anzeigen.

Weitergehen könnte es dann, indem diese Technik soweit miniaturisiert wird, um ins Auge zu
passen.\footnote{Eventuell direkt in Verbindung mit einer \nameref{sec:Robin:future:hearing} um auch
eine Audioschnittstelle zum Gehirn zu haben.
} Dadurch wird die Brille überflüssig und alles, was man
sieht, wird über einen Computer vorverarbeitet. Die Möglichkeiten, die sich daraus ergeben, sind sehr
vielseitig, da jetzt nicht nur Einblendungen machbar sind, sondern direkt auf das wahrnehmbare Bild
Einfluss genommen werden kann. Es könnte als ebenfalls\footnote{Wie auch schon bei der
\nameref{sec:Robin:future:hearing} beschrieben.} eine Aufzeichnungs- und Abspielfunktion möglich
werden. Auch Zeitlupe und das Einfrieren des Bildes sind durchaus denkbar. Weiterhin bietet eine
Videobearbeitung zum Beispiel die Möglichkeit Bilder zu entzerren oder auf optische Täuschung
hinzuweisen, die unser Gehirn eventuell nicht erkennen würde. Es wird sicher auch viele Programme
geben, die die Wirklichkeit verändert darstellen. Beispielsweise eine Darstellung der Realität als
\href{http://www.tony5m17h.net/MatrixCode.gif}{Matrixcode}
aus dem Film \href{http://www.imdb.de/title/tt0234215/}{Matrix Reloaded (2003)}.

\begin{figurewrapper} %% Einbindung steht im Konflikt mit dem Urheberrecht
	\includegraphics[width=0.7\hsize]{files/images/Robin/Matrixcode\imageresize}
	\captionof{figure}{Matrixcode aus Matrix Reloaded (2003)}
	\label{fig:Matrixcode}
\end{figurewrapper}

\begin{comment}
\begin{wrapfigure}{r}{0pt}
	\href%
	{http://de.wikipedia.org/w/index.php?title=Datei:Infrared_dog.jpg&filetimestamp=20050107180457}{%
		\includegraphics[width=5cm]{files/images/Robin/Goetz-eiserne-hand1}%
	}
	\captionof{figure}{Bild eines Hundes im Infrarotspektrum}%
	\label{fig:eiserneHand}
\end{wrapfigure}
\end{comment}

Ein weiterer denkbarer Schritt ist die Verbesserung der Kamera um ein breites Frequenzband aus dem
elektromagnetischen Spektrum zu sehen. Interessant ist hier beispielsweise die Infrarotstrahlung
beziehungsweise Wärmestrahlung. Man könnte wie eine Wärmebildkamera die Umgebung sehen und
Hitzequellen erkennen. Außerdem kann man die Kamera hinsichtlich der Auflösung optimieren, um dann
darüber eine Vergrößerung zu ermöglichen.%\footnote{Der Monokel der von Uhrmachern getragen wird,
%wandert direkt ins Auge.}

