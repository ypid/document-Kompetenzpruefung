\section{Blick in die Zukunft}
\label{sec:Robin:future}

In diesem Kapitel beschreibe ich meine Visionen, was in Zukunft alles möglich werden könnte. Der Blick in
die nahe Zukunft basiert zum Teil auch auf heutigen Forschungen oder absehbaren Entwicklungen.

\subsection{Hörverbesserung}
Die heutigen Hörgeräte sind schon erstaunlich gut.
In naher Zukunft wird es möglich werden auch Menschen mit vollständig defektem Innenohrs zu helfen.
Aktuell wird am Hirnstammimplantat geforscht, das direkt an den Nerven ansetzt.
Wir werden bald einen Zeitpunkt erreicht haben, ab dem Hörgeräte nicht nur die Hörleistung eines gesunden
Menschen erreichen, sondern dies noch übertreffen.
Ein solches Hörgerät könnte uns etwa dazu befähigen, ein breiteres Frequenzband
wahrzunehmen. Beispielsweise \href{http://de.wikipedia.org/wiki/Infraschall}{Infraschall}, der
besonders unter Wasser zur Kommunikation geeignet ist\footnote{Blauwale nutzen Infraschall, also
Schall unter \SI{16}{\hertz}, zu diesem Zweck.}. Aber auch
\href{http://de.wikipedia.org/wiki/Ultraschall}{Ultraschall}\footnote{Schall oberhalb von
\SI{16}{\kilo\hertz}.} bietet einiges Potenzial. So könnte ich mir vorstellen, dass sich mit einem
Ultraschallsender und mit der Schalltrichterwirkung der Ohrmuschel die Umgebung genauer abtasten
lässt. Sowohl Infraschall als auch Ultraschall können von unserem jetzigen Gehör quasi nicht
wahrgenommen werden. Weiterhin könnte ich mir vorstellen, dass ein solches Hörgerät eine
Aufzeichnungsfunktion hat, sodass man sich alles noch einmal anhören kann. Man könnte auch
verschiedene Filter auf das Signal anwenden, Rauschen herausfiltern und laute Töne automatisch leiser
zu machen. Die digitale Signalverarbeitung bietet heute schon einige Möglichkeiten.
Es ließen sich mit einem Hörgerät auch die meisten Einschränkungen aufheben, die durch die
\href{http://de.wikipedia.org/wiki/Psychoakustik}{Psychoakustik} gefunden wurden. Zum Beispiel hören
wir leise Geräusche nicht, wenn sie einige Millisekunden nach einem lauten Geräusche auftreten.

Wenn man diesen Gedanken weiterdenkt, kommt man zu dem Schluss, dass wir noch in diesem Jahrhundert
Hörgeräte haben werden, die als Handy Freisprecheinrichtung funktionieren und alle Sprachen in die
eigene Muttersprache übersetzen\footnote{Der
\href{http://de.wikipedia.org/wiki/Babelfisch}{Babelfisch} aus dem Buch \citetitle
{Per_Anhalter_durch_die_Galaxis} wird Realität.} können. Die Energie werden diese Geräte direkt dem
Körper entziehen zum Beispiel über die Körperwärme oder die Energie, die in chemischer Form durch das
Blut transportiert wird. Alles außer der Übersetzungsfunktion lässt sich meiner Ansicht nach bereits
heute mit überschaubarem Entwicklungsaufwand umsetzen. Ein Problem was für eine optimale Benutzung
noch gelöst werden muss ist aber eine praktikable Steuerung des Geräts.
\fxnote{ref>Gehirnschnittstelle}
