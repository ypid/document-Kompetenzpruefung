\section{Blick in die Zukunft}
\label{sec:Robin:future}

In diesem Kapitel beschreibe ich meine Visionen, was in Zukunft alles möglich werden könnte. Der Blick in
die nahe Zukunft basiert zum Teil auch auf heutigen Forschungen oder absehbaren Entwicklungen.

\subsection{Prothese}
Aktuell wird an Prothesen geforscht, die nicht nur Bewegungen ausführen, die direkt von den Nerven
abgegriffen werden, sondern auch Rückmeldungen liefern. Also hauptsächlich \mbox{Druck-,}
\mbox{Temperatur-,} und Schmerzempfinden. Die Schnittstelle zwischen zentralem Nervensystem und
Prothese könnte in Zukunft Glasfasern darstellen. Die Schnittstelle soll mit Messfühlern
funktionieren, die ihre Form\footnote{Dadurch wird vermutlich die Wellenlänge des Lichts geändert,
dass durch die Glasfasern geführt wird.} abhängig vom elektrischen Feld ändern. Diese Signale werden
von einem Mikroprozessor ausgewertet und in Steuersignale für eine Prothese umgewandelt.
Informationen von der Prothese werden über Infrarotstrahlung über Glasfasern an die Nerven geleitet.
Diese Entwicklung befindet sich allerdings noch in den
Anfängen.\footcite{Spektrum:Weg_zu_intelligenten_Prothesen}

Durch diese Entwicklung wird es bald Prothesen geben, die den biologischen Vorbildern ebenbürtig
sind. In der Prothetik steckt aber auch das Potenzial bessere Prothesen zu entwickeln als das
Original. Vorteile die sich daraus ergeben sind beispielsweise die hohe Kraft, die eine solche
Prothese aufbringen kann.

Im Bereich der Armprothetik sehe ich außerdem besonders den Vorteil der Spezialisierung.\footnote{Der
menschliche Körper ist sehr anpassungsfähig. Das heißt aber auch, dass er für viele Anwendungen nicht
optimal ist.} Ich könnte mir vorstellen, dass sich Armprothesen austauschen lassen und so immer ein
optimales Werkzeug, das an die aktuelle Aufgabe angepasst ist, zur Verfügung steht.

Bei Beinprothesen wird mehr der Leistungsaspekt eine Rolle spielen.


\subsection{Dem Körper Energie entziehen}
Jedes Implantat, das elektronische Komponenten besitzt, benötigt Energie. Um nicht in regelmäßigen
Zeitabständen durch Nachladen (beispielsweise über Induktion oder
Infrarotstrahlung) oder Auswechseln diese Energie bereitzustellen zu müssen, wird
es meiner Ansicht nach dazu kommen, die Energie direkt dem Körper zu entziehen.
Zum Beispiel über die Körperwärme\footnote{Es wird ein Temperaturunterschied%
	benötigt, dann lässt sich zum Beispiel mit einem
\href{http://de.wikipedia.org/wiki/Peltier-Element}{Peltierelement} Strom erzeugen.
Das Peltierelement sollte also optimalerweise auf einem Muskel in der Nähe der Haut implantiert
werden.},
Bewegungsenergie\footnote{Ähnlich einem Automatik-Uhrwerk, das mit Schwungmasse funktioniert.
Die zweite Möglichkeit, die ich sehe, ist die Nutzung eines
\href{http://de.wikipedia.org/wiki/Piezoelement}{Piezoelements} beispielsweise unter der
Fußsohle um durch die Druckänderung elektrische Energie zu generieren.}%, Solarzelle
%% Dem Körper Energie entziehen
oder später auch die Energie, die in chemischer Form
durch das Blut
transportiert wird. Da die benötigte Energie bei den meisten Implantaten so gering ist, werden wir
dies nicht einmal merken. Damit wäre das Energieproblem für Implantate (eventuell auch für Prothesen)
endlich gelöst.\footnote{Diese Zukunftsvision kommt bereits in greifbare Nähe:
\cite{Heise:Hummer_erzeugt_Strom}}
%Diese Technik wird für jedes Implantat zum Einsatz kommen.

\subsection{Licht Implantat}
%\begin{comment}
\begin{wrapfigure}{r}{0pt}
	\href{http://de.wikipedia.org/w/index.php?title=Datei:Lampyris_noctiluca.jpg&filetimestamp=20050618231628}{%
		\includegraphics[width=4.5cm]{files/images/Robin/Lampyris_noctiluca-cut-rot}%
	}
	\captionof{figure}[Ein weibliches Glühwürmchen]%
		{Ein weibliches Glühwürmchen\footnotemark}%
	\label{fig:Firefly}
\end{wrapfigure}
\footnotetext{Es handelt sich, um genau zu sein, um ein Weibchen des Großen Leuchtkäfers (Lampyris
noctiluca).
Bild zugeschnitten und gedreht.
Das Original, wie auch meine Abwandlung, stehen unter
\href{http://creativecommons.org/licenses/by-sa/2.0/de/deed.de}{Creative Commons BY-SA 2.0}.
Das Bild stammt vom Benutzer \href{http://de.wikipedia.org/wiki/Benutzer:Wofl}{Wofl}.}
%\end{comment}

Ein leicht umsetzbares Implantat ist meiner Ansicht nach die Nachrüstung eines
\href{http://de.wikipedia.org/wiki/Leuchtorgane}{Leuchtorgans}, wie von
\href{http://de.wikipedia.org/wiki/Leuchtkäfer}{Glühwürmchen} bekannt, zum Beispiel an der Hand. Ein
solches Leuchtorgan, wie es bei Tieren\footnote{Hauptsächlich bei Tiefsee Meerestieren.}
vorkommt, hat einen
extrem hohen Wirkungsgrad von bis zu 95\Prozent. Die Energie könnte direkt dem Blutkreislauf entzogen
werden. So könnte ich mir, in den nächsten 15 Jahren, eine Lampe
vorstellen, die sich in einen Finger implantieren lässt und beispielsweise durch eine ungewöhnliche
Bewegung des Fingers an- und ausgeschaltet wird.\footnote{Interessanterweise habe ich zu dieser Idee
keine Quellen im Internet gefunden, die darauf näher eingehen. Die Idee scheint als noch recht
unbekannt zu sein.}

\subsection{Hörverbesserung} %% used with \nameref
\label{sec:Robin:future:hearing}
Die heutigen Hörgeräte sind schon erstaunlich gut.
In naher Zukunft wird es möglich werden auch Menschen mit vollständig defektem Innenohr zu helfen.
Aktuell wird am Hirnstammimplantat geforscht, das direkt an den Nerven ansetzt.
Wir werden bald einen Zeitpunkt erreicht haben, ab dem Hörgeräte nicht nur die Hörleistung eines gesunden
Menschen erreichen, sondern diese noch übertreffen.
Ein solches Hörgerät könnte uns etwa dazu befähigen, ein breiteres Frequenzband
wahrzunehmen. Beispielsweise \href{http://de.wikipedia.org/wiki/Infraschall}{Infraschall}, der
besonders unter Wasser zur Kommunikation geeignet ist\footnote{Blauwale nutzen Infraschall, also
Schall unter \SI{16}{\hertz}, zu diesem Zweck.}. Aber auch
\href{http://de.wikipedia.org/wiki/Ultraschall}{Ultraschall}\footnote{Schall oberhalb von
\SI{16}{\kilo\hertz}.} bietet einiges Potenzial. So könnte ich mir vorstellen, dass sich mit einem
Ultraschallsender und mit der Schalltrichterwirkung der Ohrmuschel die Umgebung genauer abtasten
lässt. Sowohl Infraschall als auch Ultraschall können von unserem jetzigen Gehör quasi nicht
wahrgenommen werden. Weiterhin könnte ich mir vorstellen, dass ein solches Hörgerät eine
Aufzeichnungsfunktion hat, sodass man sich alles noch einmal anhören kann. Man könnte auch
verschiedene Filter auf das Signal anwenden, Rauschen herausfiltern und laute Töne automatisch leiser
machen. Die digitale Signalverarbeitung bietet heute schon einige Möglichkeiten.
Es ließen sich mit einem Hörgerät auch die meisten Einschränkungen aufheben, die durch die
\href{http://de.wikipedia.org/wiki/Psychoakustik}{Psychoakustik} gefunden wurden. Zum Beispiel hören
wir leise Geräusche nicht, wenn sie einige Millisekunden nach einem lauten Geräusche auftreten.

Wenn man diesen Gedanken weiterdenkt, kommt man zu dem Schluss, dass wir noch in diesem Jahrhundert
Hörgeräte haben werden, die als Handy Freisprecheinrichtung fungieren und alle
\phantomsection\label{sec:Robin:future:hearing:Babel_Fish}%
Sprachen in die
eigene Muttersprache übersetzen\footnote{Der
\href{http://de.wikipedia.org/wiki/Babelfisch}{Babelfisch} aus dem Buch \citetitle
{Per_Anhalter_durch_die_Galaxis} wird Realität.} können. Alles außer der Übersetzungsfunktion lässt
sich meiner Ansicht nach bereits
heute mit überschaubarem Entwicklungsaufwand umsetzen. Ein Problem was für eine optimale Benutzung
noch gelöst werden muss ist aber eine praktikable Steuerung des Geräts
(\siehe{sec:Robin:future:BCI}).

\subsection{Sehverbesserung}
Mit dem noch experimentellen Retina-Implantat ist es bereits heute möglich, einfache Bilder direkt
an den Sehnerv zu übertragen. Besonders im subretinalen Implantat sehe ich ein großes Potenzial. Es
wird zu einer erheblichen Verbesserung des Bildes kommen, sodass mit einem Sehimplantat dann auch in
Farbe gesehen werden kann. Es wird noch eine Weile dauern, bis ein künstliches Auge unser
biologisches Auge übertreffen wird. Aber wenn wir an diesem Punkt sind, stehen uns noch mehr
Möglichkeiten offen, diesen Sinn zu erweitern, als beim
\hyperref[sec:Robin:future:hearing]{Hörsinn}. Die Entwicklung wird also nicht
Enden.

Ich könnte mir vorstellen, dass sich der Sehsinn so um eine Einblendung verschiedener Informationen
erweitern lässt. Diese Entwicklung ist heute schon durch den Bereich der
\href{http://de.wikipedia.org/wiki/Erweiterte_Realität}{erweiterten Realität}
absehbar.
Dabei geht es darum, die Wahrnehmung der (meist visuellen) Realität computergestützt zu erweitern. So
gibt es schon heute Systeme, die Objekte erkennen und über eine Videobrille zusätzliche Informationen
zu diesen mit ins aktuelle Bild einblenden. Die Möglichkeiten, die damit geboten werden, wurden vor
einigen Monaten durch die \enquote{Google-Brille} wieder ins Bewusstsein
gerückt. Mit dieser sollen sich Funktionen wie Navigation, Videos und Terminerinnerungen über eine
Brille direkt ins Blickfeld einblenden lassen.\footcite{Heise:TR:Project_Glass}
Eine solche Brille\footnote{oder später auch als
Kontaktlinse} könnte in einigen Jahren auf den Markt kommen und den Handys Konkurrenz machen. Dies
hätte den Vorteil, dass gewisse Informationen ständig im Blickfeld sind.\footnote{Beispielsweise
bekannt aus den Terminator Filmen und Serien.} Man könnte Telemetriedaten vom eigenen
Körper\footnote{die über entsprechende Implantate erfasst werden}
hierüber Anzeigen. Beispielweise Blutzuckerspiegel, Herzschlag, freigesetzte Hormone, vorhandene
Energiereserven, Temperaturen, Sauerstoffgehalt im Blut,
\href{http://de.wikipedia.org/wiki/Elektrokardiogramm}{Elektrokardiogramm}\footnote{%
Ein Diagramm jeder elektrischen Erregung, die im
Normalfall vom Sinusknoten ausgeht. Durch diese elektrische Erregung wird der Herzschlag
kontrolliert. Ein \href{http://de.wikipedia.org/wiki/Elektrokardiogramm}{EKG} ist in
\hyperref[sec:Robin:topical:Pacemaker]{heutigen Herzschrittmachern}
und implantierbaren Kardioverter-Defibrillatoren
bereits integriert.} und so weiter.
Der Hauptanwendungsfall wird aber dem eines
Smartphones\footnote{\href{http://de.wikipedia.org/wiki/Smartphone}{Wikipedia}: Ein Mobiltelefon, das
mehr Computerfunktionalität und -konnektivität als ein herkömmliches fortschrittliches Mobiltelefon
zur Verfügung stellt.} nahe kommen. Es wird aber auch viele Programme geben, die die Umgebung
analysieren und Wichtiges hervorheben und zusätzliche Informationen zur Umgebung einblenden. Als
Analogie zum Hörgerät, \hyperref[sec:Robin:future:hearing:Babel_Fish]{das jede Sprache in die
Muttersprache übersetzt}, könnte es Programme geben die Texte übersetzt einblenden oder auch falsch
geschriebene Wörter signalisieren aber auch Programme die Bedienungsanleitungen und Arbeitsabläufe
anzeigen.

Weitergehen könnte es dann, indem diese Technik soweit miniaturisiert wird, um ins Auge zu
passen.\footnote{Eventuell direkt in Verbindung mit einer \nameref{sec:Robin:future:hearing} um auch
eine Audioschnittstelle zum Gehirn zu haben.
} Dadurch wird die Brille überflüssig und alles, was man
sieht, wird über einen Computer vorverarbeitet. Die Möglichkeiten, die sich daraus ergeben, sind sehr
vielseitig, da jetzt nicht nur Einblendungen machbar sind, sondern direkt auf das wahrnehmbare Bild
Einfluss genommen werden kann.

%\begin{comment}
\begin{wrapfigure}{r}{0pt}
	\href%
	{http://de.wikipedia.org/w/index.php?title=Datei:Infrared_dog.jpg&filetimestamp=20050107180457}{%
		\includegraphics[width=5cm]{files/images/Robin/Infrared_dog}%
	}
	\captionof{figure}{Bild eines Hundes im Infrarotspektrum}%
	\label{fig:Infrared_dog}
\end{wrapfigure}
%\end{comment}

Es könnte als ebenfalls\footnote{Wie auch schon bei der
\nameref{sec:Robin:future:hearing} beschrieben.} eine Aufzeichnungs- und Abspielfunktion möglich
werden. Auch Zeitlupe und das Einfrieren des Bildes sind durchaus denkbar. Weiterhin bietet eine
Videobearbeitung zum Beispiel die Möglichkeit Bilder zu entzerren oder auf optische Täuschungen
hinzuweisen, die unser Gehirn eventuell nicht erkennen würde. Es wird sicher auch viele Programme
geben, die die Wirklichkeit verändert darstellen. Beispielsweise eine Darstellung der Realität als
\href{http://www.tony5m17h.net/MatrixCode.gif}{Matrixcode}
aus dem Film \href{http://www.imdb.de/title/tt0133093/}{Matrix (1999)} (siehe
Abb.~\vref{fig:Matrixcode}).

\begin{figurewrapper} %% Einbindung steht im Konflikt mit dem Urheberrecht
	\includegraphics[width=0.7\hsize]{files/images/Robin/Matrixcode\imageresize}
	\captionof{figure}{Matrixcode aus Matrix (1999)}
	\label{fig:Matrixcode}
\end{figurewrapper}

Ein weiterer denkbarer Schritt ist die Verbesserung der Kamera um ein breiteres Frequenzband aus dem
elektromagnetischen Spektrum zu sehen. Interessant ist hier beispielsweise die Infrarotstrahlung
beziehungsweise Wärmestrahlung. Man könnte wie eine Wärmebildkamera die Umgebung sehen und
Hitzequellen erkennen. Außerdem kann man die Kamera hinsichtlich der Auflösung optimieren, um dann
darüber eine Vergrößerung zu ermöglichen.%\footnote{Der Monokel der von Uhrmachern getragen wird,
%wandert direkt ins Auge.}

\subsection{Gehirn-Computer-Schnittstelle}
\label{sec:Robin:future:BCI}
Das für mich interessanteste Thema in diesem Kapitel ist die
\href{http://de.wikipedia.org/wiki/Brain-Computer-Interface}{Gehirn-Computer-Schnittstelle}.
Damit wird es möglich werden, Informationen beziehungsweise Gedanken direkt mit dem Computer oder
anderen Menschen auszutauschen.

Praktische Anwendungsfälle sind beispielsweise stark körperlich behinderte Menschen, die weder
Sprechen noch ihre Hände zum Schreiben benutzen können. Diesen Menschen wird mit einer
Gehirn-Computer-Schnittstelle die Möglichkeit eröffnet, wieder mit der Außenwelt zu kommunizieren.
Daran wird derzeit intensiv geforscht.

Die Gehirn-Computer-Schnittstelle bildet aber auch die Grundlage für viele weitere Anwendungen.
Beispielsweise die direkte Steuerung von Maschinen, Implantaten und Prothesen.

Durch die Gehirn-Computer-Schnittstelle werden aber auch Gedankenexperimente wie
\href{http://de.wikipedia.org/wiki/Gehirn_im_Tank}{Gehirn im Tank} durchführbar. Dabei geht es darum,
dass ein Wissenschaftler ein Gehirn an einen Computer anschließt und den Computer einen menschlichen
Körper und die dazugehörige Umwelt simulieren lässt. Für das Gehirn sieht es also so aus, also ob es
einen realen Körper kontrolliert. Mit einer Gehirn-Computer-Schnittstelle wird es meiner Ansicht nach
möglich werden, sich in simulierte Welten einzuklinken und dabei den physikalischen Körper zu
vergessen. Diese Idee wurde zum Beispiel durch den recht erfolgreichen Science-Fiction-Film Matrix
(1999) eindrucksvoll demontiert.

An der Stelle könnte man endlos weiter philosophieren. Aber um es kurz zu machen, die
Gehirn-Computer-Schnittstelle bietet nahezu unbegrenzte Möglichkeiten.

\subsection{Probleme und Risiken}
\label{sec:Robin:future:problems}
Ob man sich eines der beschriebenen erweiternden Technologien implantieren lassen möchte, hängt
natürlich von der Umsetzung ab. Beispielsweise inwieweit man sich auf die Sicherheit eines solchen
Gerätes verlassen kann. Da
\href{http://de.wikipedia.org/wiki/Schadprogramm}{Schadprogramme}
auch problemlos auf frei programmierbaren Implantaten
ausführbar sind. Dies könnte passieren, wenn Schwachstellen im System gefunden werden. Dieses Problem
wird umso schlimmer je wichtiger das Implantat zum Überleben ist. Es gibt schon heute Probleme mit
der Sicherheit von Implantaten.\footcite{Heise:Pacemaker_Hacker, MIT:Protect_implants_from_attack}
Computerspezialisten der Universität von Washington und Massachusetts ist es experimentell gelungen,
Herzschrittmacher und implantierbare Defibrillatoren über dessen Funkschnittstelle zu kontrollieren.
Darüber wird es im schlimmsten Falle möglich eine Person mithilfe der lebensnotwendigen Implantate zu
töten. Um allerdings eine Kommunikation zum Implantat aufzubauen, muss der Sender sehr nah am
Implantat sein.
Die unautorisierte Manipulation solcher Geräte über die Funkschnittstelle war eigentlich voraussehbar
und hätte verhindert werden müssen.

Weiterhin können ethische Gründe eine Rolle bei der Entscheidung spielen. Es besteht die
Möglichkeit, dass durch die Optimierungen einzelner Menschen eine Zweiklassengesellschaft noch
verstärkt wird.

Dabei stellt sich dann auch die Frage, \emph{was noch fair ist?} Was würde beispielsweise passieren,
wenn in
naher Zukunft eine Hörverbesserung, mit allen \hyperref[sec:Robin:future:hearing]{von mir
beschriebenen Funktionen}, auf den Markt kommt für einen Preis von vielleicht \EUR{20000}. Es würde
dazu führen, dass die jetzt schon gut gestellten Menschen noch Mächtiger werden. Um dies zu
verhindern, müsste jeder zum Beispiel mit erreichen der Volljährigkeit entscheiden können, ob er sich
diese Verbesserung kostenlos implantieren lassen möchte. Also eine Erweiterung des momentan noch
Diskutierten bedingungslosen Grundeinkommens.

\begin{comment}
Eine weitere Frage ist, bis wann wir noch als Menschen zählen?
„Ist man noch ein Mensch, wenn man den Teil im Gehirn abgeschaltet hat, der für Schuldgefühle
zuständig ist?“\footcite{23C3:body_hacking}
\end{comment}

Es könnte auch zu gezielter Manipulation der Implantatträger kommen. Da die Implantate, die ich
beschrieben habe, ja zwischen Sinnesorganen und Gehirn sitzen, könnte der Träger vollständig
manipuliert werden. Um dies zu verhindern, muss meiner Ansicht nach die Entwicklung solcher
Implantate komplett offen ablaufen.

Gerechtigkeit ist damit vermutlich immer noch nicht gewährleistet. Ich könnte mir zum Beispiel
vorstellen, dass es verschiedene Generationen der Implantate geben wird. Dadurch würden Menschen mit
neueren Implantaten bevorzugt werden.

Auch im Bereich der Nanotechnologie gibt es noch einige ungeklärte Fragen. Beispielsweise ob diese
neue Technologie überhaupt angewendet werden darf, da die Risiken (noch) unabschätzbar sind:
\enquote{[\dots] [Es] ließen sich (auch) Nanobots bauen, die trillionenfach in der Atmosphäre
herumschwirren, bestimmte Personen an ihrem genetischen Muster erkennen und töten
können.} (Zitat von Andreas Eschbach: \cite[13]{Heise:Telepolis:Mensch:Unsterblichkeit})

Es wird bereits deutlich, dass ich diese Frage nicht beantworten kann. Fakt ist aber, dass für solche
Probleme eine Lösung gefunden werden muss. Da sich die technische Entwicklung nur schwer aufhalten
lässt. Wir werden meiner Ansicht nach mit den Verbesserungen, die ich mir in diesem Kapitel
vorgestellt habe, in Zukunft konfrontiert werden.

Und werden uns somit immer mehr zum kybernetischen
Organismus\footnote{\href{http://de.wikipedia.org/wiki/Cyborg}{Wikipedia}: Mischwesen aus lebendigem
Organismus und Maschine. Die gängige Abkürzung ist Cyborg.} weiterentwickeln.
Das wird aber nicht das Ende sein. Ich halte es für wahrscheinlich, dass \enquote{wir} in ferner
Zukunft \enquote{nur noch} als ein kollektives Bewusstsein
existieren werden.\footcite{Heise:Telepolis:Mensch:globales_Gehirn}
