\section{Jetziger stand der Technik}
\label{sec:Robin:topical}

In diesem Kapitel gebe ich einen Überblick über die heute eingesetzten Verfahren in der Medizin um
verlohrengegengene Funktionelle wiederherzustellen oder neue zu ergänzen\fxwarning{Neue Funktionen
in diesem Kapitel vorhanden?}.

\subsection{Hörgeräte}
Heute sind zwei Formen von Hörgeräten üblich. Die Hinter-dem-Ohr- und die Im-Ohr-Geräte. Für
bestimmte Fälle werden Teile des Hörgeräts auch direkt implantiert, um beschädigte Teile im Ohr zu
ersetzen.

Bei den Hinter-dem-Ohr-Hörgeräten sitzt die komplette Technik hinter der Ohrmuschel. Die verstärkten
Töne werden über einen Schallschlauch und ein individuelles Ohrpassstück ins Ohr des Trägers
weitergeleitet. Diese Geräte sind am vielseitigsten einsetzbar, da der Teil hinter dem Ohr
ausreichend Platz für die Technik zur Verfügung stellt. So können mit diesen Hörgeräten hohe
Verstärkungsleistungen erreicht werde und viele Audiotransformationen angewendet werden.

Die Im-Ohr-Geräte verschwinden im Gehörgang. Somit muss die komplette Technik in einer individuellen
Hohlschale Platz finden. Der Vorteil ist, dass die anatomischen Gegebenheiten des Außenohres genutzt
werden können und so eine bessere Positionsbestimmung der Tonquelle möglich wird. Allerdings lässt
sich ein solches Gerät erst ab einer ausreichenden Größe des Gehörgangs realisieren. Probleme sind
hier fehlender Druckausgleich und Rückkopplungseffekte bei hohen Verstärkungsleistungen. Zudem wird
die Ohrenschmelzbildung durch die geringe Belüftung begünstigt was zu mehr Ausfällen der Geräte
führt.

\subsection{Sehimplantate}
Die einfachste Form eines Augenimplantats ist, das heute immer noch Eingesetze, Glasauge. Allerdings
ist dieser chirurgischen Eingriff heute so perfektioniert worden, dass ein Unterschied zu einem
richtigen Auge erst bei näherer Betrachtung auffällt. Die natürliche Bewegung des Augapfels wird
ebenfalls wieder hergestellt, indem das Glasauge mit den Muskeln in der Augenhöhle verbunden wird.
Allerdings muss ein Glasauge jedes Jahr gegen ein neues ausgetauscht werden, da sich die glatte
Oberfläche des Glases oder Kunststoffes durch Schmutz und Tränenflüssigkeit abnutzt und rauer
wird. Die raue Oberfläche würde zu Beschädigung der Augenhöhle führen.
Die Kosten für ein Glasauge werden von den Krankenkassen in Deutschland fast vollständig übernommen.

Mit einem Glasauge ist natürlich kein sehen möglich. Um eine wirkliche Hilfe zu bieten, wird derzeit
am sogenannten Retinaimplantat geforscht. Dieses befindet sich noch im experimentellen Stadium.
Es setzt direkt am
Sehnerv an und ist somit nicht auf eine funktionierende Netzhaut (Retina) angewiesen. Ein
Retinaimplantat wandelt das Bild der Umgebung in elektrische Signale um leitet es an einen digitalen
Signalprozessor weiter. Dieser reizt über ein Stimulations-Elektroden-Array den Sehnerv. Die
Energieversorgung erfolgt in der Regel über Infrarotstrahlung oder Induktion von einer Brille aus.
Aktuell ist mit diesem Implantat eine Unterscheidung zwischen Licht und Schatten möglich. Es gibt
momentan zwei Systeme im experimentellen Stadium. Das subretinale Implantat und das epiretinale
Implantat. Der wesentliche Unterschied ist die genaue Position der Implantate. Während das
subretinale Implantat zwischen Netzhaut und Aderhaut fixiert wird, sitzt das epiretinale Implantat
auf der Netzhaut. Weiterhin kommt bei einem subretinalen Implantat eine Matrix aus circa
\numprint{1500} Photodioden zum Einsatz um das Bild in Helligkeitssignale zu wandeln. Beim
epiretinalen
Implantat übernimmt dies eine externe Kamera die meist in einer Brille integriert ist. Durch die
externe Kamera ergibt sich ein wesentlicher Nachteil für den Träger, da die Beweglichkeit des Auges
nicht genutzt werden kann.

\begin{figurewrapper}
	\href{http://www.oe.uni-duisburg-essen.de/latestnews/augenblicke/Abb5.jpg}{%
		\includegraphics[width=0.6\hsize]{files/images/Robin/eye/subretinale-Implantat-Abb5}%
	}
	\captionof{figure}{Funktionsweise eines subretinalen Implantates}
	\label{fig:Subretinal_Implant}
\end{figurewrapper}

Das subretinale Implantat (siehe Abb.~\vref{fig:Subretinal_Implant}) ist momentan das
vielversprechendste Retinaimplantat. Ende 2010 erhielten
drei Patienten, mit dieser Technik, wieder die Fähigkeit zu sehen. In Tests gelang es einem 44
Jährigen sogar eine Banane von einem Apfel zu unterscheiden, sich in einem Raum bewegen und
Riesenbuchstaben zu lesen.\footcite{Independent:retina_chip}
