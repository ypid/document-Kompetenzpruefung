\section{Jetziger Stand der Technik}
\label{sec:Robin:topical}

In diesem Kapitel gebe ich einen Überblick über die heute eingesetzten Verfahren in der Medizin, um
verlohrengegengene Funktionelle wiederherzustellen.

\subsection{Beinprothese}

\newcommand{\URLCLeg}{http://www.tuvie.com/wp-content/uploads/c-leg.jpg}
\begin{wrapfigure}{r}{0pt} %% Einbindung steht im Konflikt mit dem Urheberrecht
	\href{\URLCLeg}{\includegraphics[width=1.8cm]%
		{files/images/Robin/C-Leg/c-leg-rm-bg}%
	}
	\captionof{figure}[C-Leg]%
	{C-Leg\footnotemark}%
	\label{fig:C-Leg}
\end{wrapfigure}
\footnotetext{Hintergrund entfernt und zugeschnitten.
Das Original ist hier zu finden: \url{\URLCLeg}}

1997 kam mit dem C-Leg\footnote{\enquote{Leg} ist englisch und bedeutet Bein.} von der Otto Bock
HealthCare GmbH eine revolutionäre Beinprothese auf den Markt. Ein Zusammenspiel von
Mikroprozessoren, Sensoren, Programmlogik und einem Hydrauliksystem sorgt bei dieser Prothese für
sicheren Stand (Standbeinsicherung). Beim gehen zeigen sich die größten Neuerungen. Durch die
komplexe Sensorik, die mit \SI{50}{\hertz} ausgewertet wird, ist ein sicheres gehen bis laufen
möglich. Mit dieser Prothese sind auch Schrägen oder Treppen kein Problem. Über einen
Aktivitätsmodus können neue Bewegungsarten wie Eislaufen oder Ski-Langlauf individuell Einstellung
werden. Das C-Leg ist für Patienten bis zu einem Gewicht von \SI{136}{\kilo\gram} ausgelegt. Die
Akkukapazität ist mit 30 bis 35 Stunden angegeben.
Diese Prothese wurde über die vergangenen Jahre, vor allem durch Rückmeldung der Kunden, sehr
verbessert. Damit zählt es heute zu der ausgereiftesten Beinprothese.
Der Hersteller verspricht eine Unbeschwerte, vorher nie erreichte Beweglichkeit mit dieser Prothese.
Dieses Versprechen kann er offensichtlich auch halten. Momentan ist das C-Leg bei circa
\numprint{40000} Patienten im
Einsatz.\footnote{\url{http://www.healthcare.ottobock.de/oba/ch/sites/orth/c_leg.html}}

\subsection{Armprothese}
Heutige Armprothesen Arbeiten teils mit Myoelektrik. Dabei wird über empfindliche Elektroden die
Spannung\footnote{im Mikrovoltbereich} an der Hautoberfläche gemessen. Diese Spannung wird von den
Muskelzellen erzeugt. Somit kann mit noch vorhandenen Muskeln eine Prothese gesteuert werden.

Sowohl das Laufen mit einer Beinprothese, als auch der Umgang mit einer Armprothese muss heute mühsam
erlernt werden, da die entsprechenden Gliedmaßen über andere Muskeln angesteuert werden müssen. Dies
ist möglich, da sich unser Gehirn sehr gut neu konfigurieren kann.\footnote{Diese Eigenschaft ist
noch relevanter bei der Ergänzung neuer Funktionen, über die der Mensch vorher nicht verfügte.}

\subsection{Hörgeräte}
\label{sec:Robin:topical:hearing_aid}
Heute sind zwei Formen von Hörgeräten üblich. Die Hinter-dem-Ohr- und die Im-Ohr-Geräte. Für
bestimmte Fälle werden Teile des Hörgeräts auch direkt implantiert, um beschädigte Teile im Ohr zu
ersetzen.

Bei den Hinter-dem-Ohr-Hörgeräten sitzt die komplette Technik hinter der Ohrmuschel. Die verstärkten
Töne werden über einen Schallschlauch und ein individuelles Ohrpassstück ins Ohr des Trägers
weitergeleitet. Diese Geräte sind am vielseitigsten einsetzbar, da der Teil hinter dem Ohr
ausreichend Platz für die Technik zur Verfügung stellt. So können mit diesen Hörgeräten hohe
Verstärkungsleistungen erreicht werde und viele Audiotransformationen angewendet werden.

Die Im-Ohr-Geräte verschwinden im Gehörgang. Somit muss die komplette Technik in einer individuellen
Hohlschale Platz finden. Der Vorteil ist, dass die anatomischen Gegebenheiten des Außenohres genutzt
werden können und so eine bessere Positionsbestimmung der Tonquelle möglich wird. Allerdings lässt
sich ein solches Gerät erst ab einer ausreichenden Größe des Gehörgangs realisieren. Probleme sind
hier fehlender Druckausgleich und Rückkopplungseffekte bei hohen Verstärkungsleistungen. Zudem wird
die Ohrenschmelzbildung durch die geringe Belüftung begünstigt was zu mehr Ausfällen der Geräte
führt.

\bigskip

\begin{wrapfigure}{r}{0pt} %% Einbindung steht im Konflikt mit dem Urheberrecht
	\href{\URLCLeg}{\includegraphics[width=9cm]%
		{files/images/Robin/esteem}%
	}
	\captionof{figure}{System Esteem}%
	\label{fig:Esteem}
\end{wrapfigure}
Da eine Hörprothese nur eingesetzt werden kann, wenn ein Mensch nicht komplett taub ist, wird
versucht, näher an die Ursache zu kommen, um auch diese Fälle abzudecken. Dies geschieht mit
Implantaten. Da der Defekt an vielen Stellen liegen kann, wurden auch viele Systeme entwickelt, um
diese abdecken zu können.

Eine Lösung kann ein Mittelohrimplantat sein. Es gibt mehrere Systeme, die dieses Prinzip umsetzen.
Ich beschreibe das Prinzip am System Esteem. Dieses Implantat besteht aus Audioprozessor (dieser
enthält auch die Batterie), Sensor und einem Treiber. Der Sensor wandelt die Schwingungen an der
Gehörknöchelchenkette piezoelektrisch in elektrische Signale um. Diese werden im Audioprozessor
verstärkt. Es lassen sich auch individuelle Filter auf das Signal anwenden. Der Treiber setzt dieses
Signal wieder in Schwingungen um und leitet sie an den Steigbügel weiter. Damit dieses Implantat
funktioniert, muss die Verbindung zwischen Amboss und Steigbügel unterbrochen werden, um
Rückkopplungseffekte zu vermeiden. Durch dieses Prinzip bleibt die
Schalltrichterwirkung der Ohrmuschel weiterhin nutzbar. Das Implantat
kann über eine Fernbedienung ein- und ausgeschaltet, sowie die Lautstärke reguliert werden. Das
Einstellen der Lautstärke ist eine häufig eingesetzte Funkton\fxwarning{?}, da die natürliche
Regulierung außer Kraft gesetzt ist. Bei ausgeschaltetem Hörgerät tritt eine extreme
Schallleitungsschwerhörigkeit auf. Deshalb muss die Verbindung der Gehörknöchelchenkette bei einem
Entfernen des Implantats wieder hergestellt werden.

Das Innenohrimplantat hingegen setzt direkt an der Cochlea (Hörschnecke) an. Über eine externe
Einheit, bestehend aus Akku, Mikrofon, digitalen Signalprozessor und einer Sendespule wird der
Schall zum Implantat übertragen. Diese Einheit wird mit Magneten an der Kopfhaut hinter dem Ohr,
direkt über dem Implantat, befestigt. Das Implantat empfängt die Signale der externen Einheit über
eine Spule und reizt über Stimulationselektroden die Hörschnecke. Das Implantat verfügt dabei über
keine eigene Stromversorgung. Stattdessen wird die benötigte Energie über Induktion zum Implantat
übertragen.

Bei Erwachsenen, die bereits vor dem Erlernen der ersten Sprache taub wurden, wird meist von einem
Innenohrimplantat abgeraten, da es unwahrscheinlich ist, dass dieser Mensch das Hören noch erlernen
kann.

Die Kosten für zwei Cochlea-Implantate werden in Deutschland heutzutage von den Krankenkassen
übernommen. Anfangs (vor 2000) wurde meist nur ein Cochlea-Implantat eingesetzt (also entweder für
das rechte oder linke Ohr) auch wenn auf beiden Seiten eine Schwerhörigkeit vorlag. Dies lag daran,
dass die Krankenkassen nur eines finanziert haben und wegen der Hoffnung nach besseren Implantaten
oder Behandlungsmetoden. Heute werden meist gleich beide Ohren mit einem Implantat versorgt.

\subsection{Sehprothesen}
\label{sec:Robin:topical:visual_prosthesis}
Die einfachste Form einer Sehprothesen ist das heute immer noch eingesetzte Glasauge. Allerdings
ist dieser chirurgische Eingriff heute so perfektioniert worden, dass ein Unterschied zu einem
richtigen Auge erst bei näherer Betrachtung auffällt. Die natürliche Bewegung des Augapfels wird
ebenfalls wieder hergestellt, indem das Glasauge mit den Muskeln in der Augenhöhle verbunden wird.
Allerdings muss ein Glasauge jedes Jahr gegen ein neues ausgetauscht werden, da sich die glatte
Oberfläche des Glases oder Kunststoffes durch Schmutz und Tränenflüssigkeit abnutzt und rauer
wird. Die raue Oberfläche würde zu Beschädigung der Augenhöhle führen.
Die Kosten für ein Glasauge werden von den Krankenkassen in Deutschland fast vollständig übernommen.

Mit einem Glasauge ist natürlich kein Sehen möglich. Um eine wirkliche Hilfe zu bieten, wird derzeit
am sogenannten Retinaimplantat geforscht. Dieses befindet sich noch im experimentellen Stadium.
Es setzt direkt am
Sehnerv an und ist somit nicht auf eine funktionierende Netzhaut (Retina) angewiesen. Ein
Retinaimplantat wandelt das Bild der Umgebung in elektrische Signale um und leitet diese an einen
digitalen
Signalprozessor weiter. Dieser reizt über ein Stimulations-Elektroden-Array den Sehnerv. Die
Energieversorgung erfolgt in der Regel über Infrarotstrahlung oder Induktion von einer Brille aus.
Aktuell ist mit diesem Implantat eine Unterscheidung zwischen Licht und Schatten möglich. Es gibt
momentan zwei Systeme im experimentellen Stadium. Das subretinale Implantat und das epiretinale
Implantat. Der wesentliche Unterschied ist die genaue Position der Implantate. Während das
subretinale Implantat zwischen Netzhaut und Aderhaut fixiert wird, sitzt das epiretinale Implantat
auf der Netzhaut. Weiterhin kommt bei einem subretinalen Implantat eine Matrix aus circa
\numprint{1500} Photodioden zum Einsatz um das Bild in Helligkeitssignale zu wandeln. Beim
epiretinalen
Implantat übernimmt dies eine externe Kamera die meist in einer Brille integriert ist. Durch die
externe Kamera ergibt sich ein wesentlicher Nachteil für den Träger, da die Beweglichkeit des Auges
nicht genutzt werden kann.

\begin{figurewrapper}
	\href{http://www.oe.uni-duisburg-essen.de/latestnews/augenblicke/Abb5.jpg}{%
		\includegraphics[width=0.6\hsize]{files/images/Robin/eye/subretinale-Implantat-Abb5-mod}%
	}
	\captionof{figure}[Funktionsweise eines subretinalen Implantates]%
		{Funktionsweise eines subretinalen Implantates\footnotemark}
	\label{fig:Subretinal_Implant}
\end{figurewrapper}
\footnotetext{Modifiziert}

Das subretinale Implantat (siehe Abb.~\vref{fig:Subretinal_Implant}) ist momentan das
vielversprechendste Retinaimplantat. Ende 2010 erhielten
drei Patienten, mit dieser Technik, wieder die Fähigkeit zu sehen. In Tests gelang es einem 44
Jährigen sogar eine Banane von einem Apfel zu unterscheiden, sich in einem Raum zu bewegen und
Riesenbuchstaben zu lesen.\footcite{Independent:retina_chip}

\subsection{Herzschrittmacher}
\label{sec:Robin:topical:Pacemaker}
Heutzutage sind Herzschrittmacher gut funktionierende Systeme, die bei Personen mit Herzproblemen mit
75 Jahren\footnote{Statistischer Mittelwert} zum Einsatz kommen. Durch einen Herzschrittmacher werden
Medikamente überflüssig, die ohne Herzschrittmacher eingenommen werden müssten.
In bestimmten Fällen wird auch ein implantierbarer Kardioverter-Defibrillator implantiert.

Ein Herzschrittmacher besteht aus der Hauptkomponente und der Elektrode. Die Hauptkomponente vereint
Impulsgeber, Steuerelektronik und eine Energiequelle in einem Gehäuse. Die Elektrode dient sowohl zur
Stimulation des Herzens als auch zur Messung der Herzfunktion. Aktuelle Herzschrittmacher überwachen
permanent das Herz und greifen nur noch ein, wenn innerhalb einer einstellbaren Zeitdauer kein
Herzschlag erkannt wird. Die Konfiguration kann von einem Arzt drahtlos erfolgen.

Als Energiequelle dient heute eine Lithiumbatterie. Damit sind Laufzeiten von durchschnittlich acht
Jahren möglich. Eine Aufladung über Induktion ist heute nicht mehr gebräuchlich, stattdessen wird der
Herzschrittmacher einfach ausgetauscht.

\subsection{Implantate nach dem Tod}
Die Entfernung eines Implantates ist nur mit Zustimmung des Verstorbenen oder seiner Angehörigen
möglich. Das Implantat wird also in der Regel auch nach dem Tod im Körper bleiben. In den 70er Jahren
wurden teils Herzschrittmacher aus den Verstorbenen entfernt. Diese wurden dann, nach einer
Aufarbeitung, wieder in Patienten eingesetzt. Dabei wurde verschwiegen, dass es sich um gebrauchte
Geräte handelte.\footcite{Spiegel:Herzschrittmacher-Prozess}
