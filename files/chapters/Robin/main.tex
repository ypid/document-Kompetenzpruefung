\chapter{Robin Schneider}
\label{sec:Robin_Schneider}
\newcommand{\URLeiserneHand}%
{http://commons.wikimedia.org/w/index.php?title=File:Götz-eiserne-hand1.jpg&oldid=52409346}

\newcommand{\URLQuinnMagnet}{http://makezine.com/hackszine/magnet.jpg}
\newcommand{\URLsubretinaleImplantat}%
{http://www.oe.uni-duisburg-essen.de/latestnews/augenblicke/Abb5.jpg}

% Physikalische Veränderung um Einschränkungen abzubauen oder neue Funktionen zu erhalten

\section{Vorwort}
In diesem Kapitel werde ich mich mit verschiedenen Möglichkeiten der Veränderung am menschlichen Körper beschäftigen.
Es geht mir in erster Linie um physikalische Veränderung die zu dem Zweck durchgeführt werden,
um körperliche Einschränkungen auszugleichen oder um den Menschen um neue Funktionen zu erweitern.
Ich werde mich nicht mit Veränderungen aus Schönheitsgründen
oder anderen gesellschaftlichen Idealen beschäftigen.
Diesen Teil hat \nameref{sec:Hannah_Haid} übernommen.
Ebenfalls werde ich nicht Körperveränderung zum Verbessern des Sexuallebens beleuchten.
Auch chemische Substanzen, die das Gehirn beeinflussen,
um eine Leistungssteigerung zu erreichen, werde ich außen vor lassen.

%% Die technische Dokumentation geht nicht sehr in die Tiefe -> nicht das Ziel dieser Arbeit.

\bigskip
Anfangen werde ich mit einem \hyperref[sec:Robin:historical_overview]{geschichtlichen Überblick},
in dem ich die wichtigsten Entwicklungsschritte kurz anspreche, mit denen der heutige
Entwicklungsstand erst Möglich wurde.

Anschließend beschreibe ich, was heute alles Möglich ist.
Dies ist der umfangreichste Teil meiner Arbeit. Einerseits habe ich mich darin mit den
\hyperref[sec:Robin:topical]{medizinischen Prothesen und Implantaten} beschäftigt, die heute
verwendet werden. Schwerpunkte sind für mich die
\nameref{sec:Robin:topical:hearing_aid} und \nameref{sec:Robin:topical:visual_prosthesis}.
Andererseits habe ich zwei eigene \hyperref[sec:Robin:experiments]{Experimente} durchgeführt, um am
eigenen Leib zu erfahren, wo sich, mit geringem Aufwand, etwas am eigenen Körper erweitern lässt.

Den Abschluss bildet ein \nameref{sec:Robin:future} in dem ich Entwicklungen aufgreife, die sich
bereits ankündigen, aber auch Dinge, die wahrscheinlich noch Jahrzehnte auf sich warten lassen
werden. Am Ende dieses Kapitels habe ich mich noch mit Problemen, Risiken und ethischen
Fragen\fxwarning{Habe ich?} beschäftigt.

\section{Geschichtlicher Überblick}
\label{sec:Robin:historical_overview}

Durch Kriege, Unfälle oder durch Krankheiten kann eine Person durch fehlende Gliedmaßen oder andere
Probleme deutlich eingeschränkt sein. Da Körperteile nicht einfach nachwachsen, musste sich die
Menschheit etwas ausdenken, um diese Komponenten anderweitig zu ersetzen. Für diesen Zweck wurden
viele Hilfsmittel und Methoden erfunden. Ein sehr früh in der Geschichte auftretendes Hilfsmittel ist
die Prothese. Eine Prothese ist ein Ersatz für Gliedmaßen oder Organe durch ein künstliches Produkt.
Ist die Prothese außerhalb des Körpers, spricht man von einer Exoprothese (in der Alltagssprache
meist nur als Prothese bezeichnet). Hierunter fallen Arm-, Bein-, Hand- und Fußprothesen. Wenn die
Prothese vollständig im Körper sitzt, spricht man hingegen von einer Endoprothese oder einem
Implantat. Beispiele sind künstliche Gelenke, Herzklappen oder die Ersetzung von Organen wie Herz
oder Leber. Allerdings lassen sich nicht alle Prothesen so klar unterteilen. Es gibt auch Prothesen,
die aus dem Körper herausragen wie zum Beispiel ein Zahnimplantat. Ich werde im folgenden die
wichtigsten Schritte in chronologischer Reichenfolge  beleuchten, die den heutigen Stand der
Entwicklung erst ermöglichten.
% Im weiteren Text werde ich die Begriffe Prothese und Implantat benutzen.

Die Prothetik, also die Entwicklung und Herstellung von Prothesen, hat eine lange Tradition. So gehen
Medizinhistoriker davon aus, dass bereits 2000 Jahre vor Christus amputierte Gliedmaßen durch
künstliche Körperteile ersetzt wurden. Dies erfüllten anfangs nur ästetische Zwecke, was sich im
Laufe der Geschichte aber deutlich änderte.

Die älteste gefundene Prothese wurde 600 vor Christus von einer Ägypterin benutzt. Sie wurde mit 50
Jahren mit einer hölzernen Zehprothese beigesetzt. Der künstliche Zeh zeigte deutliche
Abnutzungsspuren, sodass davon ausgegangen werden kann, dass es sich hierbei nicht um eine
Grabbeigabe für ein Leben nach dem Tod handelte, sondern um ein nützliches
Hilfsmittel.\footcite{Lancet:origin_prosthetic} Es gibt viele weitere archäologische Funde von frühen
Prothesen die über zwei Jahrtausende alt sind. So wurden in vielen Kulturen schon viele zerstörte
Zähne durch passende aus Elfenbein, Holz oder menschlichen Zähnen ersetzt. Diese neuen Zähne wurden
mit Draht an den anderen Zähnen befestigt. Doch dieser frühe Zahnersatz wurde nur aus ästetischen
Gründen und zu Verbesserung der Aussprache getragen. Beim Essen hatte dieser keine positive Funktion.
Im Gegenteil: Der Zahnersatz führte zu Entzündungen im Mund.

In der Geschichte wurden schon sehr früh Holzbeine eingesetzt um ein fehlendes Bein zu ersetzten. Es
handelte sich meist um eine einfache Holzstelze mit denen eine Fortbewegung wieder möglich wurde. Ein
Beleg dafür ist der französischer Pirat François Le Clerc, der aus diesem Grund auch den Spitznamen
\enquote{Holzbein} bekam. Er lebte im 16. Jahrhundert und verlor bei einem Überfall ein Bein.
%% http://de.wikipedia.org/wiki/François_Le_Clerc
Das Bild, das wir heute von Piraten haben, ist also nicht ohne Grund ein einbeiniger mit Holzbein,
der eine Augenklappe trägt. Aber auch schon die Ägypter nutzten Holzbeine.

Nach diesen einfachen Prothesen tauchte 1504 die sogenannte eiserne Hand des Ritters Götz von
Berlichingen auf. Nachdem die rechte Hand von Götz in einer Schlacht von einer Kanonenkugel
zerschmettert wurde und diese vorsorglich amputiert wurde, konstruiert ein Waffenschmied eine, für
diese Zeit, sehr fortschrittliche Handprothese, die auch längere Zeit danach einmalig blieb. Die
eiserne Hand sah wie ein Handschuh (siehe Abb.~\vref{fig:eiserneHand}) aus und wurde mit
Lederriemen am noch
\begin{wrapfigure}{r}{0pt}
	\href{\URLeiserneHand}{\includegraphics[width=5cm]%
		{files/images/Robin/Goetz-eiserne-hand1}%
	}
	\captionof{figure}{Die eiserne Hand des Ritters Götz von Berlichingen}%
	\label{fig:eiserneHand}
\end{wrapfigure}
vorhandenem Unterarm befestigt. Die Finger konnten gedreht und fixiert werden. Die Prothese konnte
außerdem nach oben und unten geschwenkt werden. So konnte Götz unter Zuhilfename seiner gesunden
linken Hand mit dieser Prothese Gegenstände greifen und festhalten. Er konnte sogar sein Schwert
halten und dadurch sein Beruf mit gewissen Einschränkungen, weiter ausüben.
%% http://de.wikipedia.org/wiki/Eiserne_Hand_(Götz_von_Berlichingen)

Im Mittelalter tauchten auch erste Implantate für ein fehlendes Auge auf. Diese Einlegeaugen waren
anfangs aus Edelmetallen wie Gold und Silber gefertigt. Im 17. Jahrhundert traten dann die ersten
Glasaugen auf. Diese Augenprothesen
wurden hauptsächlich eingesetzt,
um die Gesichtsharmonie wieder herzustellen.

Im 17. Jahrhundert wurden erste Hörgeräte erfunden. Dabei handelte es sich noch um einfache Hörrohre
die Umgebungsgeräusche um 20 bis 30 Dezibel verstärkten. Eine Verbesserung wurde erst 1878 von
Werner von Siemens mit einem speziellen Telefonhörer erfunden. 1914 wurde dieses Hörgerät mit
Einsteckhörer ausgestattet, um es unauffälliger zu machen. Die Geräte wurden nun immer weiter
miniaturisiert. 1952 kamen erstmals Transistoren zum Einsatz, wodurch die Größe auf eine
Zigarettenschachtel schrumpfte. 1966 brachte Siemens ein Hörgerät auf den Markt, das vollständig im
Gehörgang verschwand. Die nächsten Schritte waren dann der Einsatz von digitalen Signalprozessoren
und die damit verbundene Leistungssteigerung und weitere Miniaturisierung.

Nachdem die Entwicklung an Beinprothese Lagezeit fast stillstand, wurde das simple Holzbein im 19.
Jahrhundert bis zu einem Modell mit gefedertem Prothesenfuß und beweglichem Kniegelenk weiter
entwickelt. Dadurch konnte der Träger die Beinprothese beim Sitzen wie ein normales Bein am
Kniegelenk einklappen.

Die Wichtigkeit der Prothetik nahm im 20. Jahrhundert aufgrund der vielen Verstümmelten des Ersten
und Zweiten Weltkrieges enorm zu. Dies Sorgte für eine rasante Entwicklung in dieser Zeit. So wurden
immer mehr Prothesen mit Sprungfedern und Gelenken konstruiert. Es ist beispielsweise möglich
geworden, dass künstliche Kniegelenke beim Laufen mechanisch einknickten, wenn es zu einer
Schwerpunktverlagerung kam. Zusätzlich wurden Seilzüge und Bolzen dazu genutzt, die Kraft von noch
vorhandenen Muskeln zur Betätigung von einfachen Greifern zu benutzen. Eine Umsetzung dieser Idee
lieferte 1916 der deutsche Chirurg Ferdinand Sauerbruch mit der nach ihm benannten Sauerbruch-Arm.
Für diese Handprothese wurde ein Kanal durch die Muskeln gelegt. In diese wurden dann Bolzen
eingeführt, mit denen die Handprothese und sogar die Finger gesteuert werden konnten. Die Kraft, die
mit dieser Prothese ausgeübt werden konnte, war aber bei Weitem schwächer als die einer natürliche
Hand. Der Sauerbruch-Arm konnte sich nie richtig Durchsetzten. Dies lag hauptsächlich an zwei
Problemen. Zum einen traten in dem Kanal oft Entzündungen und Infektionen
auf.\footcite{thesis:Karpa:Geschichte_Armprothesen}
Andererseits war die
Prothese für die meisten verwundeten Soldaten des Ersten Weltkrieges zu teuer.

Ende der 50. Jahre wurde der erste Herzschrittmacher, der vollständig im Körper untergebracht wurde,
bei Arne Larsson eingesetzt. Das Gerät war noch sehr einfach aufgebaut. Es bestand aus zwei
Transistoren, die eine Kippschaltung bildeten, einer Quecksilberakkuzelle und einer Spule zum
Aufladen des Akkus über Induktion. Das externe Nachladen war jede Woche nötig, da die Energiedichte
dieser Akkutechnologie noch deutlich unter der heutigen war. Die Bauteile wurden in einer mit
Epoxidharz versiegelten Schuhcremedose untergebracht. Aus dem Gehäuse ging ein Kabel zu den
Elektroden, die auf das Herz aufgenäht wurden.

Ein wesentliches Problem war die kurze Akkulaufzeit. Deshalb wurden in der nächsten Generation
von Herzschrittmachern der Zerfallsprozess von \SI{200}{\milli\gram} Plutonium-238 als Energiequelle
genutzt. Die Entsorgung dieser geringen Mengen radioaktiven
Materials stellt heute gewisse Probleme dar.\footcite{DRadio:strahlendes_Herz}

1980 wurde es erstmals möglich, Armprothesen mit Motoren zu konstruieren. Dadurch musste nicht mehr,
wie dies noch bei dem Sauerbruch-Arm der Fall war, die Kraft von den verbleibenden Muskeln über
Mechanik zur Hand gelenkt werden. Diese Prothesen konnten über die Kontraktion noch vorhandenen
Muskeln gesteuert werden.

\bigskip
Soviel zur Vergangenheit. An eine funktionelle Ersetzung komplizierterer Körperteile war vor einiger
Zeit noch nicht zu denken. Heute sind viele Eingriffe zur Routine geworden.

\section{Jetziger stand der Technik}
\label{sec:Robin:topical}

In diesem Kapitel gebe ich einen Überblick über die heute eingesetzten Verfahren in der Medizin um
verlohrengegengene Funktionelle wiederherzustellen oder neue zu ergänzen\fxwarning{Neue Funktionen
in diesem Kapitel vorhanden?}.

\subsection{Beinprothese}

\newcommand{\URLCLeg}{http://www.tuvie.com/wp-content/uploads/c-leg.jpg}
\begin{wrapfigure}{r}{0pt} %% Einbindung steht im Konflikt mit dem Urheberrecht
	\href{\URLCLeg}{\includegraphics[width=2.3cm]%
		{files/images/Robin/C-Leg/c-leg-rm-bg}%
	}
	\captionof{figure}[C-Leg]%
	{C-Leg\footnotemark}%
	\label{fig:Quinn_Norton_magnet}
\end{wrapfigure}
\footnotetext{Hintergrund entfernt und zugeschnitten.
Das Original ist hier zu finden: \url{\URLCLeg}}

1997 kam mit dem C-Leg\footnote{\enquote{Leg} ist englisch und bedeutet Bein.} von der Otto Bock
HealthCare GmbH eine revolutionäre Beinprothese auf den Markt. Ein Zusammenspiel von
Mikroprozessoren, Sensoren, Programmlogik und einem Hydrauliksystem sorgt bei dieser Prothese für
sicheren Stand (Standbeinsicherung). Beim gehen zeigen sich die größten Neuerungen. Durch die
komplexe Sensorik, die mit \SI{50}{\hertz} ausgewertet werden, ist ein sicheres gehen bis laufen
möglich. Mit dieser Prothese sind auch Schrägen oder Treppen kein Problem. Über einen
Aktivitätsmodus können neue Bewegungsarten wie Eislaufen oder Ski-Langlauf individuell Einstellung
werden. Das C-Leg ist für Patienten bis zu einem Gewicht von \SI{136}{\kilo\gram} ausgelegt. Die
Akkukapazität ist mit 30 bis 35 Stunden angegeben.
Diese Prothese wurde über die vergangenen Jahre, vor allem durch Rückmeldung der Kunden, sehr
verbessert. Damit zählt es heute zu der ausgereiftesten Beinprothese.
Der Hersteller verspricht eine Unbeschwerte, vorher nie erreichte Beweglichkeit mit dieser Prothese.
Dieses Versprechen kann er offensichtlich auch halten. Momentan ist das C-Leg bei circa
\numprint{40000} Patienten im
Einsatz.\footnote{\url{http://www.healthcare.ottobock.de/oba/ch/sites/orth/c_leg.html}}

%Dies ist nicht zuletzt der Informatik zu verdanken, die zusätzlich eine Programmierbarkeit erlaubt.
%Häufige Bewegungen wie etwa Treppensteigen sind dadurch viel einfacher ausführbar.

\subsection{Hörgeräte}
\label{sec:Robin:topical:hearing_aid}
Heute sind zwei Formen von Hörgeräten üblich. Die Hinter-dem-Ohr- und die Im-Ohr-Geräte. Für
bestimmte Fälle werden Teile des Hörgeräts auch direkt implantiert, um beschädigte Teile im Ohr zu
ersetzen.

Bei den Hinter-dem-Ohr-Hörgeräten sitzt die komplette Technik hinter der Ohrmuschel. Die verstärkten
Töne werden über einen Schallschlauch und ein individuelles Ohrpassstück ins Ohr des Trägers
weitergeleitet. Diese Geräte sind am vielseitigsten einsetzbar, da der Teil hinter dem Ohr
ausreichend Platz für die Technik zur Verfügung stellt. So können mit diesen Hörgeräten hohe
Verstärkungsleistungen erreicht werde und viele Audiotransformationen angewendet werden.

Die Im-Ohr-Geräte verschwinden im Gehörgang. Somit muss die komplette Technik in einer individuellen
Hohlschale Platz finden. Der Vorteil ist, dass die anatomischen Gegebenheiten des Außenohres genutzt
werden können und so eine bessere Positionsbestimmung der Tonquelle möglich wird. Allerdings lässt
sich ein solches Gerät erst ab einer ausreichenden Größe des Gehörgangs realisieren. Probleme sind
hier fehlender Druckausgleich und Rückkopplungseffekte bei hohen Verstärkungsleistungen. Zudem wird
die Ohrenschmelzbildung durch die geringe Belüftung begünstigt was zu mehr Ausfällen der Geräte
führt.

Da eine Hörprothese nur eingesetzt werden kann, wenn ein Mensch nicht komplett taub ist, wird
versucht, näher an die Ursache zu kommen, um auch diese Fälle abzudecken. Dies geschieht mit
Implantaten. Da der Defekt an vielen Stellen liegen kann, wurden auch viele Systeme entwickelt, um
diese abdecken zu können.

Eine Lösung kann ein Mittelohrimplantat sein. Es gibt mehrere Systeme, die dieses Prinzip umsetzen.
Ich beschreibe das Prinzip am System Esteem. Dieses Implantat besteht aus Audioprozessor (dieser
enthält auch die Batterie), Sensor und einem Treiber. Der Sensor wandelt die Schwingungen an der
Gehörknöchelchenkette piezoelektrisch in elektrische Signale um. Diese werden im Audioprozessor
verstärkt. Es lassen sich auch individuelle Filter auf das Signal anwenden. Der Treiber setzt dieses
Signal wieder in Schwingungen um und leitet sie an den Steigbügel weiter. Damit dieses Implantat
funktioniert, muss die Verbindung zwischen Ambos und Steigbügel unterbrochen werden, um
Rückkopplungseffekte zu vermeiden. Durch dieses Prinzip bleibt die
Schalltrichterwirkung der Ohrmuschel weiterhin nutzbar. Das Implantat
kann über eine Fernbedienung ein- und ausgeschaltet, sowie die Lautstärke reguliert werden. Das
Einstellen der Lautstärke ist eine häufig eingesetzte Funkton\fxwarning{?}, da die natürliche
Regulierung außer Kraft gesetzt ist. Bei ausgeschaltetem Hörgerät tritt eine extreme
Schallleitungsschwerhörigkeit auf. Deshalb muss die Verbindung der Gehörknöchelchenkette bei einem
Entfernen des Implantats wieder hergestellt werden.

Das Innenohrimplantat hingegen setzt direkt an der Cochlea (Hörschnecke) an. Über eine externe
Einheit, bestehend aus Akku, Mikrofon, digitalen Signalprozessor und einer Sendespule wird der
Schall zum Implantat übertragen. Diese Einheit wird mit Magneten an der Kopfhaut hinter dem Ohr,
direkt über dem Implantat, befestigt. Das Implantat empfängt die Signale der externen Einheit über
eine Spule und reizt über Stimulationselektroden die Hörschnecke. Das Implantat verfügt dabei über
keine eigene Stromversorgung. Stattdessen wird die benötigte Energie über Induktion zum Implantat
transportiert.

Bei Erwachsenen, die bereits vor dem Erlernen der ersten Sprache taub wurden, wird meist von einem
Innenohrimplantat abgeraten, da es unwahrscheinlich ist, dass dieser Mensch das Hören noch erlernen
kann.

Die Kosten für zwei Cochlea-Implantate werden in Deutschland heutzutage von den Krankenkassen
übernommen. Anfangs (vor 2000) wurde meist nur ein Cochlea-Implantat eingesetzt (also entweder für
das rechte oder linke Ohr) auch wenn auf beiden Seiten eine Schwerhörigkeit vorlag. Dies lag daran,
dass die Krankenkassen nur eines finanziert haben und wegen der Hoffnung nach besseren Implantaten
oder Behandlungsmetoden. Heute werden meist gleich beide Ohren mit einem  Cochlea-Implantat versorgt.

\subsection{Sehprothesen}
\label{sec:Robin:topical:visual_prosthesis}
Die einfachste Form eines Augenimplantats ist, das heute immer noch Eingesetze, Glasauge. Allerdings
ist dieser chirurgischen Eingriff heute so perfektioniert worden, dass ein Unterschied zu einem
richtigen Auge erst bei näherer Betrachtung auffällt. Die natürliche Bewegung des Augapfels wird
ebenfalls wieder hergestellt, indem das Glasauge mit den Muskeln in der Augenhöhle verbunden wird.
Allerdings muss ein Glasauge jedes Jahr gegen ein neues ausgetauscht werden, da sich die glatte
Oberfläche des Glases oder Kunststoffes durch Schmutz und Tränenflüssigkeit abnutzt und rauer
wird. Die raue Oberfläche würde zu Beschädigung der Augenhöhle führen.
Die Kosten für ein Glasauge werden von den Krankenkassen in Deutschland fast vollständig übernommen.

Mit einem Glasauge ist natürlich kein sehen möglich. Um eine wirkliche Hilfe zu bieten, wird derzeit
am sogenannten Retinaimplantat geforscht. Dieses befindet sich noch im experimentellen Stadium.
Es setzt direkt am
Sehnerv an und ist somit nicht auf eine funktionierende Netzhaut (Retina) angewiesen. Ein
Retinaimplantat wandelt das Bild der Umgebung in elektrische Signale um leitet es an einen digitalen
Signalprozessor weiter. Dieser reizt über ein Stimulations-Elektroden-Array den Sehnerv. Die
Energieversorgung erfolgt in der Regel über Infrarotstrahlung oder Induktion von einer Brille aus.
Aktuell ist mit diesem Implantat eine Unterscheidung zwischen Licht und Schatten möglich. Es gibt
momentan zwei Systeme im experimentellen Stadium. Das subretinale Implantat und das epiretinale
Implantat. Der wesentliche Unterschied ist die genaue Position der Implantate. Während das
subretinale Implantat zwischen Netzhaut und Aderhaut fixiert wird, sitzt das epiretinale Implantat
auf der Netzhaut. Weiterhin kommt bei einem subretinalen Implantat eine Matrix aus circa
\numprint{1500} Photodioden zum Einsatz um das Bild in Helligkeitssignale zu wandeln. Beim
epiretinalen
Implantat übernimmt dies eine externe Kamera die meist in einer Brille integriert ist. Durch die
externe Kamera ergibt sich ein wesentlicher Nachteil für den Träger, da die Beweglichkeit des Auges
nicht genutzt werden kann.

\begin{figurewrapper}
	\href{http://www.oe.uni-duisburg-essen.de/latestnews/augenblicke/Abb5.jpg}{%
		\includegraphics[width=0.6\hsize]{files/images/Robin/eye/subretinale-Implantat-Abb5}%
	}
	\captionof{figure}{Funktionsweise eines subretinalen Implantates}
	\label{fig:Subretinal_Implant}
\end{figurewrapper}

Das subretinale Implantat (siehe Abb.~\vref{fig:Subretinal_Implant}) ist momentan das
vielversprechendste Retinaimplantat. Ende 2010 erhielten
drei Patienten, mit dieser Technik, wieder die Fähigkeit zu sehen. In Tests gelang es einem 44
Jährigen sogar eine Banane von einem Apfel zu unterscheiden, sich in einem Raum bewegen und
Riesenbuchstaben zu lesen.\footcite{Independent:retina_chip}

\subsection{Herzschrittmacher}
\label{sec:Robin:topical:Pacemaker}
Heutzutage sind Herzschrittmacher gut funktionierende Systeme, die bei Personen mit Herzproblemen mit
75 Jahren\footnote{Statistischer Mittelwert} zum Einsatz kommen. Durch einen Herzschrittmacher werden
Medikamente überflüssig, die ohne Herzschrittmacher eingenommen werden müssten.
In bestimmten Fällen wird auch ein implantierbarer Kardioverter-Defibrillator implantiert.

Ein Herzschrittmacher besteht aus der Hauptkomponente und der Elektrode. Die Hauptkomponente vereint
Impulsgeber, Steuerelektronik und eine Energiequelle in einem Gehäuse. Die Elektrode dient sowohl zur
Stimulation des Herzens als auch zur Messung der Herzfunktion. Aktuelle Herzschrittmacher überwacht
permanent das Herz und setzt nur noch ein, wenn innerhalb einer einstellbaren Zeitdauer kein
Herzschlag erkannt wird. Die Konfiguration kann von einem Arzt drahtlos erfolgen.

Als Energiequelle dient heute eine Lithiumbatterie. Damit sind Laufzeiten von durchschnittlich acht
Jahren möglich. Eine Aufladung über Induktion ist heute nicht mehr gebräuchlich, stattdessen wird der
Herzschrittmacher einfach ausgetauscht.

\subsection{Implantate nach dem Tod}
Die Entfernung eines Implantates ist nur mit Zustimmung des Verstorbenen oder seiner Angehörigen
möglich. Das Implantat wird also in der Regel auch nach dem Tod im Körper bleiben. In den 70er Jahren
wurden teils Herzschrittmacher aus den Verstorbenen entfernt. Diese wurden dann, nach einer
Aufarbeitung, wieder in Patienten eingesetzt. Dabei wurde verschwiegen, dass es sich um gebrauchte
Geräte handelte.\footcite{Spiegel:Herzschrittmacher-Prozess}


\section{Eigene Versuche -- ein sechster Sinn}
\label{sec:Robin:experiments}
In diesem Kapitel beschreibe ich eigene Versuche, die ich zur Erforschung der Erweiterbarkeit
des Menschen und der Anpassungsfähigkeit des Gehirns durchgeführt habe.


\newcommand{\Linkfeelspace}%
{\enquote{\href{http://feelspace.cogsci.uni-osnabrueck.de/}{Feel Space}}}
\subsection{Haptischer Kompass-Gürtel}
Die Universität Osnabrück entwickelte unter dem Projektname \Linkfeelspace{}
einen Gürtel, der über Vibration dem Träger die Himmelsrichtung mitteilt.
Dazu werden zwölf beziehungsweise 30 Vibrationsmotoren mit einem Gürtel um den Bauch angebracht.
Der Träger weiß dadurch, wo Norden ist,
ohne auf einen Kompass schauen zu müssen. Diese Erweiterung der Positionsbestimmung des Menschen
wird mit der Zeit unterbewusst vom Gehirn zur Navigation mit einbezogen. So berichtet Udo Wächter
nach dem Tragen des Gürtels über sechs Wochen, dass er sich besser in Osnabrück orientieren konnte.
Wächters Gehirn hat also eine neue Verbindung zwischen Reizung der
Bauchnerven\footnote{beziehungsweise Rückennerven} und der
Himmelsrichtung hergestellt.\footcite{noz:compass_belt}
Somit stand Wächter, nach einer Lernphase an die nun fühlbare Himmelsrichtung, ein neuer
Sinn für die Zeit des Experiments zur Verfügung.

\subsubsection{Nachbau eines Kompass-Gürtels}
\label{sec:Robin:experiments:myCompassBelt}
Da ich mir die Wirksamkeit dieses \enquote{sechsten Sinn} sehr gut vorstellen konnte, wollte
ich diesen auch selbst testen. Deshalb entschloss ich mich, eine einfache Variante dieses Gürtels
mit nur zwei Vibrationsmotoren nachzubauen. Da zwei Motoren allerdings fast schon das Minimum
darstellen und
damit noch keine wirkliche Orientierung möglich war, rüstete ich später noch auf vier Vibrationsmotoren
auf.
Ein weiterer Grund für die Entscheidung zum Nachbau war,
dass ich die meisten Komponenten und auch das nötige Wissen bereits besaß. Ein solcher Nachbau wurde
schon von ein paar technikinteressierten Menschen umgesetzt, allerdings fand ich keine Anleitung.
Deshalb überlegte ich mir selbst eine Schaltung.

Begonnen habe ich mit einem
\href{http://www.mikrokopter.de/ucwiki/MK3Mag}{elektronischen Kompass}%
\footnote{Nummer 1 auf Abb.~\vref{fig:Technik_compass_belt}}
vom \href{http://www.mikrokopter.de}{MikroKopter Projekt}.
Dieser gibt unter anderem eine Gradangabe aus. Daraus geht die Blickrichtung relativ zum Nordpol
hervor. Ich entschied mich für einen PWM-Ausgang am Kompassmodul, um die Gradangabe in einem
externen Mikrokontroller\footnote{\href{http://www.atmel.com/Images/doc8161.pdf}{ATmega328}, das
rechteckige schwarze Bauteil bei Nummer 2 auf Abb.~\vref{fig:Technik_compass_belt}}
weiter zu verarbeiten. Den Mikrokontroller lötete ich auf eine
Lochstreifenplatine. Eine direkte Verarbeitung der Gradangabe und die Ansteuerung der Motoren mit
dem Mikrokontroller des Kompassmoduls schloss ich aus, um mehr Steuerausgänge zur Verfügung zu
haben\footnote{und um nicht auf der Platine löten zu müssen. Zum Beispiel das direkte abgreifen
von Signalen am SMD-Mikrokontroller.}.

\begin{wrapfigure}{r}{0pt}
	\includegraphics[width=5cm]{files/images/Robin/compass_belt/IMG_7994-rm-bg\imageresize}
	\captionof{figure}{Komponenten meines Kompass-Gürtels}
	\label{fig:Technik_compass_belt}
\end{wrapfigure}
Den externen Mikrokontroller programmierte ich so, dass dieser einen von vier Steuerausgängen
für 150 Millisekunden anschaltet, wenn Norden entweder vorne, rechts, hinten oder links ist.
Um dieses schwache\footnote{Die maximale Belastbarkeit ist mit \SI{40}{\milli\ampere} pro Pin
angegeben.}
Steuersignal zum Schalten eines Vibrationsmotors verwenden zu können, verstärkte ich es über eine
Transistorschaltung. Diese baute ich auf Lochraster (\hyperref[fig:Technik_compass_belt]{Nr. 3})
auf. Die Vibrationsmotoren (\hyperref[fig:Technik_compass_belt]{Nr. 4}) baute ich aus alten
Handys
aus und verkabelte sie mit der Transistorschaltung. Ich opferte anfangs nur zwei Handys, da ich
bei einem Erfolg weitere Vibrationsmotoren sehr günstig
\href{http://www.pollin.de/shop/dt/NDA2OTg2OTk-/Motoren/%
Gleichstrommotoren/Vibrationsmotor_LA4_432A.html}{nachbestellen}
könnte.
Um die Vibrationsmotoren ohne Handy zu betreiben, brauchte ich noch ein Gehäuse, da sich sonst der Motor
nicht drehen kann. Bei den zwei ersten Handys schnitt ich einfach den Teil\footnote{Der Vibrationsmotor
sitzt meist ganz unten im Handy in der Nähe des Mikrofons, außer bei einem Motorola V300 (Klapphandy). Bei
diesem war er im unteren Teil des Display Gehäuses verbaut.}, wo der Motor verbaut war, aus dem Gehäuse
raus und hatte so schon passende Halterungen, die genug Platz für die Unwucht boten. Die Vibrationsmotoren
klebte ich mit Heißkleber in die Halterungen. Als ich dann noch zwei alte Handys %von einem guten Freund
bekam, entschied ich mich dafür doch keine Vibrationsmotoren zu bestellen.

\begin{wrapfigure}{r}{0pt}
	\includegraphics[width=5.1cm]{files/images/Robin/compass_belt/IMG_8070-rm-bg\imageresize}
	\captionof{figure}{Vibrationsmotoren mit Holzgehäuse}
	\label{fig:motors_compass_belt}
\end{wrapfigure}
Außerdem beschloss ich,
nicht mehr das Handygehäuse zu zerschneiden, um eine passende Hülle für die
Motoren zu bekommen, sondern selbst zwei aus Holz zu fertigen. Ich bohrte Löcher mit dem Durchmesser
der
Motoren in kleine Hartholzstücke und brachte diese dann in die Form von Quadern. Anschließend klebte ich
die Vibrationsmotoren im Holz fest. Dabei war darauf zu achten, dass sich die Unwucht weiterhin drehen
konnte.

Somit standen mir vier Vibrationsmotoren zur Verfügung.
Die komplette Technik brachte ich zusammen mit einem Akkupack
an einem Gürtel an. Da das Kompassmodul keine Beschleunigungssensoren zur Verfügung hat und somit
nicht weiß, wo \enquote{unten} ist, musste ich bei der Befestigung am Gürtel, auf eine waggerechte
Lage achten. Um ständiges Vibrieren zu vermeiden, änderte ich das Programm, sodass
nur eine Vibration an den Träger weiter gegeben wird, wenn sich dieser zur nächsten Position gedreht
hat.

Das erste Problem, was mir auffiel war, dass der Bauch sehr viel empfindlicher ist als der Rücken.
Dadurch kitzelte es am Bauch schon, während das Vibrieren am Rücken kaum wahrnehmbar war. Dieses
Problem löste ich, indem ich die Vibrationsstärke, über unterschiedliche Widerstände
vor den Basen der Transistoren, änderte.

Mein Kompass-Gürtel ist auf Abb.~\vref{fig:My_compass_belt} zu sehen. Der Bau hat mich circa 31
Stunden gekostet. Der Materialwert liegt ungefähr bei \EUR{80}.
\href{http://www.cogsci.uni-osnabrueck.de/NBP/peterhome.html}%
{Prof. Dr. Peter König} schrieb mir
in einer eMail, dass neben vielen hundert Arbeitsstunden Entwicklung, der Materialwert
von ihrer Entwicklung noch vierstellig ist.
Es gibt aber aktuell Bemühungen diesen Preis zu drücken.
Mein Nachbau ist also um einiges einfacher als die Entwicklungen des \Linkfeelspace{} Projekts.

Außerdem schrieb er, dass eines der größten Probleme der hohe Stromverbrauch war. Deshalb überprüfte
ich meine eigene Schaltung diesbezüglich. Ich stellte fest, dass die Schaltung im Betrieb
\SI{44}{\milli\ampere} bei \SI{5,35}{\volt} verbrauchte. Ein laufender Vibrationsmotor verbraucht
hingegen schon \SI{80}{\milli\ampere}\footnote{Ein sehr kleiner Vibrationsmotor verbraucht sogar
\SI{130}{\milli\ampere}. Dieser höhere Verbrauch ist notwendig, da man die Vibration wegen der
kleineren Unwucht sonst nicht spürt.}.
Es sieht also gar nicht \emph{so} schlecht aus. Nach meinen Berechnungen lässt sich der Gürtel
gute 14 Stunden\footnote{Unter der Annahme, dass sich ein Motor ständig dreht und ein Akkupack mit
\SI{2}{\ampere\hour} zum Einsatz kommt.} betreiben.
Soviel zur Theorie. In der Praxis hatte ich ebenfalls Probleme mit der Stromversorgung, da die Schaltung
unterhalb von \SI{4,9}{\volt} nicht mehr zuverlässig funktioniert.
Dieses Problem löste ich mit einem Spannungswandler und einem Akkupack mit mehr Zellen.

\begin{figurewrapper}
	\includegraphics[width=0.7\hsize]{files/images/Robin/compass_belt/IMG_8009-rm-bg\imageresize}
	\captionof{figure}{Nachbau des Kompass-Gürtels}
	\label{fig:My_compass_belt}
\end{figurewrapper}

\subsubsection{Erfahrungen}
Erste Tests bestätigten meine Vermutung, dass sich mit einem solchen Gürtel die Orientierung nach
einer gewissen Zeit verbessern könnte. Hierfür sollten dann aber mindestens vier Vibrationsmotoren
zur Verfügung stehen.

Nach längeren Tests ist mir aufgefallen, dass ich immer wusste, wo Norden ist, teils auch unbewusst.
Mit der aktuellen Uhrzeit wusste ich auch immer, wo sich die Sonne befand. Dies ist, sowohl in
Gebäuden interessant, wie auch draußen.\footnote{Langzeiterfahrungen kann ich leider nicht
präsentieren, da der Gürtel erst zwei Wochen vor Abgabe der Dokumentation fertig wurde.}

Nebenbei machte ich die Erfahrung, dass die Personen, die mich auf den Gürtel ansprachen, zuerst an
einen Bombengürtel dachten~\dots

\subsection{Magnet im Finger}
\begin{wrapfigure}{r}{0pt} %% Einbindung steht im Konflikt mit dem Urheberrecht
	\href{\URLQuinnMagnet}{\includegraphics[width=5.6cm]%
		{files/images/Robin/magnet/Quinn_Norton/Magnet_am_Finger-rm-bg}%
	}
	\captionof{figure}[Magnete hängen am Ringfinger von Quinn Norton]%
	{Magnete hängen am Ringfinger von Quinn Norton\footnotemark}%
	\label{fig:Quinn_Norton_magnet}
\end{wrapfigure}
\footnotetext{Hintergrund entfernt und zugeschnitten.
Das Original ist hier zu finden: \url{\URLQuinnMagnet}}

Die US-Journalistin \href{http://quinnnorton.com/}{Quinn Norton} begann 2005, sich für funktionale
Körperveränderung zu interessieren.
Im September ließ sie sich einen kleinen Magneten in die Spitze des rechten
Ringfingers implantieren, da hier sehr viele Nerven vorhanden sind. Der Magnet bewegt sich angeregt
durch elektromagnetische Felder in ihrem Finger. In der Nähe von elektrischen Leitungen nahm sie ein
Vibrieren wahr. Bei Sicherheitsschranken in Geschäften hatte sie sogar Schmerzen, da diese offenbar
mit hohen Leistungen arbeiten. Die Fingerspitze ist so empfindlich, dass sie sogar die Festplatten
Aktivität, genauer das Anlaufen einer Festplatte, spüren konnte.

Für diese Zeit hatte sie einen sechsten Sinn. Nach ein paar Monaten trat allerdings eine Infektion
auf. Die Silikon-Schutzschicht ist offenbar beschädigt worden und der Magnet wurde von ihrem Körper
angegriffen. Ihr Arzt versuchte den zerbröselnden Magneten herauszupullen, aber dies gelang ihm
nicht. Nach diesem Zwischenfall hatte sie ihr Gespür für elektromagnetische Felder verloren. Aber
Magnete konnte sie immer noch mit ihrem Ringfinger halten
(siehe Abb.~\vref{fig:Quinn_Norton_magnet}). 2007 wurde der Magnet dann komplett entfernt. Sie ist
einerseits glücklich, dass der Magnet entfernt wurde, aber andererseits vermisst sie den
sechsten Sinn. Auch wenn ihre Experimente mit dem Magnet im Finger mit viel Schmerzen verbunden
waren, sagte sie schon, dass wenn das Problem mit der Silikon-Schutzschicht gelöst ist, wird sie
sich wieder einen Magneten implantieren lassen.\footcite{23C3:body_hacking,
mindhacks:magnet_removed}

\subsubsection{Magnet an meinem Finger}
\label{sec:Robin:experiments:myMagnet}
Um ihre Erkenntnisse nachzuvollziehen, überlegte ich mir, wie ich einen Magneten an meinem
Ringfinger anbringen konnte, ohne diesen gleich implantieren zu lassen.

Zuerst versuchte ich, verschiedene Magneten an meinem Ringfinger mit Klebeband zu befestigen. Dies
brachte aber keine Ergebnisse, da hierdurch die Durchblutung verschlechtert wird und man deswegen
weniger Gefühl im Finger hat. Außerdem spürt man den Puls zu stark.

\begin{wrapfigure}{r}{0pt}
	\includegraphics[width=6cm]%
		{files/images/Robin/magnet/Magnet_auf_dem_Finger-rm-bg-cut\imageresize}
	\captionof{figure}{Mein Ringfinger mit Magnet (befestigt mit Sekundenkleber)}
	\label{fig:My_magnet}
\end{wrapfigure}

Es war also nötig mir einen Magnet auf den Fingernagel des rechten Ringfingers zu kleben. Ich
benutze einen normalen Alleskleber, den ich zuerst auf den Fingernagel gab und dann einen
quadratischen sehr starken Magneten mit einem Volumen von \SI{125}{\cubic\milli\metre}
und einem Gewicht von \SI{0,9}{\gram} auf den
Fingernagel setzte. Dann fixierte ich den Magnet mit Klebeband. Das Aushärten des Klebers fühlte
sich etwas merkwürdig an. Das Erste, was mir auffiel, war das zusätzliche Gewicht. Ich ließ den
Kleber eine Stunde aushärten allerdings war der Kleber danach immer noch nicht fest. Als ich eine
Schere benutze, um das Klebeband durchzuschneiden, blieb der Magnet am Metall der Schere hängen und
löste sich vom Fingernagel.

Als Nächstes benutzte ich Sekundenkleber. Dies funktionierte deutlich besser
(siehe Abb.~\vref{fig:My_magnet}).
Direkt nach dem
Aufsetzen des Magneten auf den, mit Kleber bedeckten, Finger haftete der Magnet am Fingernagel.
Zusätzliches Klebeband war nicht nötig. Im Gegensatz zum Alleskleber spürte ich beim Aushärten vom
Sekundenkleber nichts.\footnote{Auf den Tuben der Flüssigkleber steht \enquote{kann
allergische Reaktionen auslösen} allerdings hatte ich damit keine Probleme. Es gibt
auch Hautkleber. Dieser ist vermutlich besser geeignet.} Auch konnte ich wären dem Aushärten
diesen Text schreiben. Nach circa 40 Minuten war der Kleber dann gut durchgehärtet.

\subsubsection{Tests und Erfahrungen}
Als Erstes ging ich mit diesem \enquote{neuen Sinn} an das Steckdosenkabel meines
Computers.\footnote{Der Computer hat mit TFT-Monitor circa einen Verbrauch von \SI{150}{\watt}.} Ab
\SI{3}{\milli\metre} Distanz zwischen Magnet und Kabel spürte ich ein leichtes Zittern am
Fingernagel. Bei ausgeschalteten Verbrauchern war hingegen kein Zittern wahrnehmbar. Bei
Trafonetzteilen lässt sich sogar ohne das daran eine Last angeschlossen ist ein deutliches Zittern
aus mehr als \SI{1}{\centi\metre} fühlen. Bei Schaltnetzteilen spürt man ohne Last gar nichts. Erst
mit Verbraucher hat man die Möglichkeit wieder das Zittern zu spüren. Dies liegt natürlich an dem
unterschiedlichen technischen Aufbau. Bei dem Zittern handelt es sich um die \SI{50}{\hertz}
Netzfrequenz. Der fließende Strom baut um die Leitungen ein Magnetfeld auf.

Mit einem Magneten am Finger lassen sich also magnetische Felder wahrnehmen. Man kann somit natürlich
auch magnetische Objekte ertasten.

Bei Festplatten habe ich allerdings nicht viel gespürt. Das bleibt wohl denen vorbehalten, die
einen Magnet unter der Haut haben.

An das zusätzliche Gewicht des Magnets gewöhnt man sich recht schnell. Der Magnet fällt mir nur noch
durch die Trägheitskraft auf.

Die geklebte Verbindung zwischen Fingernagel und Manget ist wirklich sehr gut. Der Magnet hält
Metalle und starke Magnete bis zu einem Gewicht von \SI{600}{\gram}. Mehr hält der Magnet nicht. Der
Kleber hält hingegen mindestens \SI{1}{\kilo\gram}.
Man kann kräftig an dem Magnet ziehen und er rührt sich nicht.
Mehr habe ich nicht getestet, da ab diesem
Gewicht Schmerzen auftreten.

Nach zwei Tagen mit dem Magnet auf meinem Finger begann ich zu überlegen, wie ich den Magnet wieder
entfernen konnte. Ich versuchte es mit einem Seitenschneider. Dies funktionierte auch nach ein paar
Minuten. Ich ging zwischen Fingernagel und Magnet und löste so den Magnet. Die Klebereste ließen
sich größtenteils mit einem Teppichmesser und mit einem Fingernagelknipser entfernen.
Kleinere Klebereste lassen sich dann mit einer Feile oder einer
Nass-Schleifmaschine\footnote{Nass-Schleifmaschine wegen der geringen Drehzahl. Eine normale
Schleifmaschine ist nicht empfehlenswert.} abbekommen.

Mit diesem Experiment konnte ich zumindest etwas nachvollziehen, was Quinn Norton mit einem Magnet
in der Ringfingerspitze berichtete. Obwohl der Fingernagel um einiges unempfindlicher sein dürfe als
das Innere der Fingerspitze.


\section{Blick in die Zukunft}
\label{sec:Robin:future}

In diesem Kapitel beschreibe ich meine Visionen, was in Zukunft alles möglich werden könnte. Der Blick in
die nahe Zukunft basiert zum Teil auch auf heutigen Forschungen oder absehbaren Entwicklungen.

\subsection{Prothese}
Aktuell wird an Prothesen geforscht, die nicht nur Bewegungen ausführen, die direkt von den Nerven
abgegriffen werden, sondern auch Rückmeldungen liefern. Also hauptsächlich Druck-, Temperatur-, und
Schmerzempfinden. Die Schnittstelle zwischen zentralem Nervensystem und Prothese könnten in Zukunft
Glasfasern darstellen. Die Schnittstelle soll mit Messfühlern funktionieren, die ihre
Form\footnote{Dadurch wird vermutlich die Wellenlänge des Lichts geändert, dass durch die Glasfasern
geführt wird.} abhängig vom elektrischen Feld ändern. Diese Signale werden von einem Mikroprozessor
ausgewertet und in Steuersignale für eine Prothese umgewandelt.
Informationen von der Prothese werden über Infrarotstrahlung über Glasfasern an die Nerven geleitet.
Diese Entwicklung befindet sich allerdings noch in den
Anfängen.\footcite{Spektrum:Weg_zu_intelligenten_Prothesen}

Durch diese Entwicklung wird es in bald Prothesen geben, die den Organellen ebenbürtig sind. In der
Prothetik steckt aber auch das Potenzial bessere Prothesen zu entwickeln als das Original. Vorteile
die sich daraus ergeben sind beispielsweise die hohe Kraft, die eine solche Prothese aufbringen kann.

Im Bereich der Armprothetik sehe ich außerdem besonders den Vorteil der Spezialisierung.\footnote{Der
menschliche Körper ist sehr anpassungsfähig das heißt aber auch, dass er für viele Anwendungen nicht
optimal ist.} Ich könnte mir vorstellen, dass sich Armprothesen austauschen lassen und so immer ein
optimales Werkzeug zur Verfügung steht.

Bei Beinprothesen wird mehr der Leistungsaspekt eine Rolle spielen.


\subsection{Dem Körper Energie entziehen}
Jedes Implantat, das elektronische Komponenten besitzt, benötigt Energie. Um nicht in regelmäßigen
Zeitabständen durch Nachladen, Auswechseln, Induktion oder Infrarotstrahlung diese Energie
bereitzustellen zu müssen, wird es meiner Ansicht nach dazu kommen, die Energie direkt dem Körper zu
entziehen. Zum Beispiel über die Körperwärme\footnote{Es wird ein Temperaturunterschied benötigt,
dann lässt sich zum Beispiel mit einem
\href{http://de.wikipedia.org/wiki/Peltier-Element}{Peltierelement} Strom erzeugen.
Das Peltierelement sollte also optimalerweise auf einem Muskel in der Nähe der Haut implantiert
werden.},
Bewegungsenergie\footnote{Ähnlich einem Automatik-Uhrwerk, das mit Schwungmasse funktioniert.
Die zweite Möglichkeit, die ich sehe, ist die Nutzung eines
\href{http://de.wikipedia.org/wiki/Piezoelement}{Piezoelements} beispielsweise unter der
Fußsohle um durch die Druckänderung elektrische Energie zu generieren.}%, Solarzelle
%% Dem Körper Energie entziehen
oder später auch die Energie, die in chemischer Form
durch das Blut
transportiert wird. Da die benötigte Energie bei den meisten Implantaten so gering ist, werden wir
dies nicht einmal merken. Damit wäre das Energieproblem für Implantate (eventuell auch für Prothesen)
endlich gelöst.
Diese Technik wird für jedes Implantat zum Einsatz kommen.

\subsection{Licht Implantat}
%\begin{comment}
\begin{wrapfigure}{r}{0pt}
	\href{http://de.wikipedia.org/w/index.php?title=Datei:Lampyris_noctiluca.jpg&filetimestamp=20050618231628}{%
		\includegraphics[width=4.4cm]{files/images/Robin/Lampyris_noctiluca-cut-rot}%
	}
	\captionof{figure}[Ein weibliches Glühwürmchen]%
		{Ein weibliches Glühwürmchen\footnotemark}%
	\label{fig:Firefly}
\end{wrapfigure}
\footnotetext{Es handelt sich, um genau zu sein, um ein Weibchen des Großen Leuchtkäfers (Lampyris
noctiluca).
Bild zugeschnitten und gedreht.
Das Original, wie auch meine Abwandlung, stehen unter
\href{http://creativecommons.org/licenses/by-sa/2.0/de/deed.de}{Creative Commons BY-SA 2.0}.
Das Bild stammt vom Benutzer \href{http://de.wikipedia.org/wiki/Benutzer:Wofl}{Wofl}.}
%\end{comment}

Ein leicht umsetzbares Implantat ist meiner Ansicht nach die Nachrüstung eines
\href{http://de.wikipedia.org/wiki/Leuchtorgane}{Leuchtorgans}, wie von
\href{http://de.wikipedia.org/wiki/Leuchtkäfer}{Glühwürmchen} bekannt, zum Beispiel an der Hand. Ein
solches Leuchtorgan, wie es bei Tieren\footnote{Hauptsächlich bei Tiefsee Meerestieren.}
vorkommt, hat einen
extrem hohen Wirkungsgrad von bis zu 95\Prozent. Die Energie könnte direkt dem Blutkreislauf entzogen
werden. So könnte ich mir, in den nächsten 15 Jahren, eine Lampe
vorstellen, die sich in einen Finger implantieren lässt und beispielsweise durch eine ungewöhnliche
Bewegung des Fingers an- und ausgeschaltet wird.\footnote{Interessanterweise habe ich zu dieser Idee
keine Quellen im Internet gefunden, die darauf näher eingehen. Die Idee scheint als noch recht
unbekannt zu sein.}

\subsection{Hörverbesserung} %% used with \nameref
\label{sec:Robin:future:hearing}
Die heutigen Hörgeräte sind schon erstaunlich gut.
In naher Zukunft wird es möglich werden auch Menschen mit vollständig defektem Innenohrs zu helfen.
Aktuell wird am Hirnstammimplantat geforscht, das direkt an den Nerven ansetzt.
Wir werden bald einen Zeitpunkt erreicht haben, ab dem Hörgeräte nicht nur die Hörleistung eines gesunden
Menschen erreichen, sondern dies noch übertreffen.
Ein solches Hörgerät könnte uns etwa dazu befähigen, ein breiteres Frequenzband
wahrzunehmen. Beispielsweise \href{http://de.wikipedia.org/wiki/Infraschall}{Infraschall}, der
besonders unter Wasser zur Kommunikation geeignet ist\footnote{Blauwale nutzen Infraschall, also
Schall unter \SI{16}{\hertz}, zu diesem Zweck.}. Aber auch
\href{http://de.wikipedia.org/wiki/Ultraschall}{Ultraschall}\footnote{Schall oberhalb von
\SI{16}{\kilo\hertz}.} bietet einiges Potenzial. So könnte ich mir vorstellen, dass sich mit einem
Ultraschallsender und mit der Schalltrichterwirkung der Ohrmuschel die Umgebung genauer abtasten
lässt. Sowohl Infraschall als auch Ultraschall können von unserem jetzigen Gehör quasi nicht
wahrgenommen werden. Weiterhin könnte ich mir vorstellen, dass ein solches Hörgerät eine
Aufzeichnungsfunktion hat, sodass man sich alles noch einmal anhören kann. Man könnte auch
verschiedene Filter auf das Signal anwenden, Rauschen herausfiltern und laute Töne automatisch leiser
zu machen. Die digitale Signalverarbeitung bietet heute schon einige Möglichkeiten.
Es ließen sich mit einem Hörgerät auch die meisten Einschränkungen aufheben, die durch die
\href{http://de.wikipedia.org/wiki/Psychoakustik}{Psychoakustik} gefunden wurden. Zum Beispiel hören
wir leise Geräusche nicht, wenn sie einige Millisekunden nach einem lauten Geräusche auftreten.

Wenn man diesen Gedanken weiterdenkt, kommt man zu dem Schluss, dass wir noch in diesem Jahrhundert
Hörgeräte haben werden, die als Handy Freisprecheinrichtung funktionieren und alle
\phantomsection\label{sec:Robin:future:hearing:Babel_Fish}%
Sprachen in die
eigene Muttersprache übersetzen\footnote{Der
\href{http://de.wikipedia.org/wiki/Babelfisch}{Babelfisch} aus dem Buch \citetitle
{Per_Anhalter_durch_die_Galaxis} wird Realität.} können. Alles außer der Übersetzungsfunktion lässt
sich meiner Ansicht nach bereits
heute mit überschaubarem Entwicklungsaufwand umsetzen. Ein Problem was für eine optimale Benutzung
noch gelöst werden muss ist aber eine praktikable Steuerung des Geräts
(\siehe{sec:Robin:future:BCI}).

\subsection{Sehverbesserung}
Mit dem noch experimentellen Retinaimplantat ist es bereits heute möglich, einfache Bilder direkt
an den Sehnerv zu übertragen. Besonders im subretinalen Implantat sehe ich ein großes Potenzial. Es
wird zu einer erheblichen Verbesserung des Bildes kommen, sodass mit einem Sehimplantat dann auch in
Farbe gesehen werden kann. Es wird also noch eine Weile dauern, bis ein künstliches Auge unser
biologisches Auge übertreffen wird. Aber wenn wir an diesem Punkt sind, stehen uns noch mehr
Möglichkeiten offen, diesen Sinn zu erweitern, als beim Hörsinn. Die Entwicklung wird also nicht
Enden.

Ich könnte mir vorstellen, dass sich der Sehsinn so um eine Einblendung verschiedener Informationen
erweitern lässt. Diese Entwicklung ist heute schon durch den Bereich erweiterter Realität absehbar.
Dabei geht es darum, die Wahrnehmung der (meist visuellen) Realität computergestützt zu erweitern. So
gibt es schon heute Systeme, die Objekte erkennen und über eine Videobrille zusätzliche Informationen
zu diesen mit ins aktuelle Bild einblenden. Die Möglichkeiten, die damit geboten werden, wurden vor
einigen Monaten durch die \enquote{Google-Brille} wieder ins Bewusstsein
gerückt. Mit dieser sollen sich Funktionen wie Navigation, Videos und Terminerinnerungen über eine
Brille direkt ins Blickfeld einblenden lassen.\footcite{Heise:TR:Project_Glass}
Eine solche Brille\footnote{oder später auch als
Kontaktlinse} könnte in einigen Jahren auf den Markt kommen und den Handys Konkurrenz machen. Dies
hätte den Vorteil, dass gewisse Informationen ständig im Blickfeld sind.\footnote{Beispielsweise
bekannt aus den Terminator Filmen und Serien.} Man könnte Telemetriedaten vom eigenen
Körper\footnote{die über entsprechende Implantate erfasst werden}
hierüber Anzeigen. Beispielweise Blutzuckerspiegel, Herzschlag, freigesetzte Hormone, vorhandene
Energiereserven, Temperaturen, Sauerstoffgehalt im Blut,
\href{http://de.wikipedia.org/wiki/Elektrokardiogramm}{Elektrokardiogramm}\footnote{%
Ein Diagramm jeder elektrische Erregung voraus, die im
Normalfall vom Sinusknoten ausgeht. Durch diese elektrische Erregung wird der Herzschlag
kontrolliert. Ein EKG ist in \hyperref[sec:Robin:topical:Pacemaker]{heutigen Herzschrittmachern}
und implantierbaren Kardioverter-Defibrillatoren
bereits integriert.} und so weiter.
Der Hauptanwendungsfall wird aber dem eines
Smartphones\footnote{\href{http://de.wikipedia.org/wiki/Smartphone}{Wikipedia}: Ein Mobiltelefon, das
mehr Computerfunktionalität und -konnektivität als ein herkömmliches fortschrittliches Mobiltelefon
zur Verfügung stellt.} nahe kommen. Es wird aber auch viele Programme geben, die die Umgebung
analysieren und Wichtiges hervorheben und zusätzliche Informationen zur Umgebung einblenden. Als
Analogie zum Hörgerät, \hyperref[sec:Robin:future:hearing:Babel_Fish]{das jede Sprache in die
Muttersprache übersetzt}, könnte es Programme geben die Texte übersetzt einblenden oder auch falsch
geschriebene Wörter signalisieren aber auch welche die Bedienungsanleitungen und Arbeitsabläufe
anzeigen.

Weitergehen könnte es dann, indem diese Technik soweit miniaturisiert wird, um ins Auge zu
passen.\footnote{Eventuell direkt in Verbindung mit einer \nameref{sec:Robin:future:hearing} um auch
eine Audioschnittstelle zum Gehirn zu haben.
} Dadurch wird die Brille überflüssig und alles, was man
sieht, wird über einen Computer vorverarbeitet. Die Möglichkeiten, die sich daraus ergeben, sind sehr
vielseitig, da jetzt nicht nur Einblendungen machbar sind, sondern direkt auf das wahrnehmbare Bild
Einfluss genommen werden kann. Es könnte als ebenfalls\footnote{Wie auch schon bei der
\nameref{sec:Robin:future:hearing} beschrieben.} eine Aufzeichnungs- und Abspielfunktion möglich
werden. Auch Zeitlupe und das Einfrieren des Bildes sind durchaus denkbar. Weiterhin bietet eine
Videobearbeitung zum Beispiel die Möglichkeit Bilder zu entzerren oder auf optische Täuschung
hinzuweisen, die unser Gehirn eventuell nicht erkennen würde. Es wird sicher auch viele Programme
geben, die die Wirklichkeit verändert darstellen. Beispielsweise eine Darstellung der Realität als
\href{http://www.tony5m17h.net/MatrixCode.gif}{Matrixcode}
aus dem Film \href{http://www.imdb.de/title/tt0234215/}{Matrix Reloaded (2003)}.

\begin{figurewrapper} %% Einbindung steht im Konflikt mit dem Urheberrecht
	\includegraphics[width=0.7\hsize]{files/images/Robin/Matrixcode\imageresize}
	\captionof{figure}{Matrixcode aus Matrix Reloaded (2003)}
	\label{fig:Matrixcode}
\end{figurewrapper}

%\begin{comment}
\begin{wrapfigure}{r}{0pt}
	\href%
	{http://de.wikipedia.org/w/index.php?title=Datei:Infrared_dog.jpg&filetimestamp=20050107180457}{%
		\includegraphics[width=5cm]{files/images/Robin/Infrared_dog}%
	}
	\captionof{figure}{Bild eines Hundes im Infrarotspektrum}%
	\label{fig:Infrared_dog}
\end{wrapfigure}
%\end{comment}

Ein weiterer denkbarer Schritt ist die Verbesserung der Kamera um ein breites Frequenzband aus dem
elektromagnetischen Spektrum zu sehen. Interessant ist hier beispielsweise die Infrarotstrahlung
beziehungsweise Wärmestrahlung. Man könnte wie eine Wärmebildkamera die Umgebung sehen und
Hitzequellen erkennen. Außerdem kann man die Kamera hinsichtlich der Auflösung optimieren, um dann
darüber eine Vergrößerung zu ermöglichen.%\footnote{Der Monokel der von Uhrmachern getragen wird,
%wandert direkt ins Auge.}

\subsection{Gehirn-Computer-Schnittstelle}
\label{sec:Robin:future:BCI}
Das für mich interessantes Thema in diesem Kapitel ist die Gehirn-Computer-Schnittstelle
. Damit wird
es möglich werden, Informationen beziehungsweise Gedanken direkt mit dem Computer auszutauschen.
Praktische Anwendungsfälle sind beispielsweise stark körperlich behinderte Menschen, die weder
Sprechen noch ihre Hände zum Schreiben benutzen können. Diesen Menschen wird mit einer
Gehirn-Computer-Schnittstelle die Möglichkeit eröffnet, wieder mit der Außenwelt zu kommunizieren.
Daran wird derzeit intensiv geforscht.

Die Gehirn-Computer-Schnittstelle bildet die Grundlage für viele weitere Anwendungen. Beispielsweise
die direkte Steuerung von Maschinen, Implantat und Prothesen.

Durch die Gehirn-Computer-Schnittstelle werden aber auch Gedankenexperimente wie
\href{http://de.wikipedia.org/wiki/Gehirn_im_Tank}{Gehirn im Tank} machbar. Dabei geht es darum, dass
ein Wissenschaftler ein Gehirn an einen Computer anschließt und den Computer einen menschlichen
Körper und die dazugehörige Umwelt simulieren lässt. Für das Gehirn sieht es also so aus, also ob es
einen realen Körper kontrolliert. Mit einer Gehirn-Computer-Schnittstelle wird es meiner Ansicht nach
möglich werden, sich in simulierte Welten einzuklinken und dabei den physikalischen Körper zu
vergessen. Diese Idee wurde zum Beispiel durch den recht erfolgreichen Film Matrix (1999)
eindrucksvoll demontiert.

An der Stelle könnte man endlos weiter philosophieren. Kurzum, die Gehirn-Computer-Schnittstelle
bietet nahe unbegrenzte Möglichkeiten.

\subsection{Probleme und Risiken}
\label{sec:Robin:future:problems}
Ob man sich eines der beschriebenen erweiternden Technologien implantieren lassen möchte, hängt
natürlich von der Umsetzung ab. Beispielsweise inwieweit man sich auf die Sicherheit eines solchen
Gerätes verlassen kann. Da Schadprogramme auch problemlos auf frei programmierbaren Implantaten
ausführbar sind. Dies könnte passieren, wenn Schwachstellen im System gefunden werden. Dieses Problem
wird umso schlimmer je wichtiger das Implantat zum Überleben ist. Es gibt schon heute Probleme mit
der Sicherheit von Implantaten.\footcite{Heise:Pacemaker_Hacker, MIT:Protect_implants_from_attack}
Computerspezialisten der University of Washington und Massachusetts ist es experimentell gelungen,
Herzschrittmacher und implantierbare Defibrillatoren über dessen Funkschnittstelle zu kontrollieren.
Darüber wird es im schlimmsten Falle möglich eine Person mithilfe der lebensnotwendigen Implantate zu
töten. Um allerdings eine Kommunikation zum Implantat aufzubauen, muss der Sender sehr nah am
Implantat sein.
Die unautorisierte Manipulation solcher Geräte über die Funkschnittstelle war eigentlich voraussehbar
und hätte verhindert werden müssen.

Weiterhin können ethische Gründe eine Rolle bei der Entscheidung spielen. Es besteht auch die
Möglichkeit, dass durch die Optimierungen einzelner Menschen eine Zweiklassengesellschaft noch
verstärkt wird.

Dabei stellt sich dann auch die Frage, \emph{was noch fair ist?} Was würde beispielsweise passieren,
wenn in
naher Zukunft eine Hörverbesserung, mit allen \hyperref[sec:Robin:future:hearing]{von mir
beschriebenen Funktionen}, auf den Markt kommt für einen Preis von vielleicht \EUR{20000}. Es würde
dazu führen, dass die jetzt schon gut gestellten Menschen noch Mächtiger werden. Um dies zu
verhindern, müsste jeder zum Beispiel mit erreichen der Volljährigkeit entscheiden können, ob er sich
diese Verbesserung kostenlos implantieren lassen möchte. Also eine Erweiterung des momentan noch
Diskutierten bedingungslosen Grundeinkommens.

\begin{comment}
Eine weitere Frage ist, bis wann wir noch als Menschen zählen?
„Ist man noch ein Mensch, wenn man den Teil im Gehirn abgeschaltet hat, der für Schuldgefühle
zuständig ist?“\footcite{23C3:body_hacking}
\end{comment}

Es könnte auch zu gezielter Manipulation der Implantatträger kommen. Da die Implantate, die ich
beschrieben habe, ja zwischen Sinnesorganen und Gehirn sitzen, könnte der Träger vollständig
manipuliert werden. Um dies zu verhindern, muss meiner Ansicht nach die Entwicklung solcher
Implantate komplett offen ablaufen.

Gerechtigkeit ist damit vermutlich immer noch nicht gewährleistet. Ich könnte mir zum Beispiel
vorstellen, dass es verschiedene Generationen der Implantate geben wird. Dadurch würden Menschen mit
neueren Implantaten bevorzugt werden.

Auch im Bereich der Nanotechnologie gibt es noch einige ungeklärte Fragen. Beispielsweise ob diese
neue Technologie überhaupt angewendet werden darf, da die Risiken (noch) unabschätzbar sind:
\enquote{[\dots] [Es] ließen sich (auch) Nanobots bauen, die trillionenfach in der Atmosphäre
herumschwirren, bestimmte Personen an ihrem genetischen Muster erkennen und töten
können.}\footcite[13]{Heise:Telepolis:Mensch:Unsterblichkeit}

Es wird bereits deutlich, dass ich diese Frage nicht beantworten kann. Fakt ist aber, dass für solche
Probleme eine Lösung gefunden werden muss. Da sich die technische Entwicklung nur schwer aufhalten
lässt. Wir werden meiner Ansicht nach mit den Verbesserungen, die ich mir in diesem Kapitel
vorgestellt habe, in Zukunft konfrontiert werden.

Wir uns somit immer mehr zum kybernetischer
Organismus\footnote{\href{http://de.wikipedia.org/wiki/Cyborg}{Wikipedia}: Mischwesen aus lebendigem
Organismus und Maschine. Die gängige Abkürzung ist Cyborg.} weiterentwickeln.
Das wird aber nicht das Ende sein. Ich halte es für wahrscheinlich, dass \enquote{wir} in ferner
Zukunft nur noch als ein kollektives Bewusstsein
existieren werden.\footcite{Heise:Telepolis:Mensch:globales_Gehirn}




\nocite{
	Spektrum:Weg_zu_intelligenten_Prothesen,
	thesis:Cyborg,
	Stern:pacemaker,
	Heise:Telepolis:Mensch:Cyborg,
}



%\clearpage
\printbibliography[heading=source,keyword=Robin]
