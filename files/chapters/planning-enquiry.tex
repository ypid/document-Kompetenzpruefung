\pdfbookmark[0]{Planung und Recherche}{sec:planning-enquiry}
\chapter*{Planung und Recherche}
\label{sec:planning-enquiry}
Alles ab hier wird nicht gedruckt, es dient zur Planung.

\section*{Robin}
Im Folgenden beschreibe ich die Unterthemen, die ich innerhalb unserer Gruppe für die
Kompetenzprüfung genauer ausarbeiten möchte.

\begin{description}
	\item[Dinge die ich nicht behandelt habe:]~ \vspace{-0.2cm}
\begin{multicols}{2}
\begin{itemizewrapper}
	\item Organe züchten
	\item Rollstuhl, Krücken
	\item Exoskelett
	\item \href{http://de.wikipedia.org/wiki/Parasiten_des_Menschen}{Parasiten des Menschen}
\end{itemizewrapper}
\end{multicols}
\end{description}

Zur Exoevolution: Der Gegensatz synthetische Biologie und Androiden, KI

\subsection*{Links}
Nach Priorität geordnet. Eine kleinere Zahl bedeutet eine höhere Relevanz für meinen Teil.
Nach Übernahme ins Quellenverzeichnis wird der Eintrag in dieser Liste in aller Regel entfernt,
außer der Eintrag hier enthält zusätzliche Informationen.
\begin{enumerate}
	\item Wikipedia: \href{http://de.wikipedia.org/wiki/Prothese}{Prothese},
		\href{http://de.wikipedia.org/wiki/Transhumanismus}{Transhumanismus},
		\href{http://de.wikipedia.org/wiki/Posthumanismus}{Posthumanismus}
	\item Audio Podcast \enquote{CR176 Moderne Maschinenmenschen -- Digitale Erweiterungen analoger
		Zellhaufen}
		(\href{https://wiki.chaosradio.ccc.de/Chaosradio_176}{Beschreibung im internen Wiki})
	\item \href{https://www.youtube.com/watch?v=EjcSosdtQ0k}{Arm Prothese mit Gefühl},
		\href{https://www.youtube.com/watch?v=KVDsO9NAgtE}{Moderne Prothetik},
		\href{https://www.youtube.com/watch?v=7bow1_k2LIg}{Doku-Reihe \enquote{Leben mit Prothese}}
		(Ich bin mit Folge 3 fertig)
	\item Planet-Wissen
		\href{http://www.planet-wissen.de/natur_technik/anatomie_mensch/prothesen/index.jsp}%
		{Prothesen}
	\item \enquote{Mensch mit Wissen überfordert – Computerchip implantieren?}
		(\href{http://www.heise.de/newsticker/meldung/Mensch-mit-Wissen-ueberfordert-Computerchip-implantieren-1405096.html}
		{Artikel}) \\
		Stichwörter: Wissensflut mit Gehirnimplantate in den Griff bekommen
	\item \enquote{Eine Prothese fürs Gehirn}
		(%\href{http://www.heise.de/newsticker/meldung/Hirnchip-funktioniert-im-Tiermodell-1365151.html}
		%{Artikel} und
		\href{http://heise.de/-1365074}{Artikel}) \\
		Stichwörter: Versuch an den Gehirnen von Laborratten.
			Zum besseren Verständnis, später Behandlung von Hirnschäden.
	\item BBC Dokumentation \enquote{Der menschliche Körper}
		(\href{http://www.imdb.de/title/tt1929678/}{deutsche IMDb}) \\
		E03: \enquote{Wie unser Gehirn entsteht}: Blinder nimmt Bilder über die Zunge war.
			(ab 42:36) \\
		E04: \enquote{Wie wir die Umwelt besiegen}: Durch einen Autobrand erleidet ein Mann
			schwer Verbrennungen. Seine Hände sind verstümmelt.
			Er entscheidet sich dazu, seine Hände durch die von einem kürzlich Verstorbenen ersetzen
			zu lassen. (ab 37:24)
	\item Video Podcast \enquote{Körper voller Elektronik}
		(ab Minute 7:30, \href{http://www.elektrischer-reporter.de/phase3/video/252}%
		{Beschreibung}) \\
		Stichwörter: Ausblick in eine mögliche Zukunft von kybernetischen Organismen.
	\item \enquote{Geist in der Maschine}
		(\href{http://www.heise.de/tr/artikel/Geist-in-der-Maschine-277969.html}{Artikel}) \\
		Stichwörter: Simulation des Gehirns, Intelligenz
	\item Video Podcast \enquote{Uploads des menschlichen Gehirns}
		(ab Minute 8:07, \href{http://www.elektrischer-reporter.de/phase3/video/267/}%
		{Beschreibung}) \\
		Stichwörter: Ausblick in eine mögliche Zukunft: Das Gehirn lässt sich kopieren und
			Simulieren.
	\item Audio Podcast \enquote{CRE143 Biohacking}
		(\href{http://cre.fm/cre143}{Beschreibung}) \\
		Stichwörter: Einführung in die (synthetische) Biologie und Zellenlehre,
			Analysieren der eigenen DNS, eigene Organismen programmieren,
			\href{http://partsregistry.org/Catalog}{Registry of Standard Biological Parts}
	\item Arte Dokumentation \enquote{Nanotechnologie -- Die unsichtbare Revolution}
		(\href{http://www.arte.tv/de/6338036.html}{Webseite},
		\href{https://www.youtube.com/watch?v=UoYLd4jUT5k}{YouTube}) \\
		Stichwörter \href{http://www.youtube.com/watch?v=TFLNWHsw4ro}{E01}:
			Optimierung der medizinischen Behandlung, Unsterblichkeit, Posthumanismus,
			kybernetische Organismen, Blicke in mögliche Zukünfte \\
		Stichwörter \href{http://www.youtube.com/watch?v=O2q6LfjSt_Y}{E02}:
			Photovoltaik, Wasseraufbereitung, sauberere Abgase von Dieselmotoren,
			Reinigung von kontaminierter Erde, Risiken (ab Minute 35) \\
		Stichwörter \href{http://www.youtube.com/watch?v=ATo85_43QdI}{E03}:
			Alltag: Rastertunnelmikroskopie, Nano Glasscheiben-Versiegelung,
			Kohlenstoffnanoröhren;
			Elektronik (ab Minute 16): RFID, Sensoren; totale Überwachung (Zukünfte)
			Informatik (ab Minute 28): mehr Rechenleistung, KI, menschliches Gehirn;
			Verantwortung der Wissenschaftler
	\item Film \enquote{Matrix}
		(Erscheinungsjahr 1999, \href{http://de.wikipedia.org/wiki/Matrix_(Film)}{Wikipedia},
		\href{http://www.imdb.de/title/tt0133093/}{deutsche IMDb}) \\
		Stichwörter: Science-Fiction, perfektes Brain-Computer-Interface,
		Was ist Realität?, virtuelle Realität
	\item Film \enquote{The 6th Day}
		(Erscheinungsjahr 2000, \href{http://de.wikipedia.org/wiki/The_6th_Day}{Wikipedia},
		\href{http://www.imdb.de/title/tt0216216/}{deutsche IMDb}) \\
		Stichwörter: Science-Fiction, Klonen von Organen, Tieren und Menschen,
			Augen als Brain-Computer-Interface
	\item Film \enquote{Die Insel}
		(Erscheinungsjahr 2005, \href{http://de.wikipedia.org/wiki/Die_Insel_(2005)}{Wikipedia},
		\href{http://www.imdb.de/title/tt0399201/}{deutsche IMDb}) \\
		Stichwörter: Science-Fiction, Klonen von Menschen um Organe zu züchten
			(ist etwas unrealistisch)
	\item Film \enquote{Gamer}
		(Erscheinungsjahr 2009, \href{http://de.wikipedia.org/wiki/Gamer}{Wikipedia},
		\href{http://www.imdb.de/title/tt1034032/}{deutsche IMDb}) \\
		Stichwörter: Science-Fiction, Gehirnimplantate, Nanotechnologie,
			Kontrolle/Fernsteuerung von Menschen,
			zum Tode verurteilte Menschen als Avatare
\end{enumerate}

\bigskip
Hier noch ein paar weiterführende Empfehlungen, die ich nicht in meiner Dokumentation
berücksichtigen konnte, da ich zu spät darauf aufmerksam wurde~\dots
\begin{enumerate}
	\item \fullcite{Eschbach:Black_Out}
	\item \fullcite{Eschbach:Hide_Out}
\end{enumerate}



\subsection*{Links zu dem Themengebiet von Joshuah}
\begin{enumerate}
	\item Arte Dokumentation \enquote{Das automatische Gehirn}
		(\href{http://www.arte.tv/de/4308796.html}{Webseite},
		\href{http://www.youtube.com/watch?v=fP2Czgyu6Dc}{YouTube}) \\
		Stichwörter: Viel Neurowissenschaft, Funktionsweise von Illusionen, Unterbewusstsein, Liebe
	\item Vortrag \enquote{Die psychologischen Grundlagen des Social
		Engineerings} (\href{http://events.ccc.de/camp/2011/Fahrplan/events/4478.en.html}{Beschreibung},
		\href{http://media.ccc.de/browse/conferences/camp2011/cccamp11-4478-die_psychologischen_grundlagen_des_social_engineerings-de.html}
		{Vortrag}, \href{http://www.kaishakunin.com/social-engineering/index.html}{Homepage}) \\
		Stichwörter: Grundlagen, Beispiele
	\item Vortrag \enquote{A Beginner's Guide to Social Engineering}
		(Deutsch, \href{http://www.mitternachtshacking.de/blog/%
			1157-eh2010-a-beginners-guide-to-social-engineering}%
		{Beschreibung},
		\href{http://media.ccc.de/browse/conferences/eh2010/EH2010-3765-de-socialengineering.html}
		{Vortrag})
		\\ Stichwörter: Grundlagen, Beispiele
	\item Podcast über Social Engineering
		(Englisch, \href{http://www.2600.com/offthehook/2011/0511.html}{Beschreibung},
		\href{http://www.2600.com/offthehook/mp3files/2011/off_the_hook__20110504.mp3}{Audio}) \\
		Stichwörter: Viele Beispiele, Kevin Mitnick
	\item Vortrag \enquote{Die fünf Pforten der Manipulation}
		(\href{http://events.ccc.de/congress/2002/fahrplan/event/453.de.html}{Beschreibung},
		\href{ftp://ftp.ccc.de/congress/2002/video/19C3-453-die-fuenf-pforten-der-manipulation.mp4}{Videodatei}
		die auf der Webseite funktioniert nicht~\dots) \\
		Ich hab den Vortrag vor circa \CalcAge[print-year-suffix=true]{2008}{03}{04} %% guess
		gesehen und war vor allem von der Manipulation der Parasiten bei Tieren beeindruckt.
		Der Vortrag besteht aber hauptsächlich aus der psychischen Manipulation
		und fällt somit in Joshuahs Themengebiet. \\
		Stichwörter: Manipulation von Tieren und Menschen, viele Beispiele (aus Werbung, Adam und Eva),
			Wie funktioniert Manipulation?, \enquote{Manipulation} ist negativ belegt
\end{enumerate}

\section*{Joshuah}
\begin{enumerate}
	\item \url{http://www.nachdenkseiten.de/?p=2368}
	\item \url{http://www.youtube.com/user/ingoakablaze}
	\item \url{http://www.youtube.com/user/WeThePeople082}
	\item \url{http://www.youtube.com/watch?v=h0IWYvVGC10}
	\item \url{http://www.youtube.com/watch?v=sDNsof92Na4}
	\item \url{http://www.youtube.com/watch?v=90H1IUyCi9M}
	\item \url{http://www.youtube.com/watch?v=29C_KNZ7RNs}
	\item \url{http://www.youtube.com/watch?v=kqmljQM3VNM}
	\item \url{http://www.youtube.com/watch?v=LRk8k1VhciQ}
	\item \url{http://vimeo.com/26628923}
	\item \url{http://www.petersdurchblick.com}
	\item \url{http://www.kannnichtsein.com}
	\item \url{http://whaaat.de/truewords}
	\item \url{http://www.true-words.eu}
	\item \url{http://whaaat.de/truewords/?p=3918}
	\item \url{http://www.movie2k.to/Waking-Life-watch-movie-774544.html}
\end{enumerate}
